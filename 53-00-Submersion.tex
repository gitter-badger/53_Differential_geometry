\documentclass[12pt]{article}
\usepackage{pmmeta}
\pmcanonicalname{Submersion}
\pmcreated{2013-03-22 15:28:49}
\pmmodified{2013-03-22 15:28:49}
\pmowner{pbruin}{1001}
\pmmodifier{pbruin}{1001}
\pmtitle{submersion}
\pmrecord{4}{37335}
\pmprivacy{1}
\pmauthor{pbruin}{1001}
\pmtype{Definition}
\pmcomment{trigger rebuild}
\pmclassification{msc}{53-00}
\pmclassification{msc}{57R50}
\pmrelated{Immersion}

% this is the default PlanetMath preamble.  as your knowledge
% of TeX increases, you will probably want to edit this, but
% it should be fine as is for beginners.

% almost certainly you want these
\usepackage{amssymb}
\usepackage{amsmath}
\usepackage{amsfonts}

% used for TeXing text within eps files
%\usepackage{psfrag}
% need this for including graphics (\includegraphics)
%\usepackage{graphicx}
% for neatly defining theorems and propositions
%\usepackage{amsthm}
% making logically defined graphics
%%%\usepackage{xypic}

% there are many more packages, add them here as you need them

% define commands here
\begin{document}
A differentiable map $f\colon X\to Y$ \PMlinkescapetext{between} differential manifolds $X$ and $Y$ is called a \emph{submersion at a point} $x\in X$ if the tangent map
$$
\mathrm{T}f(x)\colon\mathrm{T}X(x)\to\mathrm{T}Y(f(x))
$$
between the tangent spaces of $X$ and $Y$ at $x$ and $f(x)$ is surjective.

If $f$ is a submersion at every point of $X$, then $f$ is called a \emph{submersion}.  A submersion $f\colon X\to Y$ is an open mapping, and its image is an open submanifold of $Y$.

A fibre bundle $p\colon X\to B$ over a manifold $B$ is an example of a submersion.
%%%%%
%%%%%
\end{document}
