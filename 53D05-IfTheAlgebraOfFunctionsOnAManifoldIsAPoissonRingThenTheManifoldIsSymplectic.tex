\documentclass[12pt]{article}
\usepackage{pmmeta}
\pmcanonicalname{IfTheAlgebraOfFunctionsOnAManifoldIsAPoissonRingThenTheManifoldIsSymplectic}
\pmcreated{2013-03-22 14:46:34}
\pmmodified{2013-03-22 14:46:34}
\pmowner{rspuzio}{6075}
\pmmodifier{rspuzio}{6075}
\pmtitle{if the algebra of functions on a manifold is a Poisson ring then the manifold is symplectic}
\pmrecord{18}{36422}
\pmprivacy{1}
\pmauthor{rspuzio}{6075}
\pmtype{Theorem}
\pmcomment{trigger rebuild}
\pmclassification{msc}{53D05}

% this is the default PlanetMath preamble.  as your knowledge
% of TeX increases, you will probably want to edit this, but
% it should be fine as is for beginners.

% almost certainly you want these
\usepackage{amssymb}
\usepackage{amsmath}
\usepackage{amsfonts}

% used for TeXing text within eps files
%\usepackage{psfrag}
% need this for including graphics (\includegraphics)
%\usepackage{graphicx}
% for neatly defining theorems and propositions
%\usepackage{amsthm}
% making logically defined graphics
%%%\usepackage{xypic}

% there are many more packages, add them here as you need them

% define commands here
\begin{document}
Let $M$ be a smooth manifold and let $A$ be the algebra of smooth functions from $M$ to $\mathbb{R}$.  Suppose that there exists a bilinear operation $[,] \colon A \times A \to A$ which makes $A$ a Poisson ring.

For this proof, we shall use the fact that $T^*(M)$ is the sheafification of the $A$-module generated by the set $\{ df | f \in A \}$ modulo the relations
\begin{itemize}
\item $d(f+g) = df + dg$
\item $dfg = g \, df + f \, dg$
\end{itemize}

Let us define a map $\omega \colon T^*(M) \to T(M)$ by the following conditions:
\begin{itemize}
\item $\omega (df) (g) = [f,g]$ for all $fg \in A$
\item $\omega (fX + gY) = f \omega(X) + g \omega(Y)$ for all $f,g \in A$ and all $X,Y \in T^*(M)$
\end{itemize}
For this map to be well-defined, it must respect the relations:
 $$\omega (f+g)(h) = [f+g,h] = [f,h] + [g,h] = \omega (f)(h) +  \omega (g)(h)$$
 $$\omega (fg)(h) = [fg,h] = f[g,h] + g[f,h] = f \omega (g)(h) + g \omega (g)(h)$$
These two equations show that $\omega$ is a well-defined map from the presheaf hence, by general nonsense, a well defined map from the sheaf.  The fact that $\omega (f \, dg)$ is a derivation readily follows from the fact that $[,]$ is a derivation in each slot.

Since $[,]$ is non-degenerate, $\omega$ is invertible.  Denote its inverse by $\Omega$.  Since our manifold is finite-dimensional, we may naturally regard $\Omega$ as an element of $T^*(M) \otimes T^*(M)$.  The fact that $\Omega$ is an antisymmetric tensor field (in other words, a 2-form) follows from the fact that $\Omega (df)(g) = [f,g] = -[g,f] = - \Omega (dg)(f)$.

Finally, we will use the Jacobi identity to show that $\Omega$ is
closed.  If $u,v,w \in T(M)$ then, by a general identity of
differential geometry,
 \[\langle d \Omega, u \wedge v \wedge w \rangle = \langle u, d
 \langle \Omega, v \wedge w \rangle \rangle + \langle v, d \langle
 \Omega, w \wedge u \rangle \rangle + \langle w, d \langle \Omega, u
 \wedge v \rangle \rangle\]
Since this identity is trilinear in $u,v,w$, we can restrict attention
to a generating set.  Because of the non-degeneracy assumption, vector
fields of the form $ad_f$ where $f$ is a function form such a set. 

By the definition of $\Omega$, we have $\langle \Omega, ad_f
\wedge ad_g \rangle = [f,g]$.  Then $\langle ad_f, d\,
\langle \Omega, ad_g \wedge ad_h \rangle = [f,[g,h]]$
so the Jacobi identity is satisfied.
%%%%%
%%%%%
\end{document}
