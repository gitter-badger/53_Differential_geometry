\documentclass[12pt]{article}
\usepackage{pmmeta}
\pmcanonicalname{OrderOfContact}
\pmcreated{2013-03-22 16:59:49}
\pmmodified{2013-03-22 16:59:49}
\pmowner{rspuzio}{6075}
\pmmodifier{rspuzio}{6075}
\pmtitle{order of contact}
\pmrecord{4}{39278}
\pmprivacy{1}
\pmauthor{rspuzio}{6075}
\pmtype{Definition}
\pmcomment{trigger rebuild}
\pmclassification{msc}{53A04}
\pmsynonym{order contact}{OrderOfContact}

% this is the default PlanetMath preamble.  as your knowledge
% of TeX increases, you will probably want to edit this, but
% it should be fine as is for beginners.

% almost certainly you want these
\usepackage{amssymb}
\usepackage{amsmath}
\usepackage{amsfonts}

% used for TeXing text within eps files
%\usepackage{psfrag}
% need this for including graphics (\includegraphics)
%\usepackage{graphicx}
% for neatly defining theorems and propositions
%\usepackage{amsthm}
% making logically defined graphics
%%%\usepackage{xypic}

% there are many more packages, add them here as you need them

% define commands here

\begin{document}
Suppose that $A$ and $B$ are smooth curves in $\mathbb{R}^n$ which pass through 
a common point $P$.  We say that $A$ and $B$ have zeroth order contact if their
tangents at $P$ are distinct.

Suppose that $A$ and $B$ are tangent at $P$.  We may then set up a coordinate
system in which $P$ is the origin and the $x_1$ axis is tangent to both curves.
By the implicit function theorem, there will be a neighborhood of $P$ such that
$A$ can be described parametrically as $x_i = f_i (x_1)$ with $i = 2, \ldots, n$
and $B$ can be described parametrically as $x_i = g_i (x_1)$ with 
$i = 2, \ldots, n$.  We then define the \emph{order of contact} of $A$ and $B$ 
at $P$ to be the largest integer $m$ such that all partial derivatives of $f_i$ 
and $g_i$ of order not greater than $m$ at $P$ are equal.

%%%%%
%%%%%
\end{document}
