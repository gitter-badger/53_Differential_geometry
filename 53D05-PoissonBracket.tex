\documentclass[12pt]{article}
\usepackage{pmmeta}
\pmcanonicalname{PoissonBracket}
\pmcreated{2013-03-22 14:46:04}
\pmmodified{2013-03-22 14:46:04}
\pmowner{rspuzio}{6075}
\pmmodifier{rspuzio}{6075}
\pmtitle{Poisson bracket}
\pmrecord{11}{36412}
\pmprivacy{1}
\pmauthor{rspuzio}{6075}
\pmtype{Definition}
\pmcomment{trigger rebuild}
\pmclassification{msc}{53D05}
\pmrelated{Quantization}
\pmrelated{CanonicalQuantization}

\endmetadata

% this is the default PlanetMath preamble.  as your knowledge
% of TeX increases, you will probably want to edit this, but
% it should be fine as is for beginners.

% almost certainly you want these
\usepackage{amssymb}
\usepackage{amsmath}
\usepackage{amsfonts}

% used for TeXing text within eps files
%\usepackage{psfrag}
% need this for including graphics (\includegraphics)
%\usepackage{graphicx}
% for neatly defining theorems and propositions
%\usepackage{amsthm}
% making logically defined graphics
%%%\usepackage{xypic}

% there are many more packages, add them here as you need them

% define commands here
\begin{document}
Let $M$ be a symplectic manifold with symplectic form $\Omega$.  The \emph{Poisson bracket} is a bilinear operation on the set of differentiable functions on $M$.  In terms of local Darboux coordinates $p_1, \ldots, p_n, q_1, \ldots, q_n$, the Poisson bracket of two functions is defined as follows:
 $$[f,g] = \sum_{i=1}^n {\partial f \over \partial q_i} {\partial g \over \partial p_i} - {\partial f \over \partial p_i} {\partial g \over \partial q_i}$$
It can be shown that the value of $[f,g]$ does not depend on the choice of Darboux coordinates.  Therefore, the Poisson bracket is a well-defined operation on the symplectic manifold.  Also, some authors use a different sign convention --- what they call $[f,g]$ is what would be referred to as $-[f,g]$ here.

The Poisson bracket can be defined without reference to a special coordinate system as follows:
 $$[f,g] = \Omega^{-1} (df, dg) = \sum_{i=1}^{2n} \Omega^{ij} {\partial f \over \partial x_i} {\partial g \over \partial x_j}$$
Here $\Omega^{-1}$ is the inverse of the symplectic form, and its components in an arbitrary coordinate system are denoted $\Omega^{ij}$.

The Poisson bracket sastisfies several important algebraic identities.  It is antisymmetric:
 $$[f,g] = -[g,f]$$
It is a derivation:
 $$[fg,h] = f[g,h] + g[f,h]$$
It satisfies Jacobi's identitity:
 $$[f,[g,h]] + [g,[h,f]] + [h,[f,g]] = 0$$

The Hamilton equations can be expressed elegantly in terms of the Poisson bracket.  If $X$ is a smooth function on $M$, we can describe the time-evolution of $X$ by the equation
 $${dX \over dt} = [X,H]$$
If $X$ is a smooth function on $\mathbb{R} \times M$, we can describe the time-evolution of $X$ by the more general equation
 $${dX \over dt} = {\partial X \over \partial t} - [X,H]$$
%%%%%
%%%%%
\end{document}
