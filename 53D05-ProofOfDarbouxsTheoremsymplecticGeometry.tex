\documentclass[12pt]{article}
\usepackage{pmmeta}
\pmcanonicalname{ProofOfDarbouxsTheoremsymplecticGeometry}
\pmcreated{2013-03-22 14:09:55}
\pmmodified{2013-03-22 14:09:55}
\pmowner{rspuzio}{6075}
\pmmodifier{rspuzio}{6075}
\pmtitle{proof of Darboux's theorem (symplectic geometry)}
\pmrecord{8}{35589}
\pmprivacy{1}
\pmauthor{rspuzio}{6075}
\pmtype{Proof}
\pmcomment{trigger rebuild}
\pmclassification{msc}{53D05}
%\pmkeywords{Moser trick}

\endmetadata

% this is the default PlanetMath preamble.  as your knowledge
% of TeX increases, you will probably want to edit this, but
% it should be fine as is for beginners.

% almost certainly you want these
\usepackage{amssymb}
\usepackage{amsmath}
\usepackage{amsfonts}

% used for TeXing text within eps files
%\usepackage{psfrag}
% need this for including graphics (\includegraphics)
%\usepackage{graphicx}
% for neatly defining theorems and propositions
%\usepackage{amsthm}
% making logically defined graphics
%%%\usepackage{xypic}

% there are many more packages, add them here as you need them

% define commands here

\newcommand{\R}{\mathbb{R}^{2n}}
\begin{document}
We first observe that it suffices to prove the theorem for symplectic
forms defined on an open neighbourhood of $0 \in \R$.

Indeed, if we have a symplectic manifold $(M, \eta)$, and a point
$x_0$, we can take a (smooth) coordinate chart about $x_0$.  We can then
use the coordinate function to push $\eta$ forward to a symplectic form
$\omega$ on a neighbourhood of $0$ in $\R$.  If the result holds on $\R$,
we can compose the coordinate chart with the resulting symplectomorphism to
get the theorem in general.

Let $\omega_0 = \sum_{i=1}^{n} dx_i \wedge dy_i$.  Our goal is then to find a (local) diffeomorphism $\Psi$ so that $\Psi(0) = 0$ and $\Psi^*\omega_0 = \omega$.

Now, we recall that $\omega$ is a non--degenerate two--form.  Thus, on
$T_{0} \R$, it is a non--degenerate anti--symmetric bilinear form.  By a linear change of basis, it can be put in the standard form.  So, we
may assume that $\omega(0) = \omega_0(0)$.

We will now proceed by the ``Moser trick''.  Our goal is to find a
diffeomorphism $\Psi$ so that $\Psi(0) = 0$ and $\Psi^*\omega =
\omega_0$.
We will obtain this diffeomorphism as the time--$1$ map of the
flow of an ordinary differential equation.  We will see this as the
result of a deformation of $\omega_0$.

Let $\omega_t = t \omega_0 + (1-t) \omega$.
Let $\Psi_{t}$ be the time $t$ map of the differential equation
 $$\frac{d}{dt} \Psi_t(x) = X_t(\Psi_t(x))$$
in which $X_t$ is a vector field determined by a condition to be stated later.  

We will make the ansatz $$\Psi_{t}^* \omega = \omega_t.$$

Now, we differentiate this \PMlinkescapetext{identity}:
$$ 0 = \frac{d}{dt} \Psi_t^*\omega_t = \Psi_t^*( L_{X_t}\omega_t +
\frac{d}{dt} \omega_t). $$

($L_{X_t}\omega_t$ denotes the Lie derivative of $\omega_t$ with respect
to the vector field $X_t$.)

By applying Cartan's identity and recalling that $\omega$ is closed, we
obtain :
$$ 0 = \Psi_t^*(d \iota_{X_t} \omega_t + \omega - \omega_0) $$

Now, $\omega - \omega_0$ is closed, and hence, by Poincar\'e's Lemma,
locally exact.	So, we can write $\omega - \omega_0 = - d\lambda$.

Thus $$ 0 = \Psi_t^*( d ( i_{X_t} \omega_t - \lambda )) $$

We want to require then $$ i_{X_t} \omega_t = \lambda. $$  Now, we observe that
$\omega_0 = \omega$ at $0$, so $\omega_t = \omega_0$ at $0$.  Then, as
$\omega_0$ is non--degenerate, $\omega_t$ will be non--degenerate on an
open neighbourhood of $0$.  Thus, on this neighbourhood, we may use this to
define $X_t$ (uniquely!).

We also observe that $X_t(0) = 0$.  Thus, by choosing a sufficiently
small neighbourhood of $0$, the flow of $X_t$ will be defined for time
greater than $1$.

All that remains now is to check that this resulting flow has the desired
properties.  This follows merely by reading our \PMlinkescapetext{derivation} of the ODE,
backwards.
%%%%%
%%%%%
\end{document}
