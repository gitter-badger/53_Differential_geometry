\documentclass[12pt]{article}
\usepackage{pmmeta}
\pmcanonicalname{NotesOnTheClassicalDefinitionOfAManifold}
\pmcreated{2013-03-22 14:14:47}
\pmmodified{2013-03-22 14:14:47}
\pmowner{rmilson}{146}
\pmmodifier{rmilson}{146}
\pmtitle{notes on the classical definition of a manifold}
\pmrecord{11}{35691}
\pmprivacy{1}
\pmauthor{rmilson}{146}
\pmtype{Topic}
\pmcomment{trigger rebuild}
\pmclassification{msc}{53-03}
\pmrelated{Manifold}

% this is the default PlanetMath preamble.  as your knowledge
% of TeX increases, you will probably want to edit this, but
% it should be fine as is for beginners.

% almost certainly you want these
\usepackage{amssymb}
\usepackage{amsmath}
\usepackage{amsfonts}

% used for TeXing text within eps files
%\usepackage{psfrag}
% need this for including graphics (\includegraphics)
%\usepackage{graphicx}
% for neatly defining theorems and propositions
%\usepackage{amsthm}
% making logically defined graphics
%%%\usepackage{xypic}

% there are many more packages, add them here as you need them

% define commands here

\newcommand{\sR}[0]{\mathbb{R}}
\newcommand{\sC}[0]{\mathbb{C}}
\newcommand{\sN}[0]{\mathbb{N}}
\newcommand{\sZ}[0]{\mathbb{Z}}

 \usepackage{bbm}
 \newcommand{\Z}{\mathbbmss{Z}}
 \newcommand{\C}{\mathbbmss{C}}
 \newcommand{\R}{\mathbbmss{R}}
 \newcommand{\Q}{\mathbbmss{Q}}



\newcommand*{\norm}[1]{\lVert #1 \rVert}
\newcommand*{\abs}[1]{| #1 |}

 \newcommand{\reals}{\mathbb{R}}
 \newcommand{\natnums}{\mathbb{N}}
 \newcommand{\cnums}{\mathbb{C}}
 
 \newcommand{\lp}{\left(}
 \newcommand{\rp}{\right)}
 \newcommand{\lb}{\left[}
 \newcommand{\rb}{\right]}
 
 
 \newcommand{\cA}{\mathcal{A}}
 \newcommand{\cC}{\mathcal{C}}
 \newcommand{\cT}{\mathcal{T}}

\begin{document}
\PMlinkescapeword{types}
\PMlinkescapeword{information}
\PMlinkescapeword{relation}
\PMlinkescapeword{restricted} 
\PMlinkescapeword{order}
\PMlinkescapeword{theory}
\PMlinkescapeword{structure}

\subsubsection*{Classical Definition} 
Historically, the data for a
 manifold was specified as a collection of coordinate domains related
 by changes of coordinates. The manifold itself could be obtained by
 gluing the domains in accordance with the transition functions,
 provided the changes of coordinates were free of inconsistencies.
 
 In this formulation, a $\cC^k$ manifold is specified by two types of
 information. The first item of information is a collection of open
 sets
 $$V_\alpha\subset\reals^n,\quad \alpha\in \cA,$$
 indexed by some set
 $\cA$. The second item is a collection of transition functions, that
 is to say $\cC^k$ diffeomorphisms
 $$\sigma_{\alpha\beta}:V_{\alpha\beta}\rightarrow \reals^n,\quad
 V_{\alpha\beta}\subset
 V_\alpha,\;\text{open},\quad\alpha,\beta\in\cA,$$
 obeying certain
 consistency and topological conditions.
 
 
 We call a pair
 $$(\alpha,x),\quad \alpha\in \cA,\; x\in V_\alpha$$
 the coordinates of
 a point relative to chart $\alpha$, and define the manifold $M$
 to be the set of equivalence classes of such pairs modulo the relation
 $$(\alpha,x)\simeq (\beta,\sigma_{\alpha\beta}(x)).$$
 To ensure that the above is an
 equivalence relation we impose the following hypotheses.
 \begin{itemize}
 \item For $\alpha\in \cA$, the transition function
 $\sigma_{\alpha\alpha}$ is the identity on $V_\alpha$.
 \item For $\alpha,\beta\in \cA$ the transition functions
 $\sigma_{\alpha\beta}$ and $\sigma_{\beta\alpha}$ are inverses.
 \item For $\alpha,\beta,\gamma\in \cA$ we have for a suitably
 restricted domain
 $$\sigma_{\beta\gamma}\circ\sigma_{\alpha\beta} =
 \sigma_{\alpha\gamma}$$
 \end{itemize}
 We topologize $M$ with the least coarse topology that will make
 the
 mappings from each $V_\alpha$ to $M$ continuous. Finally, we
 demand
 that the resulting topological space be paracompact and Hausdorff.
 
 \subsubsection{Notes} 
To understand the role played by the notion of a
 differential manifold, one has to go back to classical differential
 geometry, which dealt with geometric objects such as curves and
 surface only in reference to some ambient geometric setting ---
 typically a 2-dimensional plane or 3-dimensional space. Roughly
 speaking, the concept of a manifold was created in order to treat the
 intrinsic geometry of such an object, independent of any embedding.
 The motivation for a theory of intrinsic geometry can be seen in
 results such as Gauss's famous Theorema Egregium, that showed that a
 certain geometric property of a surface, namely the scalar
 curvature, was fully determined by intrinsic metric properties of the
 surface, and was independent of any particular embedding. Riemann 
\cite{riemann}
 took this idea further in his habilitation lecture by describing
 intrinsic metric geometry of $n$-dimensional space without recourse to
 an ambient Euclidean setting. The modern notion of manifold, as a
 general setting for geometry involving differential properties evolved
 early in the twentieth century from works of mathematicians such as
 Hermann Weyl \cite{weyl}, who introduced the ideas of an atlas and transition
 functions, and Elie Cartan, who investigation global properties and
 geometric structures on differential manifolds. The modern definition
 of a manifold was introduced by Hassler Whitney \cite{whitney}
 (For more foundational information, follow \PMlinkexternal{this link}{http://web.archive.org/web/20041010165022/http://www.math.uchicago.edu/~mfrank/founddiffgeom3.html} to some old notes by \PMlinkexternal{Matthew Frank}{http://web.archive.org/web/20040511092724/www.math.uchicago.edu/~mfrank/} ).
 
\begin{thebibliography}{9}
 \bibitem{riemann}
 Riemann, B., ``\"Uber die Hypothesen welche der Geometrie zu
 Grunde liegen
 (On the hypotheses that lie at the foundations of geometry)'' in
 M. Spivak, {\em A comprehensive introduction to differential
 geometry}, vol. II.
 \bibitem{spivak} Spivak, M., {\em A comprehensive introduction to
 differential  geometry}, vols I \& II.
 \bibitem{weyl} Weyl, H., {\em The concept of a Riemann surface}, 1913
 \bibitem{whitney} Whitney, H., \emph{Differentiable Manifolds}, Annals of
 Mathematics, 1936.
 \end{thebibliography}

%%%%%
%%%%%
\end{document}
