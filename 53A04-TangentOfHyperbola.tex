\documentclass[12pt]{article}
\usepackage{pmmeta}
\pmcanonicalname{TangentOfHyperbola}
\pmcreated{2013-03-22 19:10:31}
\pmmodified{2013-03-22 19:10:31}
\pmowner{pahio}{2872}
\pmmodifier{pahio}{2872}
\pmtitle{tangent of hyperbola}
\pmrecord{12}{42083}
\pmprivacy{1}
\pmauthor{pahio}{2872}
\pmtype{Derivation}
\pmcomment{trigger rebuild}
\pmclassification{msc}{53A04}
\pmclassification{msc}{51N20}
\pmclassification{msc}{51-00}
%\pmkeywords{tangent line}
\pmrelated{Slope}
\pmrelated{TangentOfConicSection}

\endmetadata

% this is the default PlanetMath preamble.  as your knowledge
% of TeX increases, you will probably want to edit this, but
% it should be fine as is for beginners.

% almost certainly you want these
\usepackage{amssymb}
\usepackage{amsmath}
\usepackage{amsfonts}

% used for TeXing text within eps files
%\usepackage{psfrag}
% need this for including graphics (\includegraphics)
%\usepackage{graphicx}
% for neatly defining theorems and propositions
 \usepackage{amsthm}
% making logically defined graphics
%%%\usepackage{xypic}

% there are many more packages, add them here as you need them

% define commands here

\theoremstyle{definition}
\newtheorem*{thmplain}{Theorem}

\begin{document}
\PMlinkescapeword{tangent}

Let us derive the equation of the tangent line of the hyperbola
\begin{align}
\frac{x^2}{a^2}-\frac{y^2}{b^2} \;=\; 1
\end{align}
having\, $(x_0,\,y_0)$\, as the tangency point ($y_0 \neq 0$).\\



If\, $(x_1,\,y_1)$\, is another point of the hyperbola ($x_1 \neq x_0$), the secant line through both points is
\begin{align}
y\!-\!y_0 \;=\; \frac{y_1\!-\!y_0}{x_1\!-\!x_0}(x\!-\!x_0).
\end{align}
Since both points satisfy the equation (1) of the hyperbola, we have
$$0 \;=\; 1\!-\!1 
\;=\; \left(\frac{x_1^2}{a^2}-\frac{y_1^2}{b^2}\right)-\left(\frac{x_0^2}{a^2}-\frac{y_0^2}{b^2}\right)
\;=\; \frac{(x_1\!-\!x_0)(x_1\!+\!x_0)}{a^2}-\frac{(y_1\!-\!y_0)(y_1\!+\!y_0)}{b^2},$$
which implies the proportion equation
$$\frac{y_1\!-\!y_0}{x_1\!-\!x_0} \;=\; \frac{b^2(x_1\!+\!x_0)}{a^2(y_1\!+\!y_0)}.$$
Thus the \PMlinkescapetext{secant} equation (2) may be written
\begin{align}
y\!-\!y_0 \;=\; \frac{b^2(x_1\!+\!x_0)}{a^2(y_1\!+\!y_0)}(x\!-\!x_0).
\end{align}
When we let here\, $x_1 \to x_0, \;\, y_1 \to y_0$,\, this changes to the equation of the tangent:
\begin{align}
y\!-\!y_0 \;=\; \frac{b^2x_0}{a^2y_0}(x\!-\!x_0).
\end{align}
A little simplification allows to write it as
$$\frac{x_0x}{a^2}-\frac{y_0y}{b^2} \;=\; \frac{x_0^2}{a^2}-\frac{y_0^2}{b^2},$$
i.e. 
\begin{align}
\frac{x_0x}{a^2}-\frac{y_0y}{b^2} \;=\; 1.
\end{align}

\textbf{Limiting position of tangent}

Putting first\, $y := 0$\, and then\, $x := 0$\, into (5) one obtains the values 
$$x \;=\; \frac{a^2}{x_0} \quad \mbox{and} \quad y \;=\; -\frac{b^2}{y_0}$$
on which the tangent line intersects the coordinate axes.\, From these one sees that when the point of tangency unlimitedly moves away from the origin ($x_0 \to \infty,\; y_0 \to \infty$), both intersection points tend to the origin.\, At the same time, the slope $\frac{b^2x_0}{a^2y_0}$ tends to a certain limit $\frac{b}{a}$, because
$$\frac{y_0}{x_0} \;=\; \frac{b}{a}\sqrt{x_0^2\!-\!a^2}:x_0 
\;=\; \frac{b}{a}\sqrt{1\!-\!\frac{a^2}{x_0^2}}\; \longrightarrow\,\frac{b}{a}.$$
Thus one infers that the limiting position of the tangent line is the \PMlinkname{asymptote}{Hyperbola2} \,$y = \frac{b}{a}x$\, of the hyperbola.

Consequently, one can say the asymptotes of a hyperbola to be \PMlinkescapetext{tangents} whose tangency points are infinitely far.\\


The tangent (5) halves the angle between the focal radii of the hyperbola drawn from\, $(x_0,\,y_0)$.

%%%%%
%%%%%
\end{document}
