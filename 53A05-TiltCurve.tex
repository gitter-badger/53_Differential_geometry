\documentclass[12pt]{article}
\usepackage{pmmeta}
\pmcanonicalname{TiltCurve}
\pmcreated{2013-03-22 18:08:22}
\pmmodified{2013-03-22 18:08:22}
\pmowner{pahio}{2872}
\pmmodifier{pahio}{2872}
\pmtitle{tilt curve}
\pmrecord{12}{40694}
\pmprivacy{1}
\pmauthor{pahio}{2872}
\pmtype{Definition}
\pmcomment{trigger rebuild}
\pmclassification{msc}{53A05}
\pmclassification{msc}{53A04}
\pmclassification{msc}{51M04}
\pmrelated{OrthogonalCurves}
\pmrelated{Gradient}
\pmrelated{PositionVector}
\pmrelated{FirstFundamentalForm}
\pmrelated{LineOfCurvature}

\endmetadata

% this is the default PlanetMath preamble.  as your knowledge
% of TeX increases, you will probably want to edit this, but
% it should be fine as is for beginners.

% almost certainly you want these
\usepackage{amssymb}
\usepackage{amsmath}
\usepackage{amsfonts}

% used for TeXing text within eps files
%\usepackage{psfrag}
% need this for including graphics (\includegraphics)
%\usepackage{graphicx}
% for neatly defining theorems and propositions
 \usepackage{amsthm}
% making logically defined graphics
%%%\usepackage{xypic}

% there are many more packages, add them here as you need them

% define commands here

\theoremstyle{definition}
\newtheorem*{thmplain}{Theorem}

\begin{document}
The {\em tilt curves} (in German {\em die Neigungskurven}) of a surface
                          $$z = f(x,\,y)$$
are the curves on the surface which \PMlinkname{intersect}{ConvexAngle} orthogonally the level curves \,$f(x,\,y) = c$\, of the surface.\, If the gravitation acts in direction of the negative $z$-axis, then a drop of water on the surface aspires to slide along a tilt curve.\, For example, since the level curves of the sphere \,$z = \pm\sqrt{r^2-x^2-y^2}$\, are the ``latitude circles'', the tilt curves of the sphere are the ``meridian circles''.\, The tilt curves of a helicoid are circular helices.\\

If the tilt curves are projected on the $xy$-plane, the differential equation of those projection curves is
\begin{align}
\frac{dy}{dx} = \frac{f'_y(x,\,y)}{f'_x(x,\,y)}.
\end{align}
Naturally, they also cut orthogonally (the projections of) the level curves.\\

\textbf{Example.}\, Let us find the tilt curves of the elliptic paraboloid
$$z = \frac{x^2}{a^2}+\frac{y^2}{b^2}.$$
The level curves are the ellipses \,$\frac{x^2}{a^2}+\frac{y^2}{b^2} = c$.\, Now we have
$$f'_x(x,\,y) = \frac{\partial}{\partial x}\!\left(\frac{x^2}{a^2}\!+\!\frac{y^2}{b^2}\right) = \frac{2x}{a^2}, \quad
  f'_y(x,\,y) = \frac{\partial}{\partial y}\!\left(\frac{x^2}{a^2}\!+\!\frac{y^2}{b^2}\right) = \frac{2y}{b^2},$$
whence the differential equation of the tilt curves is
$$\frac{dy}{dx} = \frac{a^2}{b^2}\!\cdot\!\frac{y}{x}.$$
The separation of variables and the integration yield
$$\int\frac{dy}{y} = \frac{a^2}{b^2}\!\int\frac{dx}{x},$$
then
$$\ln|y| = \frac{a^2}{b^2}\ln|x|+\ln|C| = \ln(|C||x|^{a^2/b^2}),$$
and finally
\begin{align}
y = C|x|^{a^2/b^2}.
\end{align}
Here, we may allow for $C$ all positive and negative values.\, The curves (2) originate from the origin and continue infinitely far.\\


\textbf{Remark.}\, Given an arbitrary family of parametre curves on a surface 
             $$\vec{r} \,=\, (x(u,\,v),\;y(u,\,v),\;z(u,\,v))^\intercal$$
of $\mathbb{R}^3$, e.g. in the form
                     $$\frac{du}{dv} = f(u,\,v),$$
the family of its orthogonal curves on the surface has in the Gaussian coordinates $u,\,v$ the differential equation
\begin{align}
\frac{dv}{du} = -\frac{g_{11}+g_{12}f(u,\,v)}{g_{12}+g_{22}f(u,\,v)},
\end{align}
where 
$$g_{11} = \vec{r}\,'_u\cdot\vec{r}\,'_u, \quad g_{12} = \vec{r}\,'_u\cdot\vec{r}\,'_v, \quad 
  g_{22} = \vec{r}\,'_v\cdot\vec{r}\,'_v$$
are the \PMlinkescapetext{first order} fundamental quantities $E,\,F,\,G$ of Gauss, respectively.



%%%%%
%%%%%
\end{document}
