\documentclass[12pt]{article}
\usepackage{pmmeta}
\pmcanonicalname{SlopeAngle}
\pmcreated{2013-03-22 17:13:14}
\pmmodified{2013-03-22 17:13:14}
\pmowner{Wkbj79}{1863}
\pmmodifier{Wkbj79}{1863}
\pmtitle{slope angle}
\pmrecord{22}{39546}
\pmprivacy{1}
\pmauthor{Wkbj79}{1863}
\pmtype{Definition}
\pmcomment{trigger rebuild}
\pmclassification{msc}{53A04}
\pmclassification{msc}{51N20}
\pmrelated{Slope}
\pmrelated{LineInThePlane}
\pmrelated{ZeroesOfDerivativeOfComplexPolynomial}

\usepackage{amssymb}
\usepackage{amsmath}
\usepackage{amsfonts}
\usepackage{pstricks}
\usepackage{psfrag}
\usepackage{graphicx}
\usepackage{amsthm}
%%\usepackage{xypic}

\begin{document}
\PMlinkescapeword{axis}
\PMlinkescapeword{right}

The {\em slope angle} $\alpha$ of any line in $\mathbb{R}^2$ is the \PMlinkname{angle between}{AngleBetweenTwoLines} the line and $x$-axis; in the case that the line is descending, the angle is negative.  We always have that $-90^{\circ}<\alpha\le 90^{\circ}$.

For all non-vertical lines having a slope of $m$, the slope angle is

$$\alpha=\arctan{m},$$

and we also have that

$$m=\tan\alpha.$$

Below are some examples of slope angles of lines that intersect the $x$-axis.

\begin{center}
\begin{pspicture}(-7,-3)(7,3)
\rput[l](-7,0){.}
\rput[a](1,3){.}
\rput[b](7,-3){.}
\psline[linecolor=red]{<->}(-7,0)(-1,0)
\psline[linecolor=red]{<->}(1,0)(7,0)
\psline[linecolor=blue]{<->}(-7,-2)(-1,2)
\psline[linecolor=blue]{<->}(1,3)(7,-3)
\psarc(-4,0){0.3}{0}{33.69}
\psarc(4,0){0.3}{-40.5}{0}
\rput[r](-3.2,0.2){$\alpha$}
\rput[r](5.0,-0.3){$|\alpha|$}
\rput[r](-0.8,-0.2){$x$}
\rput[r](7.15,-0.2){$x$}
\rput(-3.5,-2.5){$\mbox{positive \,}\alpha$}
\rput(3.5,-2.5){$\mbox{negative \,}\alpha$}
\end{pspicture}
\end{center}

In some mathematical works, the slope angle of a line in $\mathbb{R}^2$ is defined to be the angle measured from the positive $x$-axis to the line in the counterclockwise direction. This definition has some drawbacks because slope angles are then allowed to be obtuse but can no longer be negative.  When this definition is used, the \PMlinkescapetext{formula} $\alpha=\arctan{m}$ no longer holds.  The convention in PlanetMath is to use the former definition.

\begin{thebibliography}{9}
\bibitem{1728}``Slope, Distance and Equation Calculator.'' {\em 1729 Software Systems.} Accessed on 24 June 2007. 
URL:  \PMlinkexternal{http://www.1728.com/distance.htm}{http://www.1728.com/distance.htm}
\bibitem{mcgrawhill}  ``Slope angle.'' \emph{McGraw-Hill Dictionary of Scientific and Technical Terms.} McGraw-Hill Companies, Inc., 2003.  Accessed via Answers.com on 7 June 2007.  URL: \PMlinkexternal{http://www.answers.com/topic/slope-angle}{http://www.answers.com/topic/slope-angle}
\end{thebibliography}
%%%%%
%%%%%
\end{document}
