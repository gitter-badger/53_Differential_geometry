\documentclass[12pt]{article}
\usepackage{pmmeta}
\pmcanonicalname{EuclideanSpaceAsAManifold}
\pmcreated{2013-03-22 15:29:48}
\pmmodified{2013-03-22 15:29:48}
\pmowner{matte}{1858}
\pmmodifier{matte}{1858}
\pmtitle{Euclidean space as a manifold}
\pmrecord{9}{37355}
\pmprivacy{1}
\pmauthor{matte}{1858}
\pmtype{Definition}
\pmcomment{trigger rebuild}
\pmclassification{msc}{53B21}
\pmclassification{msc}{53B20}

% this is the default PlanetMath preamble.  as your knowledge
% of TeX increases, you will probably want to edit this, but
% it should be fine as is for beginners.

% almost certainly you want these
\usepackage{amsmath}
\usepackage{amsfonts}
\usepackage{amssymb}

\usepackage{bbm}

\newcommand{\En}{\mathbbmss{E}^n}
\newcommand{\V}{\mathbbmss{V}}
\newcommand{\R}{\mathbbmss{R}}
\newcommand{\T}{\operatorname{T}}
\begin{document}
Let $\En$ be $n$-dimensional Euclidean space, and let $(\V,\langle
\cdot, \cdot\rangle)$ be the corresponding $n$-dimensional
inner product space of translation isometries.  Alternatively, we can
consider Euclidean space as an inner product space that has forgotten
which point is its origin.  Forgetting even more information, we have
the structure of $\En$ as a differential manifold.  We can obtain an
atlas with just one coordinate chart, a Cartesian coordinate system
$(x^1,\ldots,x^n)$ which gives us a bijection between $\En$ and $\R^n$.  The
tangent bundle is trivial, with $\T\En \cong \En \times \V.$
Equivalently, every tangent space $\T_p\En,\; p\in \En$. is isomorphic
to $\V$.

We can retain a bit more structure, and consider $\En$ as a Riemannian
manifold by equipping it with the metric tensor
\begin{eqnarray*}
g &=& dx^1 \otimes dx^1 + \cdots + dx^n \otimes dx^n \\
&=& \delta_{ij} dx^i \otimes dx^j.
\end{eqnarray*}
We can also describe $g$ in a coordinate-free fashion as
\[ g(u,v) = \langle u,v\rangle,\quad u,v\in \V.\] 
\subsubsection*{Properties}
\begin{enumerate}
\item Geodesics are straight lines in $\R^n$.
\item The Christoffel symbols vanish identically.
\item The Riemann curvature tensor vanish identically.
\end{enumerate}
Conversely, we can
characterize Eucldiean space as a connected, complete Riemannian
manifold with vanishing curvature and trivial fundamental group.
%%%%%
%%%%%
\end{document}
