\documentclass[12pt]{article}
\usepackage{pmmeta}
\pmcanonicalname{TangentMap}
\pmcreated{2013-03-22 14:06:19}
\pmmodified{2013-03-22 14:06:19}
\pmowner{matte}{1858}
\pmmodifier{matte}{1858}
\pmtitle{tangent map}
\pmrecord{7}{35506}
\pmprivacy{1}
\pmauthor{matte}{1858}
\pmtype{Definition}
\pmcomment{trigger rebuild}
\pmclassification{msc}{53-00}
\pmsynonym{push forward map}{TangentMap}
\pmsynonym{pushforward}{TangentMap}
\pmsynonym{pushforward map}{TangentMap}
\pmrelated{PullbackOfAKForm}
\pmrelated{FlowBoxTheorem}

% this is the default PlanetMath preamble.  as your knowledge
% of TeX increases, you will probably want to edit this, but
% it should be fine as is for beginners.

% almost certainly you want these
\usepackage{amssymb}
\usepackage{amsmath}
\usepackage{amsfonts}
\usepackage{amsthm}

% used for TeXing text within eps files
%\usepackage{psfrag}
% need this for including graphics (\includegraphics)
%\usepackage{graphicx}
% for neatly defining theorems and propositions
%
% making logically defined graphics
%%%\usepackage{xypic}

% there are many more packages, add them here as you need them

% define commands here

\newcommand{\sR}[0]{\mathbb{R}}
\newcommand{\sC}[0]{\mathbb{C}}
\newcommand{\sN}[0]{\mathbb{N}}
\newcommand{\sZ}[0]{\mathbb{Z}}

 \usepackage{bbm}
 \newcommand{\Z}{\mathbbmss{Z}}
 \newcommand{\C}{\mathbbmss{C}}
 \newcommand{\R}{\mathbbmss{R}}
 \newcommand{\Q}{\mathbbmss{Q}}



\newcommand*{\norm}[1]{\lVert #1 \rVert}
\newcommand*{\abs}[1]{| #1 |}



\newtheorem{thm}{Theorem}
\newtheorem{defn}{Definition}
\newtheorem{prop}{Proposition}
\newtheorem{lemma}{Lemma}
\newtheorem{cor}{Corollary}
\begin{document}
\PMlinkescapeword{represent}
\begin{defn}
Suppose $X$ and $Y$ are smooth manifolds with tangent bundles
$TX$ and $TY$, and suppose $f:X\to Y$
is a smooth mapping. Then the {\bf tangent map} of $f$ is the map
$Df\colon TX\to TY$ defined as follows: If $v\in T_x(X)$ for some 
$x\in X$, then
we can represent $v$ by some curve 
$c\colon I \to X$ with $c(0)=x$ and $I=(-1,1)$. 
Now $(Df)(v)$ is defined as the tangent vector in $T(Y)$ 
represented by the curve $f\circ c\colon I \to Y$. Thus,
since $(f\circ c)(0)=f(x)$, it follows that $(Df)(v)\in T_{f(x)}(Y)$.
\end{defn}

\subsubsection*{Properties}
Suppose $X$ and $Y$ are a smooth manifolds.
\begin{itemize}
\item If $\operatorname{id}_X$ is the identity mapping on $X$, then 
$D\mbox{id}_X$ is the identity mapping on $TX$. 
\item Suppose $X,Y,Z$ are smooth manifolds, and $f,g$ are mappings
$f\colon X\to Y$, $g\colon Y\to Z$. Then 
$$ 
  D(f\circ g) = (Df)\circ (Dg).
$$
\item If $f\colon X\to Y$ is a diffeomorphism, then the inverse of $Df$ is a diffeomorphism,
and 
$$ 
  (Df)^{-1}=D(f^{-1}).
$$
\end{itemize}

\subsubsection*{Notes}
Note that if $f\colon X\to Y$ is a mapping as in 
the definition, then the tangent map is
a mapping 
$$
 Df\colon  TX\to TY,
$$
whereas the \PMlinkname{pullback}{PullbackOfAKForm} of $f$ is a mapping
$$
  f^\ast\colon \Omega^k(Y)\to \Omega^k(X).
$$
For this reason, the tangent map is also sometimes called the pushforward map.
That is, a pullback takes objects from $Y$ to $X$, and 
a pushforward  takes objects from $X$ to $Y$.

Sometimes, the tangent map of $f$ is also denoted by $f_\ast$. However,
the motivation for denoting the tangent map by $Df$ is that if $X$ and $Y$
are open subsets in $\sR^n$ and $\sR^m$, then $Df$ is simply
the Jacobian of $f$.
%%%%%
%%%%%
\end{document}
