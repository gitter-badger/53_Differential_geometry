\documentclass[12pt]{article}
\usepackage{pmmeta}
\pmcanonicalname{BibliographyForDifferentialGeometry}
\pmcreated{2013-03-22 14:16:46}
\pmmodified{2013-03-22 14:16:46}
\pmowner{archibal}{4430}
\pmmodifier{archibal}{4430}
\pmtitle{bibliography for differential geometry}
\pmrecord{15}{35731}
\pmprivacy{1}
\pmauthor{archibal}{4430}
\pmtype{Bibliography}
\pmcomment{trigger rebuild}
\pmclassification{msc}{53-00}

% this is the default PlanetMath preamble.  as your knowledge
% of TeX increases, you will probably want to edit this, but
% it should be fine as is for beginners.

% almost certainly you want these
\usepackage{amssymb}
\usepackage{amsmath}
\usepackage{amsfonts}

% used for TeXing text within eps files
%\usepackage{psfrag}
% need this for including graphics (\includegraphics)
%\usepackage{graphicx}
% for neatly defining theorems and propositions
%\usepackage{amsthm}
% making logically defined graphics
%%%\usepackage{xypic}

% there are many more packages, add them here as you need them

% define commands here

\newtheorem{theorem}{Theorem}
\newtheorem{defn}{Definition}
\newtheorem{prop}{Proposition}
\newtheorem{lemma}{Lemma}
\newtheorem{cor}{Corollary}
\begin{document}
\PMlinkescapeword{tensor}
\PMlinkescapeword{theory}
\PMlinkescapeword{sources}
\PMlinkescapeword{series}
\section*{References for Differential Geometry, MSC 53 - XX}

The following are excellent sources for the indicated areas in differential geometry.

\subsection*{Foundations}
\begin{enumerate}
\item Richard L. Bishop and Samuel I. Goldberg \emph{\PMlinkname{Tensor}{Tensor} Analysis on Manifolds} Macmillan 1968; Dover, New York, 1980 ISBN 0-486-640439-6
\begin{quote}
Don't be fooled by the title!  Although you will find little in the way of the index gymnastics traditionally associated with tensor analysis, this book nevertheless exercise your mind in abstract thinking and lay a solid foundation for further study in differential geometry, topology, and rational mechanics.  The prerequisites for this book are advanced calculus, linear algebra, and the basics of differential equations.  (Also, familiarity with the abstract point of view in mathematics is a must.  As the authors say in the introduction ``An understanding of what it means for solutions of systems of differential equations to exist and be unique is more imporant than an ability to crank out general solutions.'') Starting with the basics of set theory, the authors develop the basics of topology and linear algebra.  On this foundation, they define manifolds and structures on manifolds.  They then develop the basics of the theory of exterior forms (including an introduction to such topics as integration of differential forms on simplicial chains, de Rham cohomology, and the formulation of systems of differential equations as Pfaffian systems), Riemannian and semi-Riemannian geometry, and culminates in a grand finale wherein the authors apply the tools of Riemannian, symplectic, and contact geometry developed in the preceding chapters to the science of mechanics.  The power and utility of the theory developed in this book is amply demonstrated by the fact that in this brief chapter of 18 pages, the authors manage to prove the conservation of energy and momentum, describe the evolution of a mechanical system as the motion of a point on a manifold, and introduce such advanced topics as non-holonomic constraints and transformation theory of phase space.  (Indeed, the development of classical mechanics given here is so sophisticated that one could arrive at general relativity and quantum mechanics in a few short steps!)
\end{quote}
\item Manfredo Perdig\~ao do Carmo \emph{Riemannian Geometry},
Birkhauser, Boston, 1992 ISBN 0-8176-3490-8 ISBN 3-7643-3490-8 QA649.C2913
\item Michael Spivak, \emph{Calculus on Manifolds}, Mathematical Monographs Series, Addison-Wesley 1965, ISBN 0-8053-9021-9.
\begin{quote} 
A basic but thorough introduction to multivariable calculus from the point of view of differential geometry.  Suitable for very bright students, it deals only with implicitly defined manifolds but defines differentiation and integration in their proper generality (for example, integration is integration of differential forms over singular $n$-chains).  It proves Stokes' theorem in its fully general form.  Requires no background beyond single-variable calculus.
\end{quote}
\end{enumerate}


\subsection*{Classical differential geometry, MSC 53 A}

\begin{enumerate}
\item Heinrich W. Guggenheimer \emph{Differential Geometry}, McGraw-Hill, New York, 1963; Dover, New York, 1977, ISBN 0-486-63433
\item Eisenhart, Luther Pfahler \emph{Differential Geometry}, Princeton Unversity Press, Princton, N.J. 1925, 1977, ISBN 0-691-08026-7
\end{enumerate}

\subsection*{Local differential geometry, MSC 53 B}
\begin{enumerate}
\item William M. Boothby, \emph{An introduction to Differentiable Manifolds and Riemannian Geometry}, Second Edition, Academic Press, 1986.
\begin{quote}
This book provides the background in differential geometry required to understand general relativity from a fully rigorous point of view.  Tensors, forms, integration and differentiation are all covered, as are curvature and geodesics.  A very brief section at the end of the book discusses spaces of constant curvature; otherwise no global theory is included.
\end{quote}
\item J. A. Schouten, W. v. d. Kulk, \emph{Pfaff's problem and its generalizations} Oxford University Press 1949, Chelsea Publishing, New York, 1969 ISBN 8284-0221-3
\begin{quote}
This is an comprehensive treastise which includes most everything if not everything that was known about the topic of Pfaff's problem up to the 1940's.  The only disadvantage is that the authors employ a complicated system of index notation which includes not only subscripts and superscripts, but also indices above and below the main symbol and a variety of fonts and alphabets.
\end{quote}
\end{enumerate}

\subsection*{Global differential geometry, MSC 53 C}

\begin{enumerate}
\item Morgan, Frank, \emph{Geometric Measure Theory} Academic Press, Can Diego, Ca. 1988 ISBN 0-12-506855-7 QA312.M67
\end{enumerate}


\subsection*{Symplectic differential geometry and contact geometry, MSC 53 D}
%%%%%
%%%%%
\end{document}
