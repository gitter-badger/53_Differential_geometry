\documentclass[12pt]{article}
\usepackage{pmmeta}
\pmcanonicalname{AlternateCharacterizationOfCurl}
\pmcreated{2013-03-22 15:29:10}
\pmmodified{2013-03-22 15:29:10}
\pmowner{stevecheng}{10074}
\pmmodifier{stevecheng}{10074}
\pmtitle{alternate characterization of curl}
\pmrecord{9}{37341}
\pmprivacy{1}
\pmauthor{stevecheng}{10074}
\pmtype{Derivation}
\pmcomment{trigger rebuild}
\pmclassification{msc}{53-01}
\pmrelated{curl}
\pmrelated{nabla}
\pmrelated{FirstOrderOperatorsInRiemannianGeometry}
\pmrelated{Curl}
\pmrelated{NablaNabla}

\endmetadata

\usepackage{amssymb}
\usepackage{amsmath}
\usepackage{amsfonts}

% used for TeXing text within eps files
%\usepackage{psfrag}
% need this for including graphics (\includegraphics)
%\usepackage{graphicx}
% making logically defined graphics
%%%\usepackage{xypic}

% define commands here
\newcommand{\real}{\mathbb{R}}

\providecommand{\abs}[1]{\lvert#1\rvert}
\providecommand{\absW}[1]{\left\lvert#1\right\rvert}
\providecommand{\absB}[1]{\Bigl\lvert#1\Bigr\rvert}
\providecommand{\norm}[1]{\lVert#1\rVert}
\providecommand{\normW}[1]{\left\lVert#1\right\rVert}
\providecommand{\normB}[1]{\Bigl\lVert#1\Bigr\rVert}
\providecommand{\defnterm}[1]{\emph{#1}}

\DeclareMathOperator{\D}{D}

\newcommand{\vA}{\mathbf{A}}
\newcommand{\vg}{\mathbf{g}}
\newcommand{\vu}{\mathbf{u}}
\newcommand{\vv}{\mathbf{v}}
\newcommand{\ve}{\mathbf{e}}
\newcommand{\vF}{\mathbf{F}}
\newcommand{\vL}{\mathbf{L}}
\newcommand{\vn}{\mathbf{n}}
\newcommand{\vp}{\mathbf{p}}
\newcommand{\DF}{\D \mathbf{F}}
\newcommand{\cross}{\times}
\DeclareMathOperator{\curl}{curl}
\DeclareMathOperator{\diverg}{div}
\begin{document}
Let $\vF$ be a smooth vector field on (an open subset of) $\real^3$.

We show that $\curl \vF$ defined using the \PMlinkname{coordinate-free definition given on the parent entry}{curl}
is the same as the curl defined
by $\nabla \cross \vF$ in Cartesian coordinates.

\section*{The case for spherical surfaces}
This will be done by directly computing the limit $\vL$ of surface 
integrals defining $\curl \vF(\vp)$,
using spheres $S^2(r, \vp)$ centered at $\vp$ of radius $r$.  The formula is:
\begin{align*}
\curl \vF(\vp) = \vL &= \lim_{r \to 0} \frac{3}{4 \pi r^3} \iint_{S^2(r, \vp)} \vn \cross \vF \, dA \\
&= \lim_{r \to 0} \frac{3 r^2}{4\pi r^3} 
\iint_{S^2} \vn \cross \vF(r\vn + \vp) \, dA\,,
\end{align*}
where $\vn$ is the outward unit normal to the surface (at each point of the surface), and 
$S^2$ is the unit sphere at the origin.

We simplify the last integral.  
Expanding $\vF(r\vn + \vp)$
in a first-degree Taylor polynomial about $\vp$, we have
\begin{align*}
\iint_{S^2} \vn \cross \vF(r\vn + \vp) \, dA &=
\iint_{S^2} \vn \cross \vF(\vp) \, dA \\
& \quad + \iint_{S^2} \vn \cross \DF(\vp)r\vn \, dA + \iint_{S^2} \vn \cross o(\norm{r\vn})  \, dA\,.
\end{align*}
The integral $\iint_{S^2} \vn \cross \vF(\vp) \, dA$ 
vanishes by symmetry of the sphere,
while
\begin{align*}
\normW{ \iint_{S^2} \vn \cross o(\norm{r\vn}) \, dA }
\leq \iint_{S^2} \norm{\vn} \, o(r) \, dA = o(r)\,.
\end{align*}
Combining these facts, we obtain
\begin{align*}
\vL &= \lim_{r \to 0} \left[ 0 + \frac{3}{4\pi r} \iint_{S^2} \vn \times \DF(\vp) r\vn \, dA  + o(1) \right]\\
&= \frac{3}{4\pi} \iint_{S^2} \vn \times \DF(\vp) \vn \, dA\,.
\end{align*}
Notice that $\vL$ depends only on the derivative of $\vF$
at $\vp$.

We want to evaluate the last integral in Cartesian coordinates.
Let $\ve_k$ be an orthonormal basis of $\real^3$
oriented positively, and let $B$ be the matrix of the derivative
$\DF(p)$ in this basis.
Then the $k$th coordinate of $\vL$ with respect to the
same basis
is 
\[
\left(\iint_{S^2} \vn \cross B\vn \, dA \right) \cdot \ve_k 
= \iint_{S^2} (\vn \cross B\vn) \cdot \ve_k \, dA
\]
The $k$th coordinate of the integrand is
\[
(\vn \cross B\vn) \cdot \ve_k =
n^i \,  (B\vn)^j \, \epsilon_{ijk}
= n^i \, B^j_l n^l \, \epsilon_{ijk}\,,
\]
where to lessen the writing, we employ
the Einstein summation convention, along
with the Levi-Civita permutation symbol $\epsilon_{ijk}$, and $B^j_l$
denotes the entry at the $j$th row, $l$th column of $B$.

In the summation above, if a summmand has $i \neq l$,
 then the integral of that summand over the sphere is zero, by symmetry.
This means that in the summation the index $l$ may be set to $i$,
and thus
\[
\left(\iint_{S^2} \vn \cross B\vn \, dA \right) \cdot \ve_k 
= \iint_{S^2} n^i B_i^j n^i \epsilon_{ijk} \, dA
= B_i^j \epsilon_{ijk} \iint_{S^2} (n^i)^2 \, dA\,.
\]
Now there is a formula for the evaluation of integrals 
of polynomials over $S^{m-1} \subset \real^m$, in terms of the gamma function;
in our case ($m = 3$) the formula reads:
\[
\iint_{S^2} (n^i)^2 \, dA = \frac{2 \Gamma(\frac{3}{2}) \, \Gamma(\frac{1}{2})
\, \Gamma(\frac{1}{2})}{\Gamma(\frac{3}{2} + \frac{1}{2} + \frac{1}{2})}
= \frac{2 \Gamma(\frac{3}{2}) \, \sqrt{\pi} \, \sqrt{\pi}}{\frac{3}{2} \Gamma(\frac{3}{2})}
= \frac{4\pi}{3}\,.
\]
(If you do not know this formula, the integral in our case
can be computed directly using spherical coordinates.)
Therefore the $k$th component of $\vL$ is
\[
\vL \cdot \ve_k = \frac{3}{4\pi} \, B_i^j \epsilon_{ijk} \iint_{S^2} (n^i)^2 \, dA
= B_i^j \epsilon_{ijk} = \left.\frac{\partial F^j}{\partial x^i}\right|_\vp \, \epsilon_{ijk}\,.
\]
But this is just $(\nabla \cross \vF(\vp)) \cdot \ve_k$.


\section*{The case for arbitrary surfaces}
Although we have only
computed
\[
\vL = \curl \vF(\vp) = \lim_{V \to 0} \frac{1}{V} \iint_S \vn \cross \vF \, dA
\]
only for spheres $S = S^2(r, \vp)$,
this formula holds for arbitrary closed surfaces $S$ that shrink nicely to $\vp$.
It is hardly obvious, especially since our computation before depended
on the symmetry of the sphere extensively.

To show the general result,
consider the triple scalar product
$(\vv \times \vF) \cdot \ve_k$.
This is a linear functional in the vector $\vv$,
so there exists
a unique vector function $\vg_k$ such that
$(\vv \times \vF) \cdot \ve_k = \vg_k \cdot \vv$ for all $\vv \in \real^3$.
We can find the components of this $\vg_k$ by evaluating the functional
at $\vv = \ve_i$:
\[
g_k^i = \vg_k \cdot \ve_i = (\ve_i \times \vF) \cdot \ve_k = \det(\ve_i, \vF, \ve_k) = F^j \epsilon_{ijk}\,.
\]
The reason for considering such expressions is that, putting $\vv = \vn$, we have
\[
\iint_{S} (\vn \times \vF) \cdot \ve_k \, dA =
\iint_{S} \vg_k \cdot \vn \, dA = \iint_{S} \vg_k \cdot d\vA\,.
\]
So we have converted the original integral into an ordinary surface integral.
And this surface integral can be changed into a volume integral, by using the divergence theorem:
\[
\iint_S \vg_k \cdot d\vA = \iiint_M \diverg \vg_k \, dV = \iiint_M \frac{\partial F^j}{\partial x^i} \, \epsilon_{ijk} \, dV\,,
\]
where $M$ is the volume whose boundary is $S$.
Hence
\begin{align*}
\vL \cdot \ve_k &= \lim_{V \to 0} \frac{1}{V} \iint_S (\vn \times \vF) \cdot \ve_k \, dA 
\\ &= 
\lim_{V \to 0} \frac{1}{V} \iiint_M \frac{\partial F^j}{\partial x^i} \, \epsilon_{ijk} \, dV \\ &=
\left.\frac{\partial F^j}{\partial x^i}\right|_\vp \, \epsilon_{ijk} = (\nabla \cross \vF(\vp)) \cdot \ve_k \,.
\end{align*}

\section*{Definition in terms of differential forms}
We mention, in passing,
a computational, yet coordinate-free, alternative to the definition of the curl,
using differential forms.
If $\omega$ is a 1-form on $\real^3$ such that $\omega(\vv) = \langle \vF, \vv \rangle$,
then the curl of $\vF$ is defined as the vector function $\vg = g^k \, e_k$ such that
\[
d\omega(\vu, \vv) = \langle \vg, \vu \cross \vv \rangle\,.
\]
In Cartesian coordinates, we have
\begin{align*}
\omega &= F^1 \, dx^1 + F^2 \, dx^2 + F^3 \, dx^3 \\
d\omega &= g^1 \, dx^2 \wedge dx^3 + g^2 \, dx^3 \wedge \, dx^1 + g^3 \, dx^1 \wedge dx^2\,,
\end{align*}
If we take the exterior derivative of the first equation for $\omega$, and then equate components
with the second equation for $d\omega$,
we find that $g^k$ = $(\nabla \cross \vF) \cdot \ve_k$,
so our new definition is equivalent to the others.
%%%%%
%%%%%
\end{document}
