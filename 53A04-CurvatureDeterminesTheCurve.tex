\documentclass[12pt]{article}
\usepackage{pmmeta}
\pmcanonicalname{CurvatureDeterminesTheCurve}
\pmcreated{2016-02-22 16:14:25}
\pmmodified{2016-02-22 16:14:25}
\pmowner{pahio}{2872}
\pmmodifier{pahio}{2872}
\pmtitle{curvature determines the curve}
\pmrecord{11}{39353}
\pmprivacy{1}
\pmauthor{pahio}{2872}
\pmtype{Theorem}
\pmcomment{trigger rebuild}
\pmclassification{msc}{53A04}
\pmsynonym{fundamental theorem of plane curves}{CurvatureDeterminesTheCurve}
%\pmkeywords{plane curve}
\pmrelated{FundamentalTheoremOfSpaceCurves}
\pmrelated{ErnstLindelof}

\endmetadata

% this is the default PlanetMath preamble.  as your knowledge
% of TeX increases, you will probably want to edit this, but
% it should be fine as is for beginners.

% almost certainly you want these
\usepackage{amssymb}
\usepackage{amsmath}
\usepackage{amsfonts}

% used for TeXing text within eps files
%\usepackage{psfrag}
% need this for including graphics (\includegraphics)
%\usepackage{graphicx}
% for neatly defining theorems and propositions
 \usepackage{amsthm}
% making logically defined graphics
%%%\usepackage{xypic}

% there are many more packages, add them here as you need them

% define commands here

\theoremstyle{definition}
\newtheorem*{thmplain}{Theorem}

\begin{document}
\PMlinkescapeword{curvature} \PMlinkescapeword{represents}

The \PMlinkname{curvature}{CurvaturePlaneCurve} of plane curve determines uniquely the form and \PMlinkescapetext{size} of the curve, i.e. one has the

\textbf{Theorem.}\; If\; $s\mapsto k(s)$\; is a continuous real function, then there exists always plane curves satisfying the equation
\begin{align}
        \kappa \;=\; k(s)
\end{align}
between their curvature $\kappa$ and the arc length $s$.\, All these curves are \PMlinkname{congruent}{Congruence}.\\

{\em Proof.}\; Suppose that a curve $C$ satisfies the condition (1).\, Let the value\, $s = 0$\, correspond to the point $P_0$ of this curve.\, We choose $O$ as the origin of the plane.\, The tangent and the normal of $C$ in $O$ are chosen as the $x$-axis and the $y$-axis, with positive directions the directions of the tangent and normal vectors of $C$, respectively.\, According to (1) and the definition of curvature, the equation
          $$\frac{d\theta}{ds} \;=\; k(s)$$
for the direction angle $\theta$ of the tangent of $C$ is valid in this coordinate system; the initial condition is
$$\theta \;=\; 0 \quad \mbox{when} \quad s \;=\; 0.$$
Thus we get
\begin{align}
     \theta \;=\; \int_0^s k(t) d\!t \;:=\; \vartheta(s),
\end{align}
which implies
\begin{align}
   \frac{dx}{ds} \;=\; \cos{\vartheta(s)}, \qquad 
   \frac{dy}{ds} \;=\; \sin{\vartheta(s)}.
\end{align}
Since\, $x = y = 0$\, when\, $s = 0$, we obtain
\begin{align}
  x \;=\; \int_0^s \cos{\vartheta(t)}\,dt, \qquad y \;=\; 
  \int_0^s \sin{\vartheta(t)}\,dt.
\end{align}
Thus the function \,$s\mapsto k(s)$\, determines uniquely these functions $x$ and $y$ of the parameter $s$, and (4) represents a curve with definite form and \PMlinkescapetext{size}.

The above reasoning shows that every curve which satisfies (1) is \PMlinkname{congruent}{Congruence} with the curve (4).

We have still to show that the curve (4) satisfies the condition (1).\, By \PMlinkname{differentiating}{HigherOrderDerivatives} the equations (4) we get the equations (3), which imply\; $(\frac{dx}{ds})^2+(\frac{dy}{ds})^2 = 1$,\; or\, $ds^2 = dx^2+dy^2$\, which means that the parameter $s$ represents the arc length of the curve (4), counted from the origin.\, Differentiating (3) we get, because\, $\vartheta'(s) = k(s)$\, by (2),
\begin{align}
\frac{d^2x}{ds^2} \;=\; -k(s)\sin{\vartheta(s)}, \qquad 
\frac{d^2y}{ds^2} \;=\; k(s)\cos{\vartheta(s)}.    
\end{align}
The equations (3) and (5) then yield
$$\frac{dx}{ds}\frac{d^2y}{ds^2}-\frac{dy}{ds}\frac{d^2x}{ds^2} \;=\; k(s),$$
i.e. the curvature of (4), according the \PMlinkname{parent entry}{CurvaturePlaneCurve}, satisfies
$$ \left| \begin{matrix} x' & y' \cr x'' & y'' \end{matrix} \right| \;=\; k(s).$$
Thus the proof is settled.

\begin{thebibliography}{8}
\bibitem{lindelof}{\sc Ernst Lindel\"of}: {\it Differentiali- ja integralilasku
ja sen sovellutukset I}.\, WSOY. Helsinki (1950).
\end{thebibliography}


%%%%%
%%%%%
\end{document}
