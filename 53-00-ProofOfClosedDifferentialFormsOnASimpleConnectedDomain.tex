\documentclass[12pt]{article}
\usepackage{pmmeta}
\pmcanonicalname{ProofOfClosedDifferentialFormsOnASimpleConnectedDomain}
\pmcreated{2013-03-22 13:32:49}
\pmmodified{2013-03-22 13:32:49}
\pmowner{paolini}{1187}
\pmmodifier{paolini}{1187}
\pmtitle{proof of closed differential forms on a simple connected domain}
\pmrecord{9}{34147}
\pmprivacy{1}
\pmauthor{paolini}{1187}
\pmtype{Proof}
\pmcomment{trigger rebuild}
\pmclassification{msc}{53-00}
\pmrelated{SubstitutionNotation}

\endmetadata

% this is the default PlanetMath preamble.  as your knowledge
% of TeX increases, you will probably want to edit this, but
% it should be fine as is for beginners.

% almost certainly you want these
\usepackage{amssymb}
\usepackage{amsmath}
\usepackage{amsfonts}

% used for TeXing text within eps files
%\usepackage{psfrag}
% need this for including graphics (\includegraphics)
%\usepackage{graphicx}
% for neatly defining theorems and propositions
\usepackage{amsthm}
% making logically defined graphics
%%%\usepackage{xypic}

% there are many more packages, add them here as you need 

% define commands here
\begin{document}
\begin{proof}[lemma 1]
Let $\gamma_0$ and $\gamma_1$ be two regular homotopic curves in $D$ with 
the same end-points.
Let $\sigma\colon[0,1]\times[0,1]\to D$ be the homotopy between $\gamma_0$ and $\gamma_1$ i.e.
\[
  \sigma(0,t) = \gamma_0(t),\qquad
  \sigma(1,t) = \gamma_1(t).
\]

Notice that we may (and shall) suppose that $\sigma$ is regular too. In fact $\sigma([0,1]\times[0,1])$ is a compact subset of $D$. Being $D$ open this compact set has positive distance from the boundary $\partial D$. So we could
regularize $\sigma$ by mollification leaving its image in $D$.

Let $\omega(x,y) = a(x,y)\, dx + b(x,y)\, dy$ be our closed differential form and let $\sigma(s,t) = (x(s,t),y(s,t))$.
Define 
\[
  F(s) = \int_0^1 a(x(s,t),y(s,t)) x_t(s,t) + b(x(s,t),y(s,t)) y_t (s,t) \, dt;
\]
we only have to prove that $F(1)=F(0)$.

We have
\[
  F'(s) = \frac{d}{ds}\int_0^1 a x_t + b y_t\, dt
\]\[
  = \int_0^1 a_x x_s x_t + a_y y_s x_t + a x_{ts} + b_x x_s y_t + 
     b_y y_s y_t + b y_{ts}\, dt.
\]
Notice now that being $a_y=b_x$ we have
\[
  \frac{d}{dt}\left[ a x_s + b y_s  \right]
  = a_x x_t x_s + a_y y_t x_s + a x_{st} + 
    b_x x_t y_s + b_y y_t y_s + b y_{st}
\]\[
  = a_x x_s x_t + b_x x_s y_t + a x_{ts} +
    a_y y_s x_t + b_y y_s y_t + b y_{ts}
\]
hence
\[
 F'(s) = \int_0^1 \frac{d}{dt}\left[ a x_s + b y_s\right]\, dt
       = \left[ a x_s + b y_s\right]_0^1.
\]

Notice, however, that $\sigma(s,0)$ and $\sigma(s,1)$ are constant hence $x_s=0$ and $y_s=0$  for $t=0,1$. So $F'(s)=0$ for all $s$ and $F(1)=F(0)$.
\end{proof}

\begin{proof}[Lemma 2]
Let us fix a point $(x_0,y_0)\in D$ and define a function $F\colon D\to \mathbb R$ by letting $F(x,y)$ be the integral of $\omega$ on any curve joining $(x_0,y_0)$ with $(x,y)$. The hypothesis assures that $F$ is well defined.
Let $\omega = a(x,y)\, dx + b(x,y)\, dy$. We only have to prove that $\partial F / \partial x = a$ and $\partial F/\partial y=b$.

Let $(x,y)\in D$ and suppose that $h\in \mathbb R$ is so small that for all $t\in [0,h]$ also $(x+t,y)\in D$. Consider the increment $F(x+h,y)-F(x,y)$.
From the definition of $F$ we know that $F(x+h,y)$ is equal to the integral of $\omega$ on a curve which starts from $(x_0,y_0)$ goes to $(x,y)$ and then goes to $(x+h,y)$ along the straight segment $(x+t,y)$ with $t\in [0,h]$.
So we understand that
\[
  F(x+h,y)-F(x,y) = \int_0^h a(x+t,y)dt.
\]

For the integral mean value theorem we know that the last integral is equal to $ha(x+\xi,y)$ for some $\xi \in [0,h]$ and hence letting $h\to 0$ we have
\[
  \frac{F(x+h,y)-F(x,y)}{h} = a(x+\xi,y) \to a(x,y)\quad h\to 0
\]
that is $\partial F(x,y)/\partial x = a(x,y)$. With a similar argument (exchange $x$ with $y$) we prove that also $\partial F/\partial y = b(x,y)$.
\end{proof}

\begin{proof}[Theorem]
Just notice that if $D$ is simply connected, then any two curves in $D$ with the same end points are homotopic. Hence we can apply Lemma~1 and then Lemma~2 to obtain the desired result.
\end{proof}
%%%%%
%%%%%
\end{document}
