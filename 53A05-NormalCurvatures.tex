\documentclass[12pt]{article}
\usepackage{pmmeta}
\pmcanonicalname{NormalCurvatures}
\pmcreated{2013-03-22 17:26:27}
\pmmodified{2013-03-22 17:26:27}
\pmowner{pahio}{2872}
\pmmodifier{pahio}{2872}
\pmtitle{normal curvatures}
\pmrecord{13}{39821}
\pmprivacy{1}
\pmauthor{pahio}{2872}
\pmtype{Topic}
\pmcomment{trigger rebuild}
\pmclassification{msc}{53A05}
\pmclassification{msc}{26B05}
\pmclassification{msc}{26A24}
\pmrelated{SecondFundamentalForm}
\pmrelated{MeusniersTheorem}
\pmrelated{ErnstLindelof}
\pmdefines{principal normal plane}
\pmdefines{principal section}
\pmdefines{principal curvature}
\pmdefines{Euler's theorem}

% this is the default PlanetMath preamble.  as your knowledge
% of TeX increases, you will probably want to edit this, but
% it should be fine as is for beginners.

% almost certainly you want these
\usepackage{amssymb}
\usepackage{amsmath}
\usepackage{amsfonts}

% used for TeXing text within eps files
%\usepackage{psfrag}
% need this for including graphics (\includegraphics)
%\usepackage{graphicx}
% for neatly defining theorems and propositions
 \usepackage{amsthm}
% making logically defined graphics
%%%\usepackage{xypic}

% there are many more packages, add them here as you need them

% define commands here

\theoremstyle{definition}
\newtheorem*{thmplain}{Theorem}

\begin{document}
\PMlinkescapeword{curvature}
Let us determine the \PMlinkname{normal curvatures}{NormalSection} $\varkappa$ of the surface
\begin{align}
z \;=\; z(x,\,y)
\end{align}
in the origin, when (1) has the continuous 1st and 2nd order partial derivatives in a neighbourhood of\, $(0,\,0)$\, and satisfies
\begin{align}
z(0,\,0) \;=\; z'_x(0,\,0) \;=\; z'_y(0,\,0) = 0.
\end{align}
It's a question of the \PMlinkname{curvature}{CurvaturePlaneCurve} of the intersection curves of the surface (1) and planes containing the $z$-axis, which is the normal of the surface in the origin.\\ \\

If the angle between the $zx$-plane and a plane $\tau$ containing the $z$-axis is denoted by $\varphi$, when the line of intersection of the plane $\tau$ and the $xy$-plane is represented by the equations
$$x \;=\; \varrho\cos\varphi,\quad y \;=\; \varrho\sin\varphi \qquad (-\infty \,<\, \varrho \,<\, \infty),$$
then equation of the the normal section curve $C_\varphi$ is
$$z \;=\; z(\varrho\cos\varphi,\,\varrho\sin\varphi),$$
where $\varrho$ is the abscissa and $z$ the ordinate.\, It follows that
$$\frac{dz}{d\varrho} \;=\; \frac{\partial z}{\partial x}\cos\varphi+\frac{\partial z}{\partial y}\sin\varphi,$$
$$\frac{d^2z}{d\varrho^2} \;=\; 
\frac{\partial^2z}{\partial x^2}\cos^2\!\varphi+2\frac{\partial^2z}{\partial x \partial y}\sin\varphi\cos\varphi+
\frac{\partial^2z}{\partial y^2}\sin^2\!\varphi;$$
thus by (2), in the origin we have
$$\frac{dz}{d\varrho} \;=\; 0,\quad 
\frac{d^2z}{d\varrho^2} \;=\; a\cos^2\!\varphi+2b\sin\varphi\cos\varphi+c\sin^2\!\varphi,$$
where $a$, $b$, $c$ \PMlinkescapetext{mean} the values of the derivatives $\frac{\partial^2z}{\partial x^2}$, 
$\frac{\partial^2z}{\partial x \partial y}$, $\frac{\partial^2z}{\partial y^2}$ in the origin.\\ \\
Using those values, we obtain for the normal curvature of $C_\varphi$ in the origin the value
\begin{align}
\varkappa(\varphi) \;=\; 
\left[\frac{\frac{d^2z}{d\varrho^2}}{\left(1+\left(\frac{dz}{d\varrho}\right)^2\right)^{3/2}}\right]_{\varrho\,=\,0}
\;=\; a\cos^2\!\varphi+2b\sin\varphi\cos\varphi+c\sin^2\!\varphi.
\end{align}
This result gets a more illustrative form when we try to express it by using the least and the greatest value of $\varkappa(\varphi)$.\, Instead to utilize the zeros of the derivative of the sum in (3), it's simpler first to transfer to the \PMlinkname{double angle}{DoubleAngleIdentity},
\begin{align}
\varkappa(\varphi) \;=\; \frac{a\!+\!c}{2}+\frac{a\!-\!c}{2}\cos2\varphi+b\sin2\varphi,
\end{align}
and here to introduce an auxiliary angle $\alpha$\, ($0 \,\le\, \alpha \,<\, \pi$) such that
$$\frac{a\!-\!c}{2} \;:=\; k\cos2\alpha,\quad b \;:=\; k\sin2\alpha.$$
This allows us to write (4) as
\begin{align}
\varkappa(\varphi) \;=\; \frac{a\!+\!c}{2}+k\,\cos2(\varphi\!-\!\alpha).
\end{align}
From this we see immediately that the curvature attains its greatest and least value 
$\displaystyle\frac{a\!+\!c}{2}\pm k$ when\, $\varphi = \alpha$\, and\, $\varphi = \alpha\!+\!\frac{\pi}{2}$.

Accordingly, the corresponding \PMlinkescapetext{normal planes} $\tau$, the {\em principal normal planes}, are perpendicular to each other; their normal section curves on the surface (1) in the origin are briefly called the {\em principal sections}.

The expression (5) of the normal curvature may still be edited.\, Let us take a new parameter angle\, $\varphi-\alpha := \theta$.\, One can write
$$\varkappa(\varphi) \;=\; \frac{a\!+\!c}{2}(\cos^2\theta+\sin^2\theta)+k(\cos^2\theta-\sin^2\theta) 
\;=\; \left(\!\frac{a\!+\!c}{2}+k\!\right)\cos^2\theta+\left(\!\frac{a\!+\!c}{2}-k\!\right)\sin^2\theta 
\;:=\; \varkappa_\theta.$$

So the final result, the so-called {\em \PMlinkname{Euler's theorem}{SecondFundamentalForm}}, can be expressed in the form
\begin{align}
\varkappa_\theta \;=\; \varkappa_1\cos^2\theta+\varkappa_2\sin^2\theta.
\end{align}
Here, the {\em principal curvatures} $\varkappa_1$ and $\varkappa_2$ are the greatest and the least value of the normal curvature, respectively, and $\theta$ is the \PMlinkname{angle between}{AngleBetweenTwoPlanes} the normal section plane corresponding $\varkappa_1$ and the normal section plane corresponding $\varkappa_\theta$.  As it becomes clear in the \PMlinkname{parent entry}{NormalSection}, the result (6) is true not only in the origin but at any point on a surface when the given function has the continuous 1st and 2nd derivatives in some neighbourhood of the point.


\begin{thebibliography}{8}
\bibitem{lindelof}{\sc Ernst Lindel\"of}: {\em Differentiali- ja integralilasku
ja sen sovellutukset II}.\, Mercatorin Kirjapaino Osakeyhti\"o, Helsinki (1932).
\end{thebibliography} 

%%%%%
%%%%%
\end{document}
