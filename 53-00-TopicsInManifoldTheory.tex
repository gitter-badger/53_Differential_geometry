\documentclass[12pt]{article}
\usepackage{pmmeta}
\pmcanonicalname{TopicsInManifoldTheory}
\pmcreated{2013-03-22 14:11:04}
\pmmodified{2013-03-22 14:11:04}
\pmowner{evin290}{5830}
\pmmodifier{evin290}{5830}
\pmtitle{topics in manifold theory}
\pmrecord{15}{35611}
\pmprivacy{1}
\pmauthor{evin290}{5830}
\pmtype{Topic}
\pmcomment{trigger rebuild}
\pmclassification{msc}{53-00}

% this is the default PlanetMath preamble.  as your knowledge
% of TeX increases, you will probably want to edit this, but
% it should be fine as is for beginners.

% almost certainly you want these
\usepackage{amssymb}
\usepackage{amsmath}
\usepackage{amsfonts}

% used for TeXing text within eps files
%\usepackage{psfrag}
% need this for including graphics (\includegraphics)
%\usepackage{graphicx}
% for neatly defining theorems and propositions
%\usepackage{amsthm}
% making logically defined graphics
%%%\usepackage{xypic}

% there are many more packages, add them here as you need them

% define commands here

\newcommand{\sR}[0]{\mathbb{R}}
\newcommand{\sC}[0]{\mathbb{C}}
\newcommand{\sN}[0]{\mathbb{N}}
\newcommand{\sZ}[0]{\mathbb{Z}}

 \usepackage{bbm}
 \newcommand{\Z}{\mathbbmss{Z}}
 \newcommand{\C}{\mathbbmss{C}}
 \newcommand{\R}{\mathbbmss{R}}
 \newcommand{\Q}{\mathbbmss{Q}}



\newcommand*{\norm}[1]{\lVert #1 \rVert}
\newcommand*{\abs}[1]{| #1 |}
\begin{document}
A {\em manifold} is a space that is
locally like $\mathbb{R}^n$, however lacking a preferred system of
coordinates. Furthermore, a manifold can have global topological
properties, such as non-contractible \PMlinkname{loops}{Curve}, that distinguish it from
the topologically trivial $\mathbb{R}^n$.
 

By imposing different restrictions on the transition functions of a manifold, one
obtain different types of manifolds:
\begin{itemize}
\item topological manifolds
\item $C^k$ manifolds, smooth manifolds
\item real analytic manifold
\item complex analytic manifold
\item symplectic manifolds, where transition functions
are symplectomorphisms. On such manifolds,  one can formulate the 
Hamilton equations. 
\end{itemize}

Special types of manifolds
\begin{itemize}
\item orientable manifolds
\item manifolds with boundary
\item compact manifolds
\end{itemize}

On manifolds, one can introduce more \PMlinkescapetext{structure}. Some examples are:
\begin{itemize}
\item Riemannian manifolds
\item contact manifolds
\item CR manifolds
\item fiber bundles and sheaves
\end{itemize}

\subsubsection*{Examples}
\begin{itemize}
\item space-time manifold in general relativity
\item phase space in mechanics
\item de Rham cohomology in algebraic topology
\end{itemize}

\subsubsection*{See also}
For the formal definition click \PMlinkname{here}{Manifold}\\
\PMlinkexternal{Manifold entry at Wikipedia}{http://en.wikipedia.org/wiki/Manifold}
%%%%%
%%%%%
\end{document}
