\documentclass[12pt]{article}
\usepackage{pmmeta}
\pmcanonicalname{DNAInequality}
\pmcreated{2013-03-22 15:31:14}
\pmmodified{2013-03-22 15:31:14}
\pmowner{PrimeFan}{13766}
\pmmodifier{PrimeFan}{13766}
\pmtitle{DNA inequality}
\pmrecord{12}{37390}
\pmprivacy{1}
\pmauthor{PrimeFan}{13766}
\pmtype{Theorem}
\pmcomment{trigger rebuild}
\pmclassification{msc}{53A04}

\endmetadata

% this is the default PlanetMath preamble.  as your knowledge
% of TeX increases, you will probably want to edit this, but
% it should be fine as is for beginners.

% almost certainly you want these
\usepackage{amssymb}
\usepackage{amsmath}
\usepackage{amsfonts}

% used for TeXing text within eps files
%\usepackage{psfrag}
% need this for including graphics (\includegraphics)
%\usepackage{graphicx}
% for neatly defining theorems and propositions
%\usepackage{amsthm}
% making logically defined graphics
%%%\usepackage{xypic}

% there are many more packages, add them here as you need them

% define commands here
\begin{document}
Given $\Gamma$, a convex \PMlinkname{simple closed curve}{Curve} in the plane, and $\gamma$ a closed curve contained in $\Gamma$, then $M(\Gamma) \leq M(\gamma)$ where $M$ is the mean curvature function.

This was a conjecture due to S. Tabachnikov and was proved by Lagarias and Richardson of Bell Labs. The idea of the proof was to show that there was a way you could reduce a curve to the boundary of its convex hull so that if it holds for the boundary of the convex hull, then it holds for the curve itself.

\emph{Conjecture} : \emph{Equality holds iff $ \Gamma$ and $\gamma $ coincide}.

It's amazing how many questions are still open in the Elementary Differential Geometry of curves and surfaces.  Questions like this often serve as a great research opportunity for undergraduates.  It is also interesting to see if you could extend this result to curves on surfaces:

Theorem : If $ \Gamma $ is a circle on $S^2$ , and $\gamma$ is a closed curve contained in $\Gamma$ then $M(\Gamma) \leq M(\gamma)$ .

It is not known whether this result holds for $\Gamma$ a simple closed convex curve on $S^2$.

It is known also that this inequality does not hold in the hyperbolic plane.
%%%%%
%%%%%
\end{document}
