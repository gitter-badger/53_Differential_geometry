\documentclass[12pt]{article}
\usepackage{pmmeta}
\pmcanonicalname{NonOrientableSurface}
\pmcreated{2013-03-22 19:02:29}
\pmmodified{2013-03-22 19:02:29}
\pmowner{juanman}{12619}
\pmmodifier{juanman}{12619}
\pmtitle{non orientable surface}
\pmrecord{33}{41918}
\pmprivacy{1}
\pmauthor{juanman}{12619}
\pmtype{Definition}
\pmcomment{trigger rebuild}
\pmclassification{msc}{53A05}
\pmclassification{msc}{57M20}
\pmclassification{msc}{14J29}
\pmrelated{Surface}

% this is the default PlanetMath preamble.  as your knowledge
% of TeX increases, you will probably want to edit this, but
% it should be fine as is for beginners.

% almost certainly you want these
\usepackage{amssymb}
\usepackage{amsmath}
\usepackage{amsfonts}

% used for TeXing text within eps files
\usepackage{psfrag}
% need this for including graphics (\includegraphics)\usepackage{graphicx}
% for neatly defining theorems and propositions
%\usepackage{amsthm}
% making logically defined graphics
%%\usepackage{xypic}

% there are many more packages, add them here as you need them

% define commands here

\begin{document}
Non orientable phenomena are a consequence about the consideration of the tangent bundles regarding an embedding. One asks if $e:A\to B$ is an embedding then how the tangent bundles $TA$ and $TB$ relate?
 


For example: we could consider the core (simple close curve) of an cylinder $S^1\times I$ or in a Mobius band 
$M\ddot{o}$. First we can observe that if $C_1=S^1\times\{ \frac{1}{2}\}$  has as a regular neighborhood whose boundary is two component disconnected curve (in fact two disjoint circles), while the boundary of a regular neighborhood $N$ of the core curve $C\ddot{o}$ is a single circle: $\partial M\ddot{o}$.

In terms of tangent bundles we see that we can choose along the cylinder core a consistent normal in the sense that if this curve is traveled then at the end we have the same basis. In contrast with happens in $C\ddot{o}$ which after a full turn we are going to find a reflexion of the normal axe.   

Now employing the standard classification of closed surfaces we will construct another kind.

\begin{quote}
{\it These are the only types of orientable surfaces: $O_0$ the sphere; $O_1$ the two torus; $O_2=O_1\# O_1$ the bitoro; $O_3=O_1\# O_1\# O_1$ the tritoro,... etc,
i.e.
$$O_g=O_1\# \cdots\# O_1$$}
\end{quote}   

So, with the connected sum device we have:
\bigskip


The projective plane
\begin{eqnarray*}
{\mathbb{R}}P^2&=&(O_0\setminus{\rm{int}}D)\cup_{\partial}M\ddot{o}\\
&=&D\cup_{\partial}M\ddot{o}
\end{eqnarray*}

\bigskip

The Klein bottle
\begin{eqnarray*}
{\mathbb{R}}P^2\#{\mathbb{R}}P^2
&=&[O_0\setminus( {\rm{int}}D_1\sqcup {\rm{int}}D_2)]\cup_{\partial}
[(M\ddot{o})_1\sqcup (M\ddot{o})_2]\\
&=&(M\ddot{o})_1\cup_{\partial}(M\ddot{o})_2
\end{eqnarray*}

\bigskip

If we standarize as $N_1={\mathbb{R}}P^2$ and $N_2={\mathbb{R}}P^2\#{\mathbb{R}}P^2$, then the genus three non orientable surface is
\begin{eqnarray*}
N_3&=&{\mathbb{R}}P^2\#{\mathbb{R}}P^2\#{\mathbb{R}}P^2\\
&=&N_2\#{\mathbb{R}}P^2\\
&=&O_1\#{\mathbb{R}}P^2\\
&=&(
[O_0\setminus({\rm{int}}D_1\sqcup {\rm{int}}D_2\sqcup {\rm{int}}D_3)]\cup_{\partial}
[(M\ddot{o})_1\sqcup (M\ddot{o})_2\sqcup (M\ddot{o})_3]\\
&=&(O_1\setminus{\rm{int}}D)\cup_{\partial}M\ddot{o}\\
&=&(N_2\setminus{\rm{int}}D)\cup_{\partial}M\ddot{o}
\end{eqnarray*}


\bigskip

$\bullet\bullet\bullet$
\bigskip

\begin{center}
\includegraphics{moandco2}
\end{center}

\begin{xy}
(0,10)*+{\mathbb{R}^2}="f"; 
(13,10)*+{TM\ddot{o}}="e";
(15,0)*+{M\ddot{o}}="m";
{\ar@{.} "f";"e"}?*!/_2mm/{\subset};
{\ar "e";"m"}?*!/_3mm/{p};
\end{xy}


%%%%%
%%%%%
\end{document}
