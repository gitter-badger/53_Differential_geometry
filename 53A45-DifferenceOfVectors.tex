\documentclass[12pt]{article}
\usepackage{pmmeta}
\pmcanonicalname{DifferenceOfVectors}
\pmcreated{2013-03-22 17:47:19}
\pmmodified{2013-03-22 17:47:19}
\pmowner{pahio}{2872}
\pmmodifier{pahio}{2872}
\pmtitle{difference of vectors}
\pmrecord{9}{40249}
\pmprivacy{1}
\pmauthor{pahio}{2872}
\pmtype{Definition}
\pmcomment{trigger rebuild}
\pmclassification{msc}{53A45}
\pmsynonym{vector difference}{DifferenceOfVectors}
\pmsynonym{vector subtraction}{DifferenceOfVectors}
\pmrelated{CommonPointOfTriangleMedians}
\pmrelated{Difference2}
\pmrelated{ProvingThalesTheoremWithVectors}
\pmrelated{VectorValuedFunction2}
\pmdefines{difference vector}
\pmdefines{opposite vector}

% this is the default PlanetMath preamble.  as your knowledge
% of TeX increases, you will probably want to edit this, but
% it should be fine as is for beginners.

% almost certainly you want these
\usepackage{amssymb}
\usepackage{amsmath}
\usepackage{amsfonts}

% used for TeXing text within eps files
%\usepackage{psfrag}
% need this for including graphics (\includegraphics)
%\usepackage{graphicx}
% for neatly defining theorems and propositions
 \usepackage{amsthm}
% making logically defined graphics
%%%\usepackage{xypic}

% there are many more packages, add them here as you need them

\usepackage{pstricks}

% define commands here

\theoremstyle{definition}
\newtheorem*{thmplain}{Theorem}

\begin{document}
Let $\vec{a}$ and $\vec{b}$ be two vectors in the plane (or in a vector space).\, The {\em difference vector} or {\em difference} \,$\vec{a}\!-\!\vec{b}$\, of $\vec{a}$ and $\vec{b}$ is a vector $\vec{d}$\, such that
                 $$\vec{b}+\vec{d} = \vec{a}.$$
Thus we have
\begin{align}
\vec{b}+(\vec{a}\!-\!\vec{b}) = \vec{a}.
\end{align}

According to the procedure of forming the sum of vectors by setting the addends one after the other, the equation (1) tallies with the picture below; when the minuend and the subtrahend emanate from a common initial point, their difference vector can be directed from the terminal point of the subtrahend to the terminal point of the minuend. 

\begin{center}
\begin{pspicture}(-2,-1)(4,2.5)
\psdot[linecolor=black](0,0)
\psline[arrows=->,arrowsize=5pt,linecolor=black](0,0)(-1.5,2)
\psline[arrows=->,arrowsize=5pt,linecolor=red](-1.5,2)(4,1)
\psline[arrows=->,arrowsize=5pt,linecolor=blue](0,0)(4,1)
\rput[a](2.1,0.2){$\vec{a}$}
\rput[a](-1,0.7){$\vec{b}$}
\rput[a](1.6,1.76){$\vec{a}\!-\!\vec{b}$}
\rput(-2,-1){.}
\rput(4,2.5){.}
\end{pspicture}
\end{center}


\textbf{Remark.}\, It is easily seen that the difference $\vec{a}\!-\!\vec{b}$ is same as the sum vector
$$\vec{a}\!+\!(-\vec{b})$$
where $-\vec{b}$ is the {\em opposite vector} of $\vec{b}$:\, it may be represented by the directed line segment from the terminal point of $\vec{b}$ to the initial point of $\vec{b}$.


%%%%%
%%%%%
\end{document}
