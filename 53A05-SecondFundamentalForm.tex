\documentclass[12pt]{article}
\usepackage{pmmeta}
\pmcanonicalname{SecondFundamentalForm}
\pmcreated{2013-03-22 15:29:02}
\pmmodified{2013-03-22 15:29:02}
\pmowner{stevecheng}{10074}
\pmmodifier{stevecheng}{10074}
\pmtitle{second fundamental form}
\pmrecord{5}{37339}
\pmprivacy{1}
\pmauthor{stevecheng}{10074}
\pmtype{Definition}
\pmcomment{trigger rebuild}
\pmclassification{msc}{53A05}
\pmrelated{FirstFundamentalForm}
\pmrelated{ShapeOperator}
\pmrelated{NormalSection}
\pmrelated{NormalCurvatures}
\pmdefines{normal curvature}
\pmdefines{principal direction}
\pmdefines{principal curvature}
\pmdefines{Weingarten matrix}
\pmdefines{Weingarten map}
\pmdefines{Gauss map}

\usepackage{amssymb}
\usepackage{amsmath}
\usepackage{amsfonts}
\usepackage{amsthm}
\usepackage{enumerate}

% used for TeXing text within eps files
%\usepackage{psfrag}
% need this for including graphics (\includegraphics)
%\usepackage{graphicx}
% making logically defined graphics
%%%\usepackage{xypic}

% define commands here
\newcommand{\complex}{\mathbb{C}}
\newcommand{\real}{\mathbb{R}}
\newcommand{\rat}{\mathbb{Q}}
\newcommand{\nat}{\mathbb{N}}

\providecommand{\abs}[1]{\lvert#1\rvert}
\providecommand{\absW}[1]{\left\lvert#1\right\rvert}
\providecommand{\absB}[1]{\Bigl\lvert#1\Bigr\rvert}
\providecommand{\norm}[1]{\lVert#1\rVert}
\providecommand{\normW}[1]{\left\lVert#1\right\rVert}
\providecommand{\normB}[1]{\Bigl\lVert#1\Bigr\rVert}
\providecommand{\defnterm}[1]{\emph{#1}}

\DeclareMathOperator{\D}{D}
\DeclareMathOperator{\linspan}{span}
\newcommand{\TpM}{\mathrm{T}_p M}
\newcommand{\ctwo}{\mathcal{II}}
\newtheorem{prop}{Proposition}
\newtheorem{thm}{Theorem}
\begin{document}
\PMlinkescapeword{normal}

In classical differential geometry
the \defnterm{second fundamental form} is a symmetric bilinear form
defined on a differentiable surface $M$ embedded in $\real^3$, which in some sense
measures the curvature of $M$ in space.

To construct the second fundamental form requires a small digression.
After the digression we will discuss how it relates to the curvature of $M$.

\section*{Construction of the second fundamental form}
Consider the tangent planes $\TpM$ of the surface $M$ for each point $p \in M$.
There are two unit normals to $\TpM$.
Assuming $M$ is orientable, we can choose one of these unit \PMlinkname{normals}{MutualPositionsOfVectors}, $n(p)$,
so that $n(p)$ varies smoothly with $p$.

Since $ n(p)$ is a unit vector in $\real^3$, it may be considered
as a point on the sphere $S^2 \subset \real^3$.  Then we
have a map $ n\colon M \to S^2$.  It is called the \defnterm{normal map}
or \defnterm{Gauss map}.

The \defnterm{second fundamental form} is the tensor field $\ctwo$ on $M$ defined
by
\begin{equation}\label{def}
\ctwo_p(\xi, \eta) = - \langle \D n_p(\xi), \eta \rangle\,, \quad \xi, \eta \in \TpM\,, 
\end{equation}
where $\langle, \rangle$ is the dot product of $\real^3$,
and we consider the tangent planes of surfaces in $\real^3$
to be subspaces of $\real^3$.

The linear transformation $\D n_p$ is in reality the tangent mapping
$\D n_p \colon \TpM \to \mathrm{T}_{ n(p)} S^2$,
but since $\mathrm{T}_{ n(p)}S^2 = \TpM$ by the definition of $ n$,
we prefer to think of $\D n_p$ as $\D n_p \colon \TpM \to \TpM$.

The tangent map $\D n$, is often called the \defnterm{Weingarten map}.

\begin{prop}\label{symmetric}
The second fundamental form is a symmetric form.
\begin{proof}
This is a computation using a coordinate chart $\sigma$ for $M$.
Let $u, v$ be the corresponding names for the coordinates.
From the equation
\[
\left\langle  n, \frac{\partial \sigma}{\partial v} \right\rangle = 0\,,
\]
differentiating with respect to $u$ using the product rule gives
\begin{equation}\label{symmetric-one}
\begin{split}
\left\langle  n, \frac{\partial^2 \sigma}{\partial u \partial v} \right\rangle &=
- \left\langle \frac{\partial  n}{\partial u}, \frac{\partial \sigma}{\partial v} \right\rangle \\
&= - \left\langle \D  n\left( \frac{\partial \sigma}{\partial u} \right), \frac{\partial \sigma}{\partial v} \right\rangle  \\
&= \ctwo\left(\frac{\partial \sigma}{\partial u}, \frac{\partial \sigma}{\partial v}\right)\,.
\end{split}
\end{equation}
(The second equality follows from the definition of the tangent map $\D n$.)
Reversing the roles of $u, v$ and repeating the last derivation,
we obtain also:
\begin{equation}\label{symmetric-two}
\left\langle  n, \frac{\partial^2 \sigma}{\partial u \partial v} \right\rangle = \left\langle  n, \frac{\partial^2 \sigma}{\partial v \partial u} \right\rangle =
\ctwo\left(\frac{\partial \sigma}{\partial v}, \frac{\partial \sigma}{\partial u}\right)\,.
\end{equation}
Since $\partial \sigma/\partial u$ and $\partial \sigma/\partial v$
form a basis for $\TpM$, combining
\eqref{symmetric-one} and \eqref{symmetric-two} proves that $\mathcal{II}$ is symmetric.
\end{proof}
\end{prop}

In view of Proposition \ref{symmetric},
it is customary to regard the second fundamental form
as a quadratic form,
as it done with the first fundamental form.
Thus, the second fundamental form is referred to with the following expression\footnote{
Unfortunately the coefficient $M$ here clashes with our use of the letter $M$ for
the surface (manifold), but whenever we write $M$, the context should make clear 
which meaning is intended.
The use of the symbols $L, M, N$ for the coefficients of the second fundamental form
is standard, but probably was established long before
anyone thought about manifolds.}:
\[
L \, du^2 + 2M \, dudv + N \, dv^2\,.
\]
Compare with the tensor notation
\[
\ctwo = L \, du \otimes du + M \, du \otimes dv + M \, dv \otimes du + N dv \otimes dv\,.
\]
Or in matrix form (with respect to the coordinates $u, v$),
\[
\ctwo = \begin{pmatrix}
L & M \\
M & N
\end{pmatrix}\,.
\]

\section*{Curvature of curves on a surface}
Let $\gamma$ be a curve lying on the surface $M$, parameterized by arc-length.
Recall that the curvature $\kappa(s)$ of $\gamma$ at $s$
is $\gamma''(s)$.  If we want to measure the curvature of the surface, it is natural
to consider the component of $\gamma''(s)$ in the normal $n(\gamma(s))$.
Precisely, this quantity is
\[
\langle \gamma''(s), n(\gamma(s)) \rangle\,,
\]
and is called the \defnterm{normal curvature} of $\gamma$ on $M$.

So to study the curvature of $M$, we ignore the component of the curvature of $\gamma$ 
in the tangent plane of $M$.
Also, physically speaking, the normal curvature is proportional to the acceleration
required to keep a moving particle \emph{on} the surface $M$.

We now come to the motivation for defining the second fundamental form:
\begin{prop}\label{principal-curvature}
Let $\gamma$ be a curve on $M$, parameterized by arc-length,
and $\gamma(s) = p$.
Then
\[
\langle \gamma''(s), n(p) \rangle = \ctwo\bigl(\gamma'(s), \gamma'(s)\bigr)\,.
\]
\begin{proof}
From the equation
\[
\langle n(\gamma(s)), \gamma'(s) \rangle = 0\,,
\]
differentiate with respect to $s$:
\begin{align*}
\langle n(\gamma(s)), \gamma''(s) \rangle &= - \left\langle \frac{d}{ds} n(\gamma(s)), \gamma'(s) \right\rangle \\
&= -\bigl\langle \D n(\gamma'(s)), \gamma'(s) \bigr\rangle \\
&= \ctwo\bigl(\gamma'(s), \gamma'(s)\bigr)\,. \qedhere
\end{align*}
\end{proof}
\end{prop}

It is now time to mention an important consequence of Proposition \ref{symmetric}:
the fact that $\ctwo$ is symmetric means that 
$-\D n$ is \emph{self-adjoint} with respect to the inner product $\mathcal{I}$
(the first fundamental form).  So, if $-\D n$ is expressed
as a matrix with orthonormal coordinates (with respect to $\mathcal{I}$),
then the matrix is symmetric.  (The minus sign in front of $\D n$
is to make the formulas work out nicely.)

Certain theorems in linear algebra tell us
that, $-\D n_p$ being self-adjoint, it has an orthonormal basis of 
eigenvectors $e_1, e_2$ with corresponding eigenvalues $\kappa_1 \leq \kappa_2$.
These eigenvalues are called the \defnterm{principal curvatures}
of $M$ at $p$.  
The eigenvectors $e_1, e_2$ are the \defnterm{principal directions}.
The terminology is justified by the following theorem:

\begin{thm}[Euler's Theorem]
The normal curvature of a curve $\gamma$ has the form
\[
\langle \gamma''(s), n(p) \rangle = \kappa_1 \, \cos^2 \theta + \kappa_2 \, \sin^2 \theta\,, \quad p = \gamma(s)\,.
\]
It follows that the minimum possible normal curvature is $\kappa_1$,
and the maximum possible is $\kappa_2$.
\begin{proof}
Since $e_1, e_2$ form an orthonormal basis for $\TpM$, we may write
\[
\gamma'(s) = \cos \theta \, e_1 + \sin \theta \, e_2
\]
for some angle $\theta$.
Then
\begin{align*}
\langle \gamma''(s), n(p) \rangle &= \ctwo\bigl(\gamma'(s), \gamma'(s) \bigr)
\\
&= \bigl\langle -\D n_p(\gamma'(s)), \gamma'(s) \bigr\rangle \\
&= \langle \kappa_1 \cos \theta \, e_1 + \kappa_2 \sin \theta \, e_2, \cos \theta \, e_1 + \sin \theta \, e_2 \rangle
\\
&= \kappa_1 \, \cos^2 \theta + \kappa_2 \, \sin^2 \theta\,. \qedhere
\end{align*}
\end{proof}
\end{thm}

\section*{Matrix representations of second fundamental form and Weingarten map}

At this point, we should find the explicit prescriptions
for calculating the second fundamental form and the Weingarten map.

Let $\sigma$ be a coordinate chart for $M$,
and $u,v$ be the names of the coordinates.
For a test vector $\xi \in \TpM$, we write $\xi_u$ and $\xi_v$
for the $u,v$ coordinates of $\xi$.

We compute the matrix $W$ for $-\D n$ in $u,v$-coordinates.
We have
\begin{align*}
\begin{pmatrix}
\xi_u & \xi_v
\end{pmatrix}
\begin{pmatrix}
L & M \\
M & N
\end{pmatrix} 
\begin{pmatrix}
\xi_u \\ \xi_v
\end{pmatrix}
&= \ctwo(\xi, \xi) = \langle -\D n(\xi), \xi \rangle \\
&= \left(
Q \begin{pmatrix}
\xi_u \\
\xi_v
\end{pmatrix}
\right)^\mathrm{T}
Q W \begin{pmatrix}
\xi_u \\
\xi_v
\end{pmatrix} \\
&= 
\begin{pmatrix}
\xi_u &
\xi_v
\end{pmatrix}
\, (Q^\mathrm{T}
Q) \, W \begin{pmatrix}
\xi_u \\
\xi_v
\end{pmatrix}\,,
\end{align*}
where $Q$ is the matrix that changes from $u,v$-coordinates to orthonormal
coordinates for $\TpM$ --- this is necessary to compute the inner product.
But 
\[
Q^\mathrm{T} Q = \begin{pmatrix}
E & F \\
F & G
\end{pmatrix} = \mathcal{I} \quad \text{(the first fundamental form),}
\]
because $Q$ is the matrix
with columns $\partial\sigma/\partial u$ and $\partial\sigma/\partial v$
expressed in orthonormal coordinates.

(More to be written...)

\begin{thebibliography}{3}
\bibitem{Spivak}
Michael Spivak.  {\it A Comprehensive Introduction to Differential Geometry}, volumes I and II.
 Publish or Perish, 1979.
\bibitem{Pressley}
Andrew Pressley. {\it Elementary Differential Geometry}.  Springer-Verlag, 2003.
\end{thebibliography}
%%%%%
%%%%%
\end{document}
