\documentclass[12pt]{article}
\usepackage{pmmeta}
\pmcanonicalname{PolarCurve}
\pmcreated{2013-03-22 15:17:08}
\pmmodified{2013-03-22 15:17:08}
\pmowner{CWoo}{3771}
\pmmodifier{CWoo}{3771}
\pmtitle{polar curve}
\pmrecord{6}{37076}
\pmprivacy{1}
\pmauthor{CWoo}{3771}
\pmtype{Example}
\pmcomment{trigger rebuild}
\pmclassification{msc}{53A04}
\pmclassification{msc}{51-01}
\pmsynonym{lima\c{c}on}{PolarCurve}
\pmrelated{AreaOfPlaneRegion}
\pmrelated{CassiniOval}
\pmdefines{lemniscate}
\pmdefines{rhodonea}
\pmdefines{cardioid}
\pmdefines{limacon}

\endmetadata

\usepackage{amssymb,amscd}
\usepackage{amsmath}
\usepackage{amsfonts}

% used for TeXing text within eps files
%\usepackage{psfrag}
% need this for including graphics (\includegraphics)
\usepackage{graphicx}
% for neatly defining theorems and propositions
%\usepackage{amsthm}
% making logically defined graphics
%%%\usepackage{xypic}

% define commands here
\begin{document}
Polar curves are plane curves in $\mathbb{R}^2$ that are expressed in polar coordinates $(r,\theta)$.  The two simplest polar curves are obtained when one of the two coordinates is set to be a constant.  If the first coordinate is set to a constant $r$, we have a circle with radius $\lvert r \rvert$, or a point when $r=0$.  When the second coordinate is the constant instead, say $c$, we have a straight line through the (polar) origin, with slope = $\tan c$.


\begin{center}
\includegraphics{polar_curve_1.eps}
\hspace{1in}
\includegraphics{polar_curve_2.eps}
\end{center}

Using polar coordinates, one can generate many visually pleasing
curves.  Below are some of the most popular ones.

\begin{center}
\includegraphics{polar_curve_3.eps}
\end{center}
\vspace{10pt}
\begin{center}
\includegraphics{polar_curve_4.eps}
\hspace{1in}
\includegraphics{polar_curve_5.eps}
\end{center}
\vspace{10pt}
\begin{center}
\includegraphics{polar_curve_6.eps}
\end{center}
\vspace{10pt}
\begin{center}
\includegraphics{polar_curve_7.eps}
\end{center}
\vspace{10pt}
\begin{center}
\includegraphics{polar_curve_8.eps}
\hspace{1in}
\includegraphics{polar_curve_9.eps}
\end{center}
%%%%%
%%%%%
\end{document}
