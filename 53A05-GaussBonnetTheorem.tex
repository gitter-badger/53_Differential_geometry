\documentclass[12pt]{article}
\usepackage{pmmeta}
\pmcanonicalname{GaussBonnetTheorem}
\pmcreated{2013-03-22 16:36:37}
\pmmodified{2013-03-22 16:36:37}
\pmowner{rspuzio}{6075}
\pmmodifier{rspuzio}{6075}
\pmtitle{Gauss-Bonnet theorem}
\pmrecord{9}{38807}
\pmprivacy{1}
\pmauthor{rspuzio}{6075}
\pmtype{Theorem}
\pmcomment{trigger rebuild}
\pmclassification{msc}{53A05}

% this is the default PlanetMath preamble.  as your knowledge
% of TeX increases, you will probably want to edit this, but
% it should be fine as is for beginners.

% almost certainly you want these
\usepackage{amssymb}
\usepackage{amsmath}
\usepackage{amsfonts}

% used for TeXing text within eps files
%\usepackage{psfrag}
% need this for including graphics (\includegraphics)
%\usepackage{graphicx}
% for neatly defining theorems and propositions
%\usepackage{amsthm}
% making logically defined graphics
%%%\usepackage{xypic}

% there are many more packages, add them here as you need them

% define commands here

\begin{document}
(Carl Friedrich Gauss and Pierre Ossian Bonnet) Given a two-dimensional compact Riemannian manifold $M$ with boundary, 
Gaussian curvature of points $G$ and geodesic curvature of points $g_x$ on the boundary $\partial M$, it is the 
case that 
\[
\int_M G \, dA + \int_{\partial M}g_x ds = 2\pi\chi(M),
\] 
where $\chi(M)$ is the Euler characteristic of the manifold, $dA$ denotes the measure with respect to area, and $ds$ denotes the measure with respect to arclength on the boundary.  This theorem expresses a topological invariant in
terms of geometrical information.
%%%%%
%%%%%
\end{document}
