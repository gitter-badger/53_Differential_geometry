\documentclass[12pt]{article}
\usepackage{pmmeta}
\pmcanonicalname{Quadratrix}
\pmcreated{2013-03-22 18:05:31}
\pmmodified{2013-03-22 18:05:31}
\pmowner{pahio}{2872}
\pmmodifier{pahio}{2872}
\pmtitle{quadratrix}
\pmrecord{12}{40630}
\pmprivacy{1}
\pmauthor{pahio}{2872}
\pmtype{Definition}
\pmcomment{trigger rebuild}
\pmclassification{msc}{53A04}
\pmsynonym{quadratrix of Hippias}{Quadratrix}
\pmsynonym{quadratrix of Dinostratus}{Quadratrix}

\endmetadata

% this is the default PlanetMath preamble.  as your knowledge
% of TeX increases, you will probably want to edit this, but
% it should be fine as is for beginners.

% almost certainly you want these
\usepackage{amssymb}
\usepackage{amsmath}
\usepackage{amsfonts}

% used for TeXing text within eps files
%\usepackage{psfrag}
% need this for including graphics (\includegraphics)
%\usepackage{graphicx}
% for neatly defining theorems and propositions
 \usepackage{amsthm}
% making logically defined graphics
%%%\usepackage{xypic}
\usepackage{pstricks}
\usepackage{pst-plot}

% there are many more packages, add them here as you need them

% define commands here

\theoremstyle{definition}
\newtheorem*{thmplain}{Theorem}

\begin{document}
\begin{center}
\begin{pspicture}(-1,-1)(4,3)
\psaxes[Dx=9,Dy=9]{->}(0,0)(-0.5,-0.5)(4,3)
\rput(4,-0.2){$x$}
\rput(-0.24,3){$y$}
\psline(0,0)(3.2,2)(3.2,0)
\psdot[linecolor=blue](3.2,2)
\psarc(0,0){0.5}{0}{34}
\psline(3,0)(3,0.2)(3.2,0.2)
\rput(0.7,0.2){$t$}
\rput(3.5,1){$kt$}
\rput(3.4,2.25){$(x,y)$}
\end{pspicture}
\end{center}

Let the polar angle $\theta$ and the ordinate of the point \,$(x,\,y)$\, of the plane be \PMlinkname{proportional}{Variation} to a parametre $t$, e.g. such that\, $\theta = t$,\; $y = kt$.\, The above diagram shows that\, $\displaystyle\frac{x}{kt} = \cot{t}$.\, Into this we can substitute\, $\displaystyle t = \frac{y}{k}$, whence we get an equation
\begin{align}
x \;=\; y\cot\frac{y}{k}
\end{align}
between $x$ and $y$.\, The given proportionalities as locus condition, the point\, $(x,\,y)$\, draws a plane curve called {\em quadratrix}.\, If we change the $x$ and $y$ coordinates, the equation of the quadratrix is
\begin{align}
y \;=\; x\cot\frac{x}{k}.
\end{align}
The equation (2) defines an even function \,$x \mapsto y$, where for $x$ is allowed all real values at which the right hand side of (2) is defined, thus\, $x \neq n\pi k$\; ($n \in \mathbb{Z}$).

\begin{center}
\begin{pspicture}(-5.5,-6.5)(5.5,5)
\psaxes[Dx=9,Dy=9]{->}(0,0)(-1.5,-5.5)(3.5,5.5)
\psplot[linecolor=blue]{-0.95}{-0.01}{x 180 mul cos x 180 mul sin div x mul}
\psplot[linecolor=blue]{0.01}{0.95}{x 180 mul cos x 180 mul sin div x mul}
\psplot[linecolor=blue]{1.06}{1.90}{x 180 mul cos x 180 mul sin div x mul}
\psplot[linecolor=blue]{2.10}{2.86}{x 180 mul cos x 180 mul sin div x mul}
\psline[linestyle=dotted](-1,-5)(-1,5)
\psline[linestyle=dotted](1,-5)(1,5)
\psline[linestyle=dotted](2,-5)(2,5)
\psline[linestyle=dotted](3,-5)(3,5)
\rput(3.5,-0.2){$x$}
\rput(-0.24,5.4){$y$}
\rput(-5.5,-6.5){.}
\rput(5.5,5){.}
\end{pspicture}
\end{center}

The quadratrix was used by the ancient Greek geometers for squaring the circle, the name comes from the Latin {\em quadratrix} = `a feminine squarer'.\\

\PMlinkexternal{Wiki}{http://de.wikipedia.org/wiki/Quadratrix}

%%%%%
%%%%%
\end{document}
