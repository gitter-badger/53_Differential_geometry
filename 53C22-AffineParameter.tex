\documentclass[12pt]{article}
\usepackage{pmmeta}
\pmcanonicalname{AffineParameter}
\pmcreated{2013-03-22 14:35:47}
\pmmodified{2013-03-22 14:35:47}
\pmowner{rspuzio}{6075}
\pmmodifier{rspuzio}{6075}
\pmtitle{affine parameter}
\pmrecord{9}{36164}
\pmprivacy{1}
\pmauthor{rspuzio}{6075}
\pmtype{Definition}
\pmcomment{trigger rebuild}
\pmclassification{msc}{53C22}
\pmdefines{affinely-parameterized}

\endmetadata

% this is the default PlanetMath preamble.  as your knowledge
% of TeX increases, you will probably want to edit this, but
% it should be fine as is for beginners.

% almost certainly you want these
\usepackage{amssymb}
\usepackage{amsmath}
\usepackage{amsfonts}

% used for TeXing text within eps files
%\usepackage{psfrag}
% need this for including graphics (\includegraphics)
%\usepackage{graphicx}
% for neatly defining theorems and propositions
%\usepackage{amsthm}
% making logically defined graphics
%%%\usepackage{xypic}

% there are many more packages, add them here as you need them

% define commands here
\begin{document}
Given a geodesic curve, an \emph{affine parameterization} for that curve is a parameterization by a parameter $t$ such that the parametric equations for the curve satisfy the geodesic equation.

Put another way, if one picks a parameterization of a geodesic curve by an arbitrary parameter $s$ and sets $u^\mu = dx^\mu / ds$, then we have
 $$u^\mu \nabla_\mu u^\nu = f(s) u^\nu$$
for some function $f$.  In general, the right hand side of this equation does not equal zero --- it is only zero in the special case where $t$ is an affine parameter.

The reason for the name ``affine parameter'' is that, if $t_1$ and $t_2$ are affine parameters for the same geodesic curve, then they are related by an affine transform, i.e. there exist constants $a$ and $b$ such that
 \[t_1 = a t_2 + b\]
Conversely, if $t$ is an affine parameter, then $at + b$ is also an affine parameter.

From this it follows that an affine parameter $t$ is uniquely determined if we specify its value at two points on the geodesic or if we specify both its value and the value of $dx^\mu / dt$ at a single point of the geodesic.
%%%%%
%%%%%
\end{document}
