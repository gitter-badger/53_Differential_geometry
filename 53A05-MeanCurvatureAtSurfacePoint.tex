\documentclass[12pt]{article}
\usepackage{pmmeta}
\pmcanonicalname{MeanCurvatureAtSurfacePoint}
\pmcreated{2013-03-22 17:26:56}
\pmmodified{2013-03-22 17:26:56}
\pmowner{pahio}{2872}
\pmmodifier{pahio}{2872}
\pmtitle{mean curvature at surface point}
\pmrecord{8}{39830}
\pmprivacy{1}
\pmauthor{pahio}{2872}
\pmtype{Theorem}
\pmcomment{trigger rebuild}
\pmclassification{msc}{53A05}
\pmclassification{msc}{26B05}
\pmclassification{msc}{26A24}
%\pmkeywords{normal section}
%\pmkeywords{curvature}
\pmrelated{AdditionFormulasForSineAndCosine}
\pmrelated{GaussianCurvature}
\pmrelated{MeanCurvature}
\pmdefines{mean curvature}

\endmetadata

% this is the default PlanetMath preamble.  as your knowledge
% of TeX increases, you will probably want to edit this, but
% it should be fine as is for beginners.

% almost certainly you want these
\usepackage{amssymb}
\usepackage{amsmath}
\usepackage{amsfonts}

% used for TeXing text within eps files
%\usepackage{psfrag}
% need this for including graphics (\includegraphics)
%\usepackage{graphicx}
% for neatly defining theorems and propositions
 \usepackage{amsthm}
% making logically defined graphics
%%%\usepackage{xypic}

% there are many more packages, add them here as you need them

% define commands here

\theoremstyle{definition}
\newtheorem*{thmplain}{Theorem}

\begin{document}
Let $P$ be a point on the surface \,$F(x,\,y,\,z) = 0$\, where the function $F$ is twice continuously differentiable on a neighbourhood of $P$.  Then the normal curvature $\varkappa_\theta$ at $P$ is, by Euler's theorem, \PMlinkescapetext{expressible} via the principal curvatures $\varkappa_1$ and $\varkappa_2$ as
\begin{align}
     \varkappa_\theta = \varkappa_1\cos^2\theta+\varkappa_2\sin^2\theta,
\end{align}
where $\theta$ is the \PMlinkname{angle between}{AngleBetweenTwoPlanes} the normal section plane corresponding $\varkappa_1$ and the normal section plane corresponding $\varkappa_\theta$.  When we apply (1) by taking instead $\theta$ the angle $\theta\!+\!\frac{\pi}{2}$, we may write
   $$\varkappa_{\theta+\frac{\pi}{2}} = 
\varkappa_1\sin^2\theta+\varkappa_2\cos^2\theta.$$
Adding this equation to (1) then yields
   $$\frac{\varkappa_\theta+\varkappa_{\theta+\frac{\pi}{2}}}{2} = 
\frac{\varkappa_1+\varkappa_2}{2}.$$

The contents of this result is the

\textbf{Theorem.}  The arithmetic mean of the \PMlinkname{curvatures}{CurvaturePlaneCurve} of two perpendicular normal sections has a \PMlinkescapetext{constant} value, which is equal to the arithmetic mean of the principal curvatures.  This mean is called the {\em mean curvature} at the point in question.

\begin{thebibliography}{8}
\bibitem{lindelof}{\sc Ernst Lindel\"of}: {\em Differentiali- ja integralilasku
ja sen sovellutukset II}.\, Mercatorin Kirjapaino Osakeyhti\"o, Helsinki (1932).
\end{thebibliography}
%%%%%
%%%%%
\end{document}
