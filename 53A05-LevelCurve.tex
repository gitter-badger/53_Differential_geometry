\documentclass[12pt]{article}
\usepackage{pmmeta}
\pmcanonicalname{LevelCurve}
\pmcreated{2013-03-22 17:35:27}
\pmmodified{2013-03-22 17:35:27}
\pmowner{pahio}{2872}
\pmmodifier{pahio}{2872}
\pmtitle{level curve}
\pmrecord{13}{40003}
\pmprivacy{1}
\pmauthor{pahio}{2872}
\pmtype{Definition}
\pmcomment{trigger rebuild}
\pmclassification{msc}{53A05}
\pmclassification{msc}{53A04}
\pmclassification{msc}{51M04}
\pmsynonym{contour curve}{LevelCurve}
\pmsynonym{isopleth}{LevelCurve}
%\pmkeywords{level set}
\pmrelated{LevelSet}
\pmrelated{ConvexAngle}
\pmdefines{level surface}
\pmdefines{contour surface}

% this is the default PlanetMath preamble.  as your knowledge
% of TeX increases, you will probably want to edit this, but
% it should be fine as is for beginners.

% almost certainly you want these
\usepackage{amssymb}
\usepackage{amsmath}
\usepackage{amsfonts}

% used for TeXing text within eps files
%\usepackage{psfrag}
% need this for including graphics (\includegraphics)
%\usepackage{graphicx}
% for neatly defining theorems and propositions
 \usepackage{amsthm}
% making logically defined graphics
%%%\usepackage{xypic}

% there are many more packages, add them here as you need them

% define commands here

\theoremstyle{definition}
\newtheorem*{thmplain}{Theorem}

\begin{document}
The {\em level curves} (in German {\em Niveaukurve}, in French {\em ligne de niveau}) of a surface 
\begin{align}
z \;=\; f(x,\,y)
\end{align}
in $\mathbb{R}^3$ are the intersection curves of the surface and the planes \,$z = \,\mathrm{constant}$.  Thus the projections of the level curves on the $xy$-plane have equations of the form
\begin{align}
f(x,\,y) \;=\; c
\end{align}
where $c$ is a constant.\\

For example, the level curves of the \PMlinkname{hyperbolic paraboloid}{RuledSurface} \,$z = xy$\, are the rectangular hyperbolas \;$xy = c$\, (cf. \PMlinkname{this entry}{GraphOfEquationXyConstant}).\\

The gradient \,$f'_x(x,\,y)\,\vec{i}\!+\!f'_y(x,\,y)\,\vec{j}$\, of the function $f$ in any point of the surface (1) is perpendicular to the level curve (2), since the slope of the gradient is $\displaystyle\frac{f'_y}{f'_x}$ and the slope of the level curve is $\displaystyle-\frac{f'_x}{f'_y}$, whence the slopes are opposite inverses.\\

Analogically one can define the {\em level surfaces} (or {\em contour surfaces})
\begin{align}
F(x,\,y,\,z) \;=\; c
\end{align}
for a function $F$ of three variables $x$, $y$, $z$.  The gradient of $F$ in a point\, $(x,\,y,\,z)$\, is parallel to the surface normal of the level surface passing through this point.




%%%%%
%%%%%
\end{document}
