\documentclass[12pt]{article}
\usepackage{pmmeta}
\pmcanonicalname{SymplecticVectorSpace}
\pmcreated{2013-03-22 13:32:22}
\pmmodified{2013-03-22 13:32:22}
\pmowner{matte}{1858}
\pmmodifier{matte}{1858}
\pmtitle{symplectic vector space}
\pmrecord{11}{34138}
\pmprivacy{1}
\pmauthor{matte}{1858}
\pmtype{Definition}
\pmcomment{trigger rebuild}
\pmclassification{msc}{53D05}
\pmdefines{symplectic vector space}
\pmdefines{linear symplectomorphism}

% this is the default PlanetMath preamble.  as your knowledge
% of TeX increases, you will probably want to edit this, but
% it should be fine as is for beginners.

% almost certainly you want these
\usepackage{amssymb}
\usepackage{amsmath}
\usepackage{amsfonts}

% used for TeXing text within eps files
%\usepackage{psfrag}
% need this for including graphics (\includegraphics)
%\usepackage{graphicx}
% for neatly defining theorems and propositions
%\usepackage{amsthm}
% making logically defined graphics
%%%\usepackage{xypic}

% there are many more packages, add them here as you need them

% define commands here
\begin{document}
A \emph{symplectic vector space} $(V,\omega)$ 
is a finite dimensional real vector space $V$ equipped with
an alternating non-degenerate 2-tensor, i.e.,
a bilinear map $\omega\colon V\times V\rightarrow\mathbb R$ 
that satisfies the following properties:
\begin{enumerate}
\item Alternating: For all $v,w\in V$, $\omega(v,w)=-\omega(w,v)$.
\item Non-degenerate: If  $\omega(v,w)=0$ for all $w\in V$, then $v=0$.
\end{enumerate}
The tensor $\omega$ is
called a \PMlinkescapetext{\emph{symplectic form}} for $V$.

A linear automorphism $T\in\mathrm{Aut}(V)$ is called \emph{linear symplectomorphism} when $T^*\omega=\omega$, i.e. $$ \omega(Tv,Tw)=\omega(v,w)\ \ \forall v,w\in W.$$
Linear symplectomorphisms of $(V,\omega)$ form a group (under composition of linear map) that is denoted by $\mathrm{Sp(V,\omega)}$.

One can show that a symplectic vector space is always even dimensional \cite{duff}.

\begin{thebibliography}{9}
\bibitem {duff} D. McDuff, D. Salamon,
        \emph{Introduction to Symplectic Topology},
        Clarendon Press, 1997.
\end{thebibliography}
%%%%%
%%%%%
\end{document}
