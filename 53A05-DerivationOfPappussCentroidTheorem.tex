\documentclass[12pt]{article}
\usepackage{pmmeta}
\pmcanonicalname{DerivationOfPappussCentroidTheorem}
\pmcreated{2013-03-22 19:36:11}
\pmmodified{2013-03-22 19:36:11}
\pmowner{pahio}{2872}
\pmmodifier{pahio}{2872}
\pmtitle{derivation of Pappus's centroid theorem}
\pmrecord{7}{42595}
\pmprivacy{1}
\pmauthor{pahio}{2872}
\pmtype{Derivation}
\pmcomment{trigger rebuild}
\pmclassification{msc}{53A05}

% this is the default PlanetMath preamble.  as your knowledge
% of TeX increases, you will probably want to edit this, but
% it should be fine as is for beginners.

% almost certainly you want these
\usepackage{amssymb}
\usepackage{amsmath}
\usepackage{amsfonts}

% used for TeXing text within eps files
%\usepackage{psfrag}
% need this for including graphics (\includegraphics)
%\usepackage{graphicx}
% for neatly defining theorems and propositions
 \usepackage{amsthm}
% making logically defined graphics
%%%\usepackage{xypic}

% there are many more packages, add them here as you need them

% define commands here

\theoremstyle{definition}
\newtheorem*{thmplain}{Theorem}

\begin{document}
\textbf{I.}\, Let $s$ denote the arc rotating about the $x$-axis (and its length) and $R$ be the $y$-coordinate of the centroid of the arc.\, If the arc may be given by the equation
$$y \;=\; y(x)$$
where\, $a \le x \le b$, the area of the formed surface of revolution is
$$A \;=\; 2\pi\!\int_a^b\!y(x)\sqrt{1\!+\![y'(x)]^2}\,dx.$$
This can be concisely written
\begin{align}
A \;=\; 2\pi\!\int_s\!y\,ds
\end{align}
since differential-geometrically, the product $\sqrt{1\!+\![y'(x)]^2}\,dx$ is the arc-element.\, We rewrite (1) as
$$A \;=\; s\cdot2\pi\cdot\frac{1}{s}\!\int_s\!y\,ds.$$
Here, the last factor is the ordinate of the centroid of the rotating arc, whence we have the result
$$A \;=\; s \cdot2 \pi R$$
which states the first Pappus's centroid theorem.\\


\textbf{II.}\, For deriving the second Pappus's centroid theorem, we suppose that the region defined by
$$a \;\le\, x \;\le\; b, \quad 0 \;\le\; y_1(x) \;\le\, y \;\le\; y_2(x),$$
having the area $A$ and the centroid with the ordinate $R$, rotates about the $x$-axis and forms the solid of revolution with the volume $V$.\, The centroid of the area-element between the arcs\, $y = y_1(x)$\, and\, $y = y_2(x)$\, is $[y_2(x)\!+\!y_1(x)]/2$ when the abscissa is $x$; the area of this element with the width $dx$ is $[y_2(x)\!-\!y_1(x)]\,dx$.\, Thus we get the equation
$$R \;=\; \frac{1}{A}\int_a^b\frac{y_2(x)\!+\!y_1(x)}{2}[y_2(x)\!-y_1(x)]\,dx$$
which may be written shortly
\begin{align}
R \;=\; \frac{1}{2A}\int_a^b(y_2^2\!-\!y_1^2)\,dx.
\end{align}
The volume of the solid of revolution is
$$V \;=\; \pi\!\int_a^b(y_2^2\!-\!y_1^2)\,dx \;=\; A\cdot2\pi\cdot\frac{1}{2A}\!\int_a^b(y_2^2\!-\!y_1^2)\,dx.$$
By (2), this attains the form
$$V \;=\; A\cdot 2\pi R.$$

%%%%%
%%%%%
\end{document}
