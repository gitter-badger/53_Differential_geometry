\documentclass[12pt]{article}
\usepackage{pmmeta}
\pmcanonicalname{HolomorphicMappingOfCurveAndTangent}
\pmcreated{2013-03-22 18:42:19}
\pmmodified{2013-03-22 18:42:19}
\pmowner{pahio}{2872}
\pmmodifier{pahio}{2872}
\pmtitle{holomorphic mapping of curve and tangent}
\pmrecord{9}{41469}
\pmprivacy{1}
\pmauthor{pahio}{2872}
\pmtype{Topic}
\pmcomment{trigger rebuild}
\pmclassification{msc}{53A30}
\pmclassification{msc}{30E20}
%\pmkeywords{conformal mapping}
\pmdefines{directly conformal}

% this is the default PlanetMath preamble.  as your knowledge
% of TeX increases, you will probably want to edit this, but
% it should be fine as is for beginners.

% almost certainly you want these
\usepackage{amssymb}
\usepackage{amsmath}
\usepackage{amsfonts}

% used for TeXing text within eps files
%\usepackage{psfrag}
% need this for including graphics (\includegraphics)
%\usepackage{graphicx}
% for neatly defining theorems and propositions
 \usepackage{amsthm}
% making logically defined graphics
%%%\usepackage{xypic}
\usepackage{pstricks}
\usepackage{pst-plot}

% there are many more packages, add them here as you need them

% define commands here

\theoremstyle{definition}
\newtheorem*{thmplain}{Theorem}

\begin{document}
Let $D$ be a domain of the complex plane and the function \,$f\!:\,D \to \mathbb{C}$\, be holomorphic.\, Then for each point $z$ of 
$D$ there is a corresponding point \,$w = f(z)\,\in \mathbb{C}$;\, we think that $z$ and $w$ both lie in their own complex planes, $z$-plane and $w$-plane.

Since $f$ is continuous in $D$, if $z$ draws a continuous curve $\gamma$ in $D$ then its image point $w$ also draws a continuous curve $\gamma_w$.\, Let $z_0$ and $z_0\!+\!\Delta z$ be two points on $\gamma$ and $w_0$ and 
$w_0\!+\!\Delta w$ their image points on $\gamma_w$.

\begin{center}
\begin{pspicture}(-5.5,-3)(5.5,4)
\psline(-5.3,0.09)(-3.6,1.96)
\psline[arrows=->](-5.3,0.09)(-2.5,-1.59)
\psplot[linecolor=blue]{-5.3}{-3.1}{x 5 add x 5 add mul}
\psdot[linecolor=blue](-5.3,0.09)
\psdot[linecolor=blue](-3.6,1.96)
\rput(-5.5,-0.15){$z_0$}
\rput(-2.75,1.96){$z_0\!+\!\Delta z$}
\rput(-3.76,0.6){$\Delta s$}
\rput(-4.7,1.1){$k$}
\rput(-2.2,-1.5){$(\varphi)$}
\rput(-2.75,3.6){$\gamma$}
\rput(-4.5,-2.5){$z$-plane}

\rput(0,-0.2){$w_0$}
\psplot[linecolor=red]{0.1}{1.6}{x 1 mul 2 x sub div sqrt x mul}
\psline(0.1,0)(1.5,2.5)
\psline[arrows=->](0.1,0)(2.5,0.9)
\psdot[linecolor=red](0.1,0)
\psdot[linecolor=red](1.47,2.45)
\rput(2.4,2.45){$w_0\!+\!\Delta w$}
\rput(1.8,3.4){$\gamma_w$}
\rput(1.5,1.2){$\Delta s_w$}
\rput(0.5,1.3){$k_w$}
\rput(3,1){$(\varphi_w)$}
\rput(1,-2.5){$w$-plane}
\end{pspicture}
\end{center}

We suppose still that the curve $\gamma$ has a tangent line at the point $z_0$ and that the value of the derivative $f'$ has in $z_0$ a nonzero value
\begin{align}
f'(z_0) \,=\, \varrho e^{i\omega}.
\end{align}
If the slope angles of the secant lines \,$(z_0,\,z_0\!+\!\Delta z)$\, and\, $(w_0,\,w_0\!+\!\Delta w)$\, are $\alpha$ and $\alpha_w$, then we have
$$\Delta z \,=\, ke^{i\alpha},  \quad \Delta w \,=\, k_we^{i\alpha_w},$$
and the difference quotient of $f$ has the form
$$\frac{\Delta w}{\Delta z} \;=\; 
\frac{f(z_0\!+\!\Delta z)-f(z_0)}{\Delta z} \,=\, \frac{k_w}{k}e^{i(\alpha_w-\alpha)}.$$
Let now\, $\Delta z \to 0$.\, Then the point $z_0\!+\!\Delta z$ tends on the curve $\gamma$ to $z_0$ and 
$$\lim_{\Delta z \to 0}\frac{\Delta w}{\Delta z} \;=\; f'(z_0).$$
This implies, by (1), that
\begin{align}
\lim_{\Delta z \to 0}\frac{k_w}{k} \;=\; \varrho.
\end{align}
From this we infer, because\, $\varrho \neq 0$\, that, up to a multiple of $2\pi$,
\begin{align}
\lim_{\Delta z \to 0}(\alpha_w-\alpha) \;=\; \omega.
\end{align}
But the limit of $\alpha$ is the slope angle $\varphi$ of the tangent of $\gamma$ at $z_0$.\, Hence (3) implies that
\begin{align}
\varphi_w \;=\; \lim_{\Delta z \to 0}\alpha_w \;=\; \varphi+\omega.
\end{align}
Accordingly, we have the 

\textbf{Theorem 1.}\, If a curve $\gamma$ has a tangent line in a point $z_0$ where the derivative $f'$ does not vanish, then the image curve $f(\gamma)$ also has in the corresponding point $w_0$ a certain tangent line with a direction obtained by rotating the tangent of $\gamma$ by the angle
$$\omega \;=\; \arg f'(z_0).$$\\

If the curve $\gamma$ is smooth, then also $\gamma_w$ is smooth, and it follows easily from (2) the corresponding limit equation between the arc lengths:
\begin{align}
\lim_{\Delta z \to 0}\frac{s_w}{s} \;=\; |f'(z_0)|.
\end{align}


\textbf{Conformality}

If we have besides $\gamma$ another curve $\gamma'$ emanating from $z_0$ with its tangent, the mapping $f$ from $D$ in $z$-plane to $w$-plane gives two curves and their tangents emanating from $w_0$.\, Thus we have two equations (4):
$$\varphi_w \;=\; \varphi+\omega, \quad \varphi_w' \;=\; \varphi'+\omega$$
By subtracting we obtain
\begin{align}
\varphi_w'-\varphi_w \;=\; \varphi'-\varphi,
\end{align}
whence we have the

\textbf{Theorem 2.}\, The mapping created by the holomorphic function $f$ preserves the magnitude of the angle between two curves in any point $z$ where\, $f'(z) \neq 0$.\, The equation (6) tells also that the orientation of the angle is preserved.\\

The facts in Theorem 2 are expressed so that the mapping is {\em directly conformal}.\, If the orientation were reversed the mapping were called {\em inversely conformal}; in this case $f$ were not holomorphic but {\em antiholomorphic.}

%%%%%
%%%%%
\end{document}
