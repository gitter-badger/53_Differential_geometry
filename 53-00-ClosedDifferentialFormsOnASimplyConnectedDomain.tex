\documentclass[12pt]{article}
\usepackage{pmmeta}
\pmcanonicalname{ClosedDifferentialFormsOnASimplyConnectedDomain}
\pmcreated{2013-03-22 13:32:46}
\pmmodified{2013-03-22 13:32:46}
\pmowner{paolini}{1187}
\pmmodifier{paolini}{1187}
\pmtitle{closed differential forms on a simply connected domain}
\pmrecord{14}{34146}
\pmprivacy{1}
\pmauthor{paolini}{1187}
\pmtype{Theorem}
\pmcomment{trigger rebuild}
\pmclassification{msc}{53-00}
%\pmkeywords{exact differential form}
\pmrelated{ClosedCurveTheorem}
\pmrelated{PoincareLemma}

\endmetadata

% this is the default PlanetMath preamble.  as your knowledge
% of TeX increases, you will probably want to edit this, but
% it should be fine as is for beginners.

% almost certainly you want these
\usepackage{amssymb}
\usepackage{amsmath}
\usepackage{amsfonts}

% used for TeXing text within eps files
%\usepackage{psfrag}
% need this for including graphics (\includegraphics)
%\usepackage{graphicx}
% for neatly defining theorems and propositions
%\usepackage{amsthm}
% making logically defined graphics
%%%\usepackage{xypic}

% there are many more packages, add them here as you need them

% define commands here

\newtheorem{theorem}{Theorem}
\newtheorem{lemma}{Lemma}
\begin{document}
Let $D\subset \mathbb R^2$ be an open set and let $\omega$ be a differential form defined on $D$.

\begin{theorem}
If $D$ is simply connected and $\omega$ is a closed differential form,
then $\omega$ is an exact differential form.
\end{theorem}

The proof of this result is a consequence of the following useful lemmas.

\begin{lemma}
Let $\omega$ be a closed differential form 
and suppose that $\gamma_0$ and $\gamma_1$ are two regular homotopic curves in $D$ (with the same end points). Then 
\[
  \int_{\gamma_0} \omega = \int_{\gamma_1}\omega.
\]
\end{lemma}

\begin{lemma}
Let $\omega$ be a continuous differential form.
If given any two curves $\gamma_0$, $\gamma_1$ in $D$ with the same end-points,
it holds
\[
  \int_{\gamma_0} \omega = \int_{\gamma_1} \omega,
\]
then $\omega$ is exact.
\end{lemma}

See the Poincar\'e Lemma for a generalization of this result on $n$-dimensional manifolds.
%%%%%
%%%%%
\end{document}
