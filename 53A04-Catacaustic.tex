\documentclass[12pt]{article}
\usepackage{pmmeta}
\pmcanonicalname{Catacaustic}
\pmcreated{2013-03-22 18:52:56}
\pmmodified{2013-03-22 18:52:56}
\pmowner{pahio}{2872}
\pmmodifier{pahio}{2872}
\pmtitle{catacaustic}
\pmrecord{10}{41730}
\pmprivacy{1}
\pmauthor{pahio}{2872}
\pmtype{Definition}
\pmcomment{trigger rebuild}
\pmclassification{msc}{53A04}
\pmclassification{msc}{51N20}
\pmclassification{msc}{26B05}
\pmclassification{msc}{26A24}
\pmsynonym{caustic}{Catacaustic}
\pmrelated{HeronsPrinciple}
\pmrelated{ExampleOfFindingCatacaustic}

% this is the default PlanetMath preamble.  as your knowledge
% of TeX increases, you will probably want to edit this, but
% it should be fine as is for beginners.

% almost certainly you want these
\usepackage{amssymb}
\usepackage{amsmath}
\usepackage{amsfonts}

% used for TeXing text within eps files
%\usepackage{psfrag}
% need this for including graphics (\includegraphics)
%\usepackage{graphicx}
% for neatly defining theorems and propositions
 \usepackage{amsthm}
% making logically defined graphics
%%%\usepackage{xypic}

% there are many more packages, add them here as you need them

% define commands here

\theoremstyle{definition}
\newtheorem*{thmplain}{Theorem}

\begin{document}
Given a plane curve $\gamma$, its \emph{catacaustic} (Greek $\varkappa\alpha\tau\acute{\alpha}\, \varkappa\alpha\upsilon\sigma\tau\iota\varkappa \acute{o}\varsigma$ `burning along') is the envelope of a family of rays reflected from $\gamma$ after having emanated from a \PMlinkescapetext{fixed} point (which may be infinitely far, in which case the rays are initially parallel).

For example, the catacaustic of a logarithmic spiral reflecting the rays emanating from the origin is a congruent spiral.\; The catacaustic of the \PMlinkname{exponential curve}{ExponentialFunction} \,$y = e^x$\, reflecting the vertical rays \,$x = t$\, is the catenary \,$y = \cosh(x\!+\!1)$.
%%%%%
%%%%%
\end{document}
