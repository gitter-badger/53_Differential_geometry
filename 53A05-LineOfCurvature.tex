\documentclass[12pt]{article}
\usepackage{pmmeta}
\pmcanonicalname{LineOfCurvature}
\pmcreated{2013-03-22 18:08:44}
\pmmodified{2013-03-22 18:08:44}
\pmowner{pahio}{2872}
\pmmodifier{pahio}{2872}
\pmtitle{line of curvature}
\pmrecord{5}{40701}
\pmprivacy{1}
\pmauthor{pahio}{2872}
\pmtype{Definition}
\pmcomment{trigger rebuild}
\pmclassification{msc}{53A05}
\pmclassification{msc}{26B05}
\pmclassification{msc}{26A24}
\pmsynonym{curvature line}{LineOfCurvature}
\pmrelated{TiltCurve}

% this is the default PlanetMath preamble.  as your knowledge
% of TeX increases, you will probably want to edit this, but
% it should be fine as is for beginners.

% almost certainly you want these
\usepackage{amssymb}
\usepackage{amsmath}
\usepackage{amsfonts}

% used for TeXing text within eps files
%\usepackage{psfrag}
% need this for including graphics (\includegraphics)
%\usepackage{graphicx}
% for neatly defining theorems and propositions
 \usepackage{amsthm}
% making logically defined graphics
%%%\usepackage{xypic}

% there are many more packages, add them here as you need them

% define commands here

\theoremstyle{definition}
\newtheorem*{thmplain}{Theorem}

\begin{document}
A line $\gamma$ on a surface $S$ is a {\em line of curvature} of $S$, if in every point of $\gamma$ one of the principal sections has common tangent with $\gamma$.

By the \PMlinkname{parent entry}{NormalCurvatures}, a surface \,$F(x,\,y,\,z) = 0$,\, where $F$ has continuous first and \PMlinkescapetext{second order} partial derivatives, has two distinct families of lines of curvature, which families are \PMlinkname{orthogonal}{ConvexAngle} to each other.

For example, the meridian curves and the circles of latitude are the two families of the lines of curvature on a surface of revolution.

On a developable surface, the other family of its curvature lines consists of the generatrices of the surface.

A necessary and sufficient condition for that the surface normals of a surface $S$ set along a curve $c$ on $S$ would form a developable surface, is that $c$ is a line of curvature of $S$.


%%%%%
%%%%%
\end{document}
