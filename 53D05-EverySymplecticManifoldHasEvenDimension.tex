\documentclass[12pt]{article}
\usepackage{pmmeta}
\pmcanonicalname{EverySymplecticManifoldHasEvenDimension}
\pmcreated{2013-03-22 15:44:05}
\pmmodified{2013-03-22 15:44:05}
\pmowner{cvalente}{11260}
\pmmodifier{cvalente}{11260}
\pmtitle{every symplectic manifold has even dimension}
\pmrecord{18}{37685}
\pmprivacy{1}
\pmauthor{cvalente}{11260}
\pmtype{Theorem}
\pmcomment{trigger rebuild}
\pmclassification{msc}{53D05}
\pmrelated{AlternatingForm}

\endmetadata

% this is the default PlanetMath preamble.  as your knowledge
% of TeX increases, you will probably want to edit this, but
% it should be fine as is for beginners.

% almost certainly you want these
\usepackage{amssymb}
\usepackage{amsmath}
\usepackage{amsfonts}

% used for TeXing text within eps files
%\usepackage{psfrag}
% need this for including graphics (\includegraphics)
%\usepackage{graphicx}
% for neatly defining theorems and propositions
%\usepackage{amsthm}
% making logically defined graphics
%%%\usepackage{xypic}

% there are many more packages, add them here as you need them

% define commands here
\DeclareMathOperator{\dimens}{dim}
\DeclareMathOperator{\linspan}{span}
\begin{document}
All we need to prove is that every finite dimensional vector space $V$ with an anti-symmetric non-degenerate linear form $\omega$ has an even dimension $2k$. This is only a linear algebra result. In the case of a symplectic manifold $V$ is just the tangent space at a point, and thus its dimension equals the manifold's dimension.

Pick any not null vector $v_0 \in V$. Since $\omega$ is non-degenerate $\omega(v_0,\cdot)$ is a not null linear form. Therefore there exists a not null vector $u_0$ such that $\omega(v_0,u_0)=1$

Now $v_0$ and $u_0$ are linearly independent because if $v_0=\lambda u_0$ then $\omega(v_0, u_0) = \omega(\lambda u_0, u_0) = \lambda \omega(u_0,u_0) = 0$ (by anti-symmetry).

Let $V_0 = \linspan\{v_0, u_0\}$. Consider a space $V_1$ of "orthogonal" elements to $V_0$ under $\omega$. That is:

$V_1 = \left\{ v_1\in V: \text{ for all $v \in V_0$},\: \omega(v,v_1)=0 \right\}$

We now prove $V = V_0 \bigoplus V_1$:

\begin{itemize}

\item $V_0 \bigcap V_1 = \{0\}$

Suppose $w \in V_0 \bigcap V_1$ is not null, then it can be written $w = \alpha v_0 + \beta u_0$ because it belongs to $V_0$. Since it also belongs to $V_1$ is is "orthogonal" to both $v_0$ and $u_0$. That is:

$\omega(v_0,w) = 0 \implies \beta \omega(v_0,u_0) = 0 \implies \beta =0$
similarly

$\omega(u_0,w) = 0 \implies \alpha \omega(u_0,v_0) = 0 \implies \alpha =0$

So $w$ must be null.

\item $V = V_0 \bigoplus V_1$

Suppose $w \in V$. Let $\alpha = \omega(v_0,w)$, $\beta = \omega(u_0,w)$, $w_0 = \alpha u_0 - \beta v_0$.

Then $\omega(v_0,w_0) = \alpha = \omega(v_0,w)$ and $\omega(u_0,w_0)=\beta=\omega(u_0,w)$.

Considering $w_1=w - w_0$ we have $w = w_0 + w_1$ (by construction) and $\omega(v_0,w_1) = \omega(v_0, w-w_0)=\omega(v_0,w) - \omega(v_0,w_0) = \omega(v_0,w) -  \omega(v_0,w) =0 $ and similarly for $\omega(u_0,w_1)$

So $w_1 \in V_1$, $w_0 \in V_0$ and $w=w_0 + w_1$ and thus $V = V_0 \bigoplus V_1$

\end{itemize}

So the matrix representation of $\omega$ is block-diagonal in $V_0 \bigoplus V_1$ and a restriction anti-symmetric bilinear for of $\omega$ to $V_1$ exists.

If $V_1$ is not null we can repeat the procedure with the restriction. Since $\dimens(V)=\dimens(V_0)+\dimens(V_1)$ and $V$ is finite dimensional the procedure must stop at a finite step.

At the end we get a decomposition $V=\bigoplus_{i=0}^{k-1}V_i$, where $\dimens(V_i)=2$ and $\dimens(V)=2k$ is even.
%%%%%
%%%%%
\end{document}
