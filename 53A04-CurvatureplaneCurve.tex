\documentclass[12pt]{article}
\usepackage{pmmeta}
\pmcanonicalname{CurvatureplaneCurve}
\pmcreated{2013-03-22 15:31:19}
\pmmodified{2013-03-22 15:31:19}
\pmowner{rspuzio}{6075}
\pmmodifier{rspuzio}{6075}
\pmtitle{curvature (plane curve)}
\pmrecord{15}{37393}
\pmprivacy{1}
\pmauthor{rspuzio}{6075}
\pmtype{Topic}
\pmcomment{trigger rebuild}
\pmclassification{msc}{53A04}
\pmrelated{CurvatureDeterminesTheCurve}

\endmetadata

% this is the default PlanetMath preamble.  as your knowledge
% of TeX increases, you will probably want to edit this, but
% it should be fine as is for beginners.

% almost certainly you want these
\usepackage{amssymb}
\usepackage{amsmath}
\usepackage{amsfonts}

% used for TeXing text within eps files
%\usepackage{psfrag}
% need this for including graphics (\includegraphics)
%\usepackage{graphicx}
% for neatly defining theorems and propositions
%\usepackage{amsthm}
% making logically defined graphics
%%%\usepackage{xypic}

% there are many more packages, add them here as you need them

% define commands here
\begin{document}
\section{Basic Intuition}

The \emph{curvature} of a plane curve is a quantity which measures the amount by which the curve differs from being a straight line.  It measures the rate at which the direction of a tangent to the curve changes.

\section{Arclength Parameterization}

The simplest way to introduce the curvature is by first parameterizing the curve with respect to arclength.  Suppose that $s$ denotes arclength and that the curve is specified by two functions $f$ and $g$ of this parameter.  In other words, a typical point of the curve is $(f(s), g(s))$, where $s$ must lie in some specified range.  Recall that the condition that $s$ be the arclength is that $(f')^2 + (g')^2 = 1$.

Let $\theta$ denote the angle which the tangent vector makes with the $x$ axis.  Because of the arclength condition mentioned above, we have
 \begin{eqnarray*}
 f' (s) &=& \cos \theta \cr
 g' (s) &=& \sin \theta .
 \end{eqnarray*}
The curvature is simply the derivative of this angle with respect to arclength:
 \[ \kappa = \frac{d \theta}{d s} \]

It is convenient to re-express the curvature in terms of $f$ and $g$.  To do this, we differentiate the previous equation:
 \begin{eqnarray*} f'' &=& \frac{d}{ds} \cos \theta = - \sin \theta \frac{d \theta}{ds} = - \kappa  \sin \theta \cr
 g'' &=& \frac{d}{ds} \sin \theta =  \cos \theta \frac{d \theta}{ds} =  \kappa  \cos \theta \end{eqnarray*}
By eliminating $\theta$, we obtain the following formulae for the curvature:
\begin{eqnarray*} 
 \kappa &=& -f'' / g' \cr
 \kappa &=& g'' / f' 
\end{eqnarray*}
Because of the condition $(f')^2 + (g')^2 = 1$, both $f'$ and $g'$ cannot simultaneously be zero, so at least one of the above formulae must be valid at any point on the curve.

At first, it may seem odd that we have obtained two different formulae for the same quantity.  The reason for this is simple.  Differentiating the arclength condition
 \[ (f')^2 + (g')^2 = 1 \]
gives
 \[ f'' f' + g'' g' = 0 \]
or, dividing out,
 \[ g'' / f' = - f'' / g' .\]
which explains why the two formulae for the curvature must agree.  In fact, one can easily derive several other formulae for the curvature by using this identity.  

\section{Rotation-Invariant Formula}

For instance, one might want to obtain a formula which is explicitly invariant under rotation.  Consider the following determinant:
 \[ \left| \begin{matrix} f' & g' \cr f'' & g'' \end{matrix} \right| = f' g'' - f'' g' \]
On the one hand, this is clearly invariant under rotation.  On the other hand, we have
 \[ f' g'' - f'' g' = \kappa (f')^2 + \kappa (g')^2 = \kappa \left( (f')^2 + (g')^2 \right) = \kappa ,\]
hence we have the explicitly rotation-invariant formula
 \[ \kappa = \left| \begin{matrix} f' & g' \cr f'' & g'' \end{matrix} \right| \]

\section{Arbitrary Parameterization}

Typically, when one is given a curve, it is not specified in terms of a parameterization by arclength.  Since reparameterizing a curve by arclength is not always easy, it is useful to have a formula for curvature which is invariant under reparameterization since one could use such a formula with any parameterization.  Such a formula can be obtained by a slight modification of the rotation-invariant formula given above.

To obtain this formula, first let us inquire into how the determinant transforms under change of parameterization.  If we apply a change of parameter $\sigma = \phi (s)$ then, by the chain rule,
 \[ \frac{df}{ds} = \frac{df}{d \sigma} \frac{d \sigma}{ds} = \phi'(s) \frac{df}{d \sigma} \]
 \[ \frac{dg}{ds} = \phi'(s) \frac{dg}{d \sigma} \]
 \[ \frac{d^2 f}{ds^2} = \frac{d}{ds} \left( \phi'(s) \frac{df}{d \sigma} \right) = \phi''(s) \frac{df}{d\sigma} + \phi'(s) \frac{d\sigma}{ds} \frac{d^2 f}{d \sigma^2} = \phi''(s) \frac{df}{d \sigma} + \left( \phi'(s) \right)^2 \frac{d^2 f}{d \sigma^2}\]
  \[ \frac{d^2 g}{ds^2} = \phi''(s) \frac{dg}{d \sigma} + \left( \phi'(s) \right)^2 \frac{d^2 g}{d\sigma^2}\]
Thus, we have the following transformation for the determinant:
 \[ \left| \begin{matrix} \frac{df}{ds} & \frac{dg}{ds} \cr \frac{d^2 f}{ds^2} & \frac{d^2 g}{ds^2} \end{matrix} \right| = \left( \phi' (s) \right)^3 \left| \begin{matrix} \frac{df}{d \sigma} & \frac{dg}{d \sigma} \cr \frac{d^2 f}{d \sigma^2} & \frac{d^2 g}{d \sigma^2} \end{matrix} \right| \]
Likewise, one has the following transform:
 \[ \left( \frac{df}{ds} \right)^2 + \left( \frac{dg}{ds} \right)^2 =  \left( \phi'(s) \right)^2 \left( \left( \frac{df}{d \sigma} \right)^2 + \left( \frac{dg}{d \sigma} \right)^2 \right) \]
Therefore, the following quantity is invariant under both rotation and reparameterization:
 \[ \frac{\left| \begin{matrix} \frac{df}{d \sigma} & \frac{dg}{d \sigma} \cr \frac{d^2 f}{d \sigma^2} & \frac{d^2 g}{d \sigma^2} \end{matrix} \right|}{\left( \left( \frac{df}{d \sigma} \right)^2 + \left( \frac{dg}{d \sigma} \right)^2 \right)^{3/2}} \]
In the particular case where $\sigma = s$, this equals the curvature; hence, by invariance, it equals the curvature for all choices of parameterization:
 \[ \kappa = \frac{\left| \begin{matrix} \frac{df}{d \sigma} & \frac{dg}{d \sigma} \cr \frac{d^2 f}{d \sigma^2} & \frac{d^2 g}{d \sigma^2} \end{matrix} \right|}{\left( \left( \frac{df}{d \sigma} \right)^2 + \left( \frac{dg}{d \sigma} \right)^2 \right)^{3/2}} \]

\section{Alternative Characterizations}

One special case is especially worth noting.  Suppose that the curve is given as the graph of a function.  That is equivalent to choosing one of $f$ or $g$ to be the identity function.  Then the formula reduces to the following:
 \[ \kappa = \frac{f''}{\left( 1 + (f')^2 \right)^{3/2}} \]
It is worth noting that, at points where $f' = 0$ (i.e. where the tangent to the curve is horizontal) the curvature simply equals the second derivative.  This observation leads to another characterization of the curvature --- the curvature of a curve at a point can be obtained by setting up a coordinate system whose abscissa is the tangent to the curve at that point, expressing the curve as the graph of a function in this coordinate system, then taking the second derivative of this function at said point.  It might also be worth pointing out the curvature of a curve at a point equals the reciprocal of the radius of the osculating circle to the curve at that point.
%%%%%
%%%%%
\end{document}
