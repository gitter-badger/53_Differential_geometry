\documentclass[12pt]{article}
\usepackage{pmmeta}
\pmcanonicalname{Differential}
\pmcreated{2014-02-23 15:34:46}
\pmmodified{2014-02-23 15:34:46}
\pmowner{pahio}{2872}
\pmmodifier{pahio}{2872}
\pmtitle{differential}
\pmrecord{23}{42097}
\pmprivacy{1}
\pmauthor{pahio}{2872}
\pmtype{Definition}
\pmcomment{trigger rebuild}
\pmclassification{msc}{53A04}
\pmclassification{msc}{26A06}
\pmclassification{msc}{26-03}
\pmclassification{msc}{01A45}
\pmsynonym{differential increment}{Differential}
\pmrelated{Derivative2}
\pmrelated{LeibnizNotation}
\pmrelated{ExactDifferentialEquation}
\pmrelated{ProductAndQuotientOfFunctionsSum}
\pmrelated{TotalDifferential}
\pmdefines{differential quotient}

% this is the default PlanetMath preamble.  as your knowledge
% of TeX increases, you will probably want to edit this, but
% it should be fine as is for beginners.

% almost certainly you want these
\usepackage{amssymb}
\usepackage{amsmath}
\usepackage{amsfonts}

% used for TeXing text within eps files
%\usepackage{psfrag}
% need this for including graphics (\includegraphics)
%\usepackage{graphicx}
% for neatly defining theorems and propositions
 \usepackage{amsthm}
% making logically defined graphics
%%%\usepackage{xypic}
\usepackage{pstricks}
\usepackage{pst-plot}

% there are many more packages, add them here as you need them

% define commands here

\theoremstyle{definition}
\newtheorem*{thmplain}{Theorem}

\begin{document}
If a real function $f$ has the derivative at a value $x$ of its argument, 
then the absolute value of the expression
$$\frac{f(x\!+\!\Delta x)\!-\!f(x)}{\Delta x}-f'(x)$$
may be made smaller than any given positive number by making $|\Delta x|$ sufficiently small.\, If we generally denote by $\langle\Delta x\rangle$ an expression having such a property, we can write
$$\frac{f(x\!+\!\Delta x)\!-\!f(x)}{\Delta x}-f'(x) \;=\; \langle\Delta x\rangle.$$
This allows us to express the increment of the function \,$f(x\!+\!\Delta x)\!-\!f(x) := \Delta f$\, in the form
\begin{align}
\Delta f \;=\; (f'(x)\!+\!\langle\Delta x\rangle)\Delta x \;=\;f'(x)\Delta x+\langle\Delta x\rangle\Delta x.
\end{align}

This result may be uttered as the

\textbf{Theorem.}\, If the derivative $f'(x)$ exists, then the increment $\Delta f$ of the function corresponding to the increment of the argument from $x$ to $x\!+\!\Delta x$ may be divided into two essentially different parts:\\
$1^\circ$. One part is proportional to the increment $\Delta x$ of the argument, i.e. it equals this increment multiplied by a coefficient $f'(x)$ which is \PMlinkescapetext{independent} on the increment.\\
$2^\circ$. The ratio of the other part $\langle\Delta x\rangle\Delta x$ to the increment $\Delta x$ of the argument tends to 0 along with $\Delta x$.

As well, the converse of the theorem is true.\\


By Leibniz, the former part $f'(x)\Delta x$ is called the \emph{differential}, or the \emph{differential increment} of the function, and denoted by $df(x)$, briefly $df$.\\

It is easily checked that when one has set the tangent line of the curve at the point \,$(x,\,f(x))$,\,  the differential increment $df(x)$ geometrically means the increment of the ordinate of the \PMlinkescapetext{tangent} corresponding to the transition from the abscissa $x$ to the ascissa $x\!+\!\Delta x$.\\


The differential of the identity function ($f(x) \equiv x$,\; $f'(x) \equiv 1$) is
$$dx \;=\; 1\cdot\Delta x \;=\; \Delta x.$$
Accordingly, one can without discrepancies denote the increment $\Delta x$ of the variable $x$ by $dx$.\, Therefore, the differential of a function $f$ gets the \PMlinkescapetext{consistent} notation
\begin{align}
df(x) \;=\; f'(x)\,dx.
\end{align}
It follows
\begin{align}
f'(x) \;=\; \frac{df(x)}{dx},
\end{align}
in which the \emph{differential quotient} is often replaced using a ``differentiation operator'':
\begin{align}
f'(x) \;=\; \frac{d}{dx}f(x)
\end{align}

\begin{center}
\psset{unit=2cm}
\begin{pspicture}(1,-0.5)(6.5,4)
\psline{->}(2,0)(6.3,0)
\psplot[linecolor=blue]{2.3}{6}{x x mul 5 div x sub 2 add}
\psdot(3,0.8)
\psplot{2.3}{6}{x 0.2 mul 0.2 add}
\psline(5,0)(5,2)
\psline[linestyle=dashed](3,0.8)(5,0.8)
\psline(3,0)(3,0.8)
\rput(6,3.38){$y=f(x)$}
\rput(3,-0.15){$x$}
\rput(5,-0.15){$x\!+\!\Delta x$}
\rput(4,0.65){$\Delta x$}
\rput(3,0.96){$P$}
\rput(5,2.16){$Q$}
\rput(5.13,1){$df$}
\rput(1,-0.5){.}
\rput(6.5,4){.}
\end{pspicture}
\end{center}

\textbf{Remark.}\, One can write certain \PMlinkescapetext{formulas} for forming differentials.\, For example, if $f$ and $g$ are two differentiable functions, one has
\begin{align}
d(f\!+\!g) \;=\; df\!+\!dg, \;\qquad d(fg) \;=\; fdg+g\,df.
\end{align}
Naturally, they seem trivial consequences of the sum rule and the product rule, but they include a deeper contents in the case where $f$ and $g$ depend on more than one variable (see \PMlinkname{total differential}{TotalDifferential}).\\
As for a composite function, e.g. $h = f\!\circ\!u$, the chain rule and (2) yield
$$df(u(x)) \;=\; f'(u(x))u'(x)dx \;=\; f'(u(x))du(x),$$
i.e. simply
$$dh \;=\; f'(u)du.$$


\begin{thebibliography}{9}
\bibitem{J} {\sc Ernst Lindel\"of}: \emph{Johdatus korkeampaan analyysiin}. Fourth edition. Werner S\"oderstr\"om Osakeyhti\"o, Porvoo ja Helsinki (1956).
\bibitem{d} {\sc E. Lindel\"of}: \emph{Einf\"uhrung in die h\"ohere Analysis}. Nach der ersten schwedischen
und zweiten finnischen Auflage auf deutsch herausgegeben von E. Ullrich. Teubner, Leipzig (1934).
\end{thebibliography}
%%%%%
%%%%%
\end{document}
