\documentclass[12pt]{article}
\usepackage{pmmeta}
\pmcanonicalname{Parabola}
\pmcreated{2013-03-22 17:08:16}
\pmmodified{2013-03-22 17:08:16}
\pmowner{mps}{409}
\pmmodifier{mps}{409}
\pmtitle{parabola}
\pmrecord{10}{39444}
\pmprivacy{1}
\pmauthor{mps}{409}
\pmtype{Definition}
\pmcomment{trigger rebuild}
\pmclassification{msc}{53A04}
\pmclassification{msc}{51N20}
\pmrelated{Hyperbola}
\pmrelated{ConicSection}
\pmrelated{ArcLengthOfParabola}
\pmdefines{directrix}
\pmdefines{focus}
\pmdefines{apex}

% this is the default PlanetMath preamble.  as your knowledge
% of TeX increases, you will probably want to edit this, but
% it should be fine as is for beginners.

% almost certainly you want these
\usepackage{amssymb}
\usepackage{amsmath}
\usepackage{amsfonts}

% used for TeXing text within eps files
%\usepackage{psfrag}
% need this for including graphics (\includegraphics)
%\usepackage{graphicx}
% for neatly defining theorems and propositions
%\usepackage{amsthm}
% making logically defined graphics
%%%\usepackage{xypic}

% there are many more packages, add them here as you need them
\usepackage{pstricks}
\usepackage{pst-plot}

% define commands here
\begin{document}
A \emph{parabola} is the locus of points $P$ in the Euclidean plane
which are equidistant from a given line $\ell$, called the
\emph{directrix}, and a given point $F$ not on the directrix, called
the \emph{focus}.

To obtain a simple equation for the parabola, assume that the
directrix is parallel to the $x$-axis, the focus is on the $y$-axis,
and the directrix and focus are the same distance from the origin.  By
reflecting the plane if necessary, this means that there is a positive
number $a$ such that the equation of the directrix $\ell$ is $y = -a$
and the position of the focus $F$ is $(0, a)$.  Then the condition
that a point $(x,y)$ is equidistant from $\ell$ and $F$ can be
interpreted as the equation
\[
  (y + a)^2 = x^2 + (y - a)^2.
\]
Since $(y + a)^2 - (y - a)^2 = 4ay$, the above equation simplifies to
\[
  y = \frac{1}{4a}x^2.
\]

% TODO: insert picture here

Below is the graph of a parabola for $a=1$:

\begin{center}
\begin{pspicture}(-4.5,-2.5)(4.5,4.5)
\psaxes{->}(0,0)(-4.5,-1.5)(4.5,4.5)
\rput[b](4.5,-0.5){$x$}
\rput[l](-0.4,4.5){$y$}
\parabola{<->}(4,4)(0,0)
\rput[l](-4.5,0){.}
\rput[b](0,-2.2){Graph of $\displaystyle y=\frac{1}{4}x^2$}
\end{pspicture}
\end{center}

From the equation
\[
  y = \frac{1}{4a}x^2,
\]
we can immediately observe some important
properties of the parabola.  First, since $x^2$ is an even function,
the parabola is symmetric with respect to the $y$-axis; this can also
be deduced directly from the geometric definition of the parabola.\, The intersection point of the parabola and the symmetry axis, is called the \emph{apex} of the parabola; it is the point of the parabola nearest the directrix.\,
Second, notice that the coefficient of $x^2$ in the equation of the
parabola is inversely proportional to $2a$, the distance between the
focus and the directrix.  So this distance controls how rapidly the
function $\frac{1}{4a}x^2$ grows.  As $a$ tends to zero, the parabola
becomes flatter and flatter, tending to the \PMlinkescapetext{straight} line $y = 0$ in
the degenerate case $a = 0$.  On the other hand, as $a$ increases, the
\PMlinkname{curvature}{CurvaturePlaneCurve} of the parabola at $0$ increases.  When $a$ tends to
infinity, the parabola begins to resemble a hairpin more and more
until it suddenly becomes a single point, the origin, in the degenerate
case $a = \infty$.

% TODO: show different a values and the parabolas they produce

The parabola is a conic section with eccentricity 1.\, All parabolas are similar, which follows directly from the definition of parabola.

% TODO: insert spooky ``shadow'' talk here

% TODO: answer ``okay, but how do we get the equation for a more general parabola?''

% TODO: mention paraboloid as generalization?

\PMlinkescapeword{necessary}
\PMlinkescapeword{simple}
\PMlinkescapeword{symmetric} % TODO: find a good link
%%%%%
%%%%%
\end{document}
