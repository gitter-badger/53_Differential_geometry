\documentclass[12pt]{article}
\usepackage{pmmeta}
\pmcanonicalname{GermOfSmoothFunctions}
\pmcreated{2013-03-22 13:05:08}
\pmmodified{2013-03-22 13:05:08}
\pmowner{rspuzio}{6075}
\pmmodifier{rspuzio}{6075}
\pmtitle{germ of smooth functions}
\pmrecord{4}{33501}
\pmprivacy{1}
\pmauthor{rspuzio}{6075}
\pmtype{Definition}
\pmcomment{trigger rebuild}
\pmclassification{msc}{53B99}
%\pmkeywords{vector fields}
%\pmkeywords{local functions}

% this is the default PlanetMath preamble.  as your knowledge
% of TeX increases, you will probably want to edit this, but
% it should be fine as is for beginners.

% almost certainly you want these
\usepackage{amssymb}
\usepackage{amsmath}
\usepackage{amsfonts}

% used for TeXing text within eps files
%\usepackage{psfrag}
% need this for including graphics (\includegraphics)
%\usepackage{graphicx}
% for neatly defining theorems and propositions
%\usepackage{amsthm}
% making logically defined graphics
%%%\usepackage{xypic}

% there are many more packages, add them here as you need them

% define commands here
\begin{document}
If $x$ is a point on a smooth manifold $M$, then a {\em germ of smooth functions near $x$} is represented by a pair $(U,f)$ where $U \subseteq M$ is an open neighbourhood of $x$, and $f$ is a smooth function $U \rightarrow \mathbb{R}$. Two such pairs $(U,f)$ and $(V,g)$ are considered equivalent if there is a third open neighbourhood $W$ of $x$, contained in both $U$ and $V$, such that $f|_W=g|_W$. To be precise, a germ of smooth functions near $x$ is an equivalence class of such pairs. 

In more fancy language: the set $\mathcal{O}_x$ of germs at $x$ is the stalk at $x$ of the sheaf $\mathcal{O}$ of smooth functions on $M$. It is clearly an $\mathbb{R}$-algebra.

Germs are useful for defining the tangent space $T_x M$ in a coordinate-free manner: it is simply the space of all $\mathbb{R}$-linear maps $X:\mathcal{O}_x \rightarrow \mathbb{R}$ satisfying Leibniz' rule $X(fg)=X(f)g+fX(g)$. (Such a map is called an $\mathbb{R}$-linear derivation of $\mathcal{O}_x$ with values in $\mathbb{R}$.)
%%%%%
%%%%%
\end{document}
