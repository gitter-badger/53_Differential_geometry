\documentclass[12pt]{article}
\usepackage{pmmeta}
\pmcanonicalname{HamiltonEquations}
\pmcreated{2013-03-22 14:45:58}
\pmmodified{2013-03-22 14:45:58}
\pmowner{CWoo}{3771}
\pmmodifier{CWoo}{3771}
\pmtitle{Hamilton equations}
\pmrecord{9}{36410}
\pmprivacy{1}
\pmauthor{CWoo}{3771}
\pmtype{Definition}
\pmcomment{trigger rebuild}
\pmclassification{msc}{53D05}
\pmclassification{msc}{70H05}
\pmrelated{Quantization}

\endmetadata

% this is the default PlanetMath preamble.  as your knowledge
% of TeX increases, you will probably want to edit this, but
% it should be fine as is for beginners.

% almost certainly you want these
\usepackage{amssymb}
\usepackage{amsmath}
\usepackage{amsfonts}
\usepackage{amsthm}

\usepackage{mathrsfs}

% used for TeXing text within eps files
%\usepackage{psfrag}
% need this for including graphics (\includegraphics)
%\usepackage{graphicx}
% for neatly defining theorems and propositions
%
% making logically defined graphics
%%%\usepackage{xypic}

% there are many more packages, add them here as you need them

% define commands here

\newcommand{\sR}[0]{\mathbb{R}}
\newcommand{\sC}[0]{\mathbb{C}}
\newcommand{\sN}[0]{\mathbb{N}}
\newcommand{\sZ}[0]{\mathbb{Z}}

 \usepackage{bbm}
 \newcommand{\Z}{\mathbbm{Z}}
 \newcommand{\C}{\mathbbm{C}}
 \newcommand{\R}{\mathbbm{R}}
 \newcommand{\Q}{\mathbbm{Q}}



\newcommand*{\norm}[1]{\lVert #1 \rVert}
\newcommand*{\abs}[1]{| #1 |}



\newtheorem{thm}{Theorem}
\newtheorem{defn}{Definition}
\newtheorem{prop}{Proposition}
\newtheorem{lemma}{Lemma}
\newtheorem{cor}{Corollary}
\begin{document}
The Hamilton equations are a formulation of the equations of motion in classical mechanics.

\subsubsection*{Local formulation}
Suppose $U\subseteq \R^n$ is an open set, suppose $I$ is an interval
(representing time), and $H\colon U\times \R^n\times I\to \R$
is a smooth function. Then the equations
\begin{align}
\label{HE1}
\dot{q}_j &= \frac{\partial H}{\partial p_j}(q(t),p(t),t), \\
\label{HE2}
\dot{p}_j &= -\frac{\partial H}{\partial q_j}(q(t),p(t),t), 
\end{align}
are the \emph{Hamilton equations} for the curve
$$
  (q, p)=(q_1,\ldots, q_n, p_1,\ldots, p_n) \colon I\to U\times \R^n.
$$
Such a solution is called a \emph{bicharacteristic}, and $H$ is
called a \emph{Hamiltonian function}. Here we use classical notation;
the $q_i$'s represent the location of the particles, 
the $p_i$'s represent the momenta of the  particles.


\subsubsection*{Global formulation}
Suppose $P$ is a symplectic manifold with symplectic form $\omega$ and that $H\colon P\to \R$
is a smooth function.  Then $X_H$, the Hamiltonian 
vector field corresponding to $H$ is determined by 
$$
  dH=\omega(X_H,\cdot).
$$
The most common case is when $P$ is the cotangent bundle of a manifold $Q$
equipped with the canonical symplectic form $\omega=-d\alpha$, 
where $\alpha$ is the \PMlinkname{Poincar\'e $1$-form}{Poincare1Form}.  (Note that other authors may have different sign convention.)  Then Hamilton's equations are the equations for the flow of the vector field $X_H$.  Given a system of coordinates $x^1, \ldots x^{2n}$ on the manifold $P$, they can be written as follows:
$$
 \dot x^i = (X_H)^i (x_1, \ldots x_{2n}, t)
$$
The relation with the former definition is that in canonical 
local coordinates $(q_i,p_j)$ for $T^\ast Q$, the flow of $X_H$ 
is determined by equations \eqref{HE1}-\eqref{HE2}.

Also, the following terminology is frequently encountered --- the manifold $P$ is known as the phase space, the manifold $Q$ is known as the configuration space, and the product $P \times\R$ is known as state space.
%%%%%
%%%%%
\end{document}
