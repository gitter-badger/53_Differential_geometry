\documentclass[12pt]{article}
\usepackage{pmmeta}
\pmcanonicalname{PerimeterOfEllipse}
\pmcreated{2013-03-22 16:58:10}
\pmmodified{2013-03-22 16:58:10}
\pmowner{pahio}{2872}
\pmmodifier{pahio}{2872}
\pmtitle{perimeter of ellipse}
\pmrecord{18}{39244}
\pmprivacy{1}
\pmauthor{pahio}{2872}
\pmtype{Example}
\pmcomment{trigger rebuild}
\pmclassification{msc}{53A04}
\pmclassification{msc}{51N20}
\pmclassification{msc}{51-00}
\pmsynonym{perimetre of ellipse}{PerimeterOfEllipse}
%\pmkeywords{eccentric anomaly}
\pmrelated{ArcLength}
\pmrelated{EllipticIntegral}
\pmrelated{PropertiesOfEllipse}

\endmetadata

% this is the default PlanetMath preamble.  as your knowledge
% of TeX increases, you will probably want to edit this, but
% it should be fine as is for beginners.

% almost certainly you want these
\usepackage{amssymb}
\usepackage{amsmath}
\usepackage{amsfonts}
\usepackage{amsthm}

\usepackage{mathrsfs}
\usepackage{pstricks}
\usepackage{pst-plot}

% used for TeXing text within eps files
%\usepackage{psfrag}
% need this for including graphics (\includegraphics)
%\usepackage{graphicx}
% for neatly defining theorems and propositions
%
% making logically defined graphics
%%%\usepackage{xypic}

% there are many more packages, add them here as you need them

% define commands here

\newcommand{\sR}[0]{\mathbb{R}}
\newcommand{\sC}[0]{\mathbb{C}}
\newcommand{\sN}[0]{\mathbb{N}}
\newcommand{\sZ}[0]{\mathbb{Z}}

 \usepackage{bbm}
 \newcommand{\Z}{\mathbbmss{Z}}
 \newcommand{\C}{\mathbbmss{C}}
 \newcommand{\F}{\mathbbmss{F}}
 \newcommand{\R}{\mathbbmss{R}}
\newcommand{\Q}{\mathbbmss{Q}}
\newcommand*{\norm}[1]{\lVert #1 \rVert}
\newcommand*{\abs}[1]{| #1 |}
\newtheorem{thm}{Theorem}
\newtheorem{defn}{Definition}
\newtheorem{prop}{Proposition}
\newtheorem{lemma}{Lemma}
\newtheorem{cor}{Corollary}
\begin{document}
The usual parametric \PMlinkescapetext{presentation} of the ellipse (blue in the diagram)
$$\frac{x^2}{a^2}+\frac{y^2}{b^2} \;=\; 1,$$
using as parameter the {\em eccentric anomaly} $t$, is
\begin{align}
x \;=\; a\cos{t}, \quad y \;=\; b\sin{t} \quad (0\leqq t < 2\pi).
\end{align}
\begin{center}
\begin{pspicture}(-3.2,-2.5)(3.5,3.5)
\psaxes[Dx=9,Dy=9]{->}(0,0)(-3.5,-3.2)(3.5,3.5)
\psellipse[linecolor=blue](0,0)(3,2)
\pscircle[linecolor=red](0,0){3}
\rput[b](3.5,-0.3){$x$}
\rput[r](-0.2,3.5){$y$}
\rput(-0.2,2.77){$a$}
\rput(-0.2,1.75){$b$}
\rput(2.8,-0.2){$a$}
\psline[linecolor=green](2,0)(2,2.236)
\psline[linecolor=green](0,0)(2,2.236)
\psarc[linecolor=green](0,0){0.5}{0}{48.19}
\psdot(2,2.20)
\psdot(2,1.49)
\rput(1.8,1.3){$P$}
\rput(2.3,2.3){$Q$}
\rput[b](0.6,0.27){$t$}
\end{pspicture}
\end{center}
If, according to the diagram, the point $Q$ on the circumscribed circle corresponds the point $P$ of the ellipse, the geometrical meaning of $t$ is the polar angle of $Q$.

From (1) we get
$$\frac{dx}{dt} \;=\; -a\sin{t},\;\;\; \frac{dy}{dt} = b\cos{t}.$$

Substituting to the arc length formula
$$s \;=\; \int_{t_1}^{t_2}\sqrt{\left(\frac{dx}{dt}\right)^2\!
+\left(\frac{dy}{dt}\right)^2}\,dt$$
the \PMlinkname{limits}{DefiniteIntegral} $0$ and $2\pi$ and the derivatives
we can express the perimeter length of the ellipse as
$$p \;=\; \int_0^{2\pi}\!\sqrt{a^2\sin^2{t}+b^2\cos^2{t}}\;dt.$$
The expression may be converted by taking the factor $a$ out and by using the formula\, $\sin^2{t} = 1-\cos^2{t}$:
$$p \;=\; a\int_0^{2\pi}\!\sqrt{1-\varepsilon^2\cos^2{t}}\;dt$$
Here we have used the eccentricity \,$\varepsilon = \frac{\sqrt{a^2-b^2}}{a}$\, 
($0 \leqq \varepsilon < 1$).\, Because of the symmetry of the ellipse in regard to the coordinate axes, we get
$$p \;=\; 4a\int_0^{\frac{\pi}{2}}\!\sqrt{1-\varepsilon^2\cos^2{t}}\;dt,$$
and since on the interval \,$[0,\,\frac{\pi}{2}]$\, the sine function attains the same values as the cosine function (but in reverse \PMlinkescapetext{order}), the length of the perimeter of ellipse may be written also as
\begin{align}
p \;=\; 4a\int_0^{\frac{\pi}{2}}\!\sqrt{1-\varepsilon^2\sin^2{t}}\;dt.
\end{align}
This is a complete elliptic integral of second kind.\, The name ``elliptic integral'' is due to this task to determine the length of the perimeter of an ellipse ($0 < \varepsilon < 1$).

The integral (2) is not expressible in closed form with elementary functions, but can be expanded e.g. to the series 
$$p \;=\; 2\pi a\left[1-\left(\frac{1}{2}\right)^2\frac{\varepsilon^2}{1}
-\left(\frac{1\cdot3}{2\cdot4}\right)^2\frac{\varepsilon^4}{3}
-\left(\frac{1\cdot3\cdot5}{2\cdot4\cdot6}\right)^2\frac{\varepsilon^6}{5}-\ldots\right]\!.$$
One quite accurate approximative formula, due to S. Ramanujan, is 
$$p \;\approx\; \pi(a\!+\!b)\left(1+\frac{3\lambda^2}{10+\sqrt{4-3\lambda^2}}\right)$$
with\, $\displaystyle\lambda = \frac{a\!-\!b}{a\!+\!b}$\, and the relative error about $\displaystyle\frac{3\varepsilon^{20}}{2^{36}}$.
%%%%%
%%%%%
\end{document}
