\documentclass[12pt]{article}
\usepackage{pmmeta}
\pmcanonicalname{ArcLengthOfParabola}
\pmcreated{2013-03-22 18:57:19}
\pmmodified{2013-03-22 18:57:19}
\pmowner{pahio}{2872}
\pmmodifier{pahio}{2872}
\pmtitle{arc length of parabola}
\pmrecord{13}{41812}
\pmprivacy{1}
\pmauthor{pahio}{2872}
\pmtype{Example}
\pmcomment{trigger rebuild}
\pmclassification{msc}{53A04}
\pmclassification{msc}{26A42}
\pmclassification{msc}{26A09}
\pmclassification{msc}{26A06}
\pmsynonym{closed-form arc lengths}{ArcLengthOfParabola}
\pmrelated{FamousCurvesInThePlane}
\pmrelated{AreaFunctions}
\pmdefines{universal parabolic constant}

\endmetadata

% this is the default PlanetMath preamble.  as your knowledge
% of TeX increases, you will probably want to edit this, but
% it should be fine as is for beginners.

% almost certainly you want these
\usepackage{amssymb}
\usepackage{amsmath}
\usepackage{amsfonts}

% used for TeXing text within eps files
%\usepackage{psfrag}
% need this for including graphics (\includegraphics)
%\usepackage{graphicx}
% for neatly defining theorems and propositions
 \usepackage{amsthm}
% making logically defined graphics
%%%\usepackage{xypic}

% there are many more packages, add them here as you need them

% define commands here
\DeclareMathOperator{\arsinh}{arsinh}

\theoremstyle{definition}
\newtheorem*{thmplain}{Theorem}

\begin{document}
The parabola is one of the quite few plane curves, the arc length of which is expressible in closed form; other ones are line, \PMlinkname{circle}{Circle}, semicubical parabola, \PMlinkname{logarithmic curve}{NaturalLogarithm2}, catenary, tractrix, cycloid, clothoid, astroid, Nielsen's spiral,  logarithmic spiral.\, Determining the \PMlinkname{arc length of ellipse}{PerimeterOfEllipse} and hyperbola leads to elliptic integrals.\\

We evaluate the \PMlinkescapetext{length} of the parabola
\begin{align}
y \;=\; ax^2 \qquad (a > 0)
\end{align}
from the apex (the origin) to the point \,$(x,\,ax^2)$.\\

The usual arc length \PMlinkescapetext{formula gives}
$$s \;=\; \int_0^x\!\sqrt{1\!+\!y'^2}\,dx \;=\; \int_0^x\!\sqrt{1\!+4a^2x^2}\,dx 
\;=\; \frac{1}{2a}\int_0^{2ax}\!\sqrt{t^2\!+\!1}\,dt.$$
where one has made the \PMlinkname{substitution}{ChangeOfVariableInDefiniteIntegral} $2ax =: t$.\, Then one can utilise the result in the entry \PMlinkname{integration of $\sqrt{x^2\!+\!1}$}{IntegrationOfSqrtx21}, whence
\begin{align}
s \;=\; \frac{1}{4a}\left(2ax\sqrt{4a^2x^2\!+\!1}+\arsinh{2ax}\right).
\end{align}

This expression for the parabola arc length becomes especially \PMlinkescapetext{simple} when the arc is extended from the apex to the end point \,$(\frac{1}{2a},\,\frac{1}{4a})$\, of the parametre, i.e. the latus rectum; this arc length is
$$\frac{1}{4a}(\sqrt{2}+\arsinh{1}) \;=\; \frac{1}{4a}\left(\sqrt{2}+\ln(1\!+\!\sqrt{2})\right).$$
Here,\, $\sqrt{2}+\ln(1\!+\!\sqrt{2}) =: P$\, is called the {\it universal parabolic constant}, since it is common to all parabolas; it is the ratio of the arc to the semiparametre.\, This constant appears also for example in the areas of some surfaces of revolution (see \PMlinkexternal{Reese and Sondow}{http://mathworld.wolfram.com/UniversalParabolicConstant.html}).

%%%%%
%%%%%
\end{document}
