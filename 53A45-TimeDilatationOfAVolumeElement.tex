\documentclass[12pt]{article}
\usepackage{pmmeta}
\pmcanonicalname{TimeDilatationOfAVolumeElement}
\pmcreated{2013-03-22 15:54:28}
\pmmodified{2013-03-22 15:54:28}
\pmowner{perucho}{2192}
\pmmodifier{perucho}{2192}
\pmtitle{time dilatation of a volume element}
\pmrecord{10}{37911}
\pmprivacy{1}
\pmauthor{perucho}{2192}
\pmtype{Definition}
\pmcomment{trigger rebuild}
\pmclassification{msc}{53A45}

% this is the default PlanetMath preamble.  as your knowledge
% of TeX increases, you will probably want to edit this, but
% it should be fine as is for beginners.

% almost certainly you want these
\usepackage{amssymb}
\usepackage{amsmath}
\usepackage{amsfonts}

% used for TeXing text within eps files
%\usepackage{psfrag}
% need this for including graphics (\includegraphics)
%\usepackage{graphicx}
% for neatly defining theorems and propositions
%\usepackage{amsthm}
% making logically defined graphics
%%%\usepackage{xypic}

% there are many more packages, add them here as you need them

% define commands here

\begin{document}
\paragraph{Introduction}
The rate of change of a volume element constitutes an important subject in certain applications related to the velocity field in fluid flow and elasticity, 
besides it admits some physical interpretations intimately entailed to  
tensor invariants, as we shall see.
\paragraph{Formulae derivation} 
Let $X^\alpha$ and $x^i$ be material and spatial coordinates, respectively. Consider  {\em diffeomorphic} the mapping \footnote{See motion of continuum}
\begin{align*}
X^\alpha \mapsto x^i(X^\alpha,\tau), \qquad  t_0\leq\tau\leq{t},
\end{align*} 
which represents the motion of continuum $\Re\subset(\mathbb{R}^3,\lVert\cdot\rVert)$ . The Jacobian $J=\vert x^i_{\;,\alpha}\vert$ (comma denoting partial differentiation with respect to the indicated coordinate) of coordinate transformation is given by
\begin{align*}
J=\epsilon^{\alpha\beta\gamma}\;x^1_{\;,\alpha}\,x^2_{\;,\beta}\,x^3_{\;,\gamma}
\equiv C^\gamma_3 \;x^3_{\;,\gamma}\;,
\end{align*}
$\epsilon^{\alpha\beta\gamma}$ being the Levi-Civita density, $C^\gamma_3$ the cofactor of $x^3_{\;,\gamma}$ in the determinant expansion and it comes expressed as
\begin{align*}
C^\gamma_3=\epsilon^{\alpha\beta\gamma}\;x^1_{\;,\alpha}\,x^2_{\;,\beta}=
\frac{1}{2} (\epsilon^{\alpha\beta\gamma}\;x^1_{\;,\alpha}\,x^2_{\;,\beta}+
\epsilon^{\beta\alpha
\gamma}\;x^1_{\;,\beta}\,x^2_{\;,\alpha})
\end{align*}
\begin{align*}
=\frac{1}{2}\epsilon^{\alpha\beta\gamma} 
\;(x^1_{\;,\alpha}\,x^2_{\;,\beta}-x^1_{\;,\beta}\,x^2_{\;,\alpha})=\frac{1}{2}
\epsilon^{\alpha\beta\gamma}\epsilon_{ij3}\;x^i_{\;,\alpha}\;x^j_{\;,\beta}.
\end{align*}
So that, for any arbitrary cofactor $C^\gamma_k,$
\begin{align*}
C^\gamma_k=\frac{1}{2}
\epsilon^{\alpha\beta\gamma}\epsilon_{ijk}\;x^i_{\;,\alpha}\;x^j_{\;,\beta}.
\end{align*}
Let us multiply by $x^n_{\;,\gamma}.$
\begin{align}
C^\gamma_k \;x^n_{\;,\gamma}=\frac{1}{2}
\epsilon_{ijk}\;\epsilon^{\alpha\beta\gamma}\;x^i_{\;,\alpha}\;x^j_{\;,\beta}
\;x^n_{\;,\gamma} =\frac{1}{2}\vert{x}^i_{\;,\alpha}\vert\epsilon_{ijk}\;\epsilon^{ijn}=
J\delta^n_k, \qquad \frac{C^\gamma_k}{J}\;x^n_{\;,\gamma}=\delta^n_k,
\end{align}
where we have used well-known alternator's properties. Moreover, since the cofactor $C^\gamma_k$ is independent on $x^k_{\;,\gamma}$ by its own definition,
\begin{align}
\frac{\partial}{\partial{x^n_{\;,\gamma}}}(J\delta^n_k)=
\frac{\partial{J}}{\partial{x^k_{\;,\gamma}}}=C^\gamma_k.
\end{align}
But
\begin{align*}
X^\gamma_{\;,k}\;x^n_{\;,\gamma}=\delta^n_k,
\end{align*}
which it is compared with Eq.(1) to obtain
\begin{align*}
X^\gamma_{\;,k}=\frac{C^\gamma_k}{J}, \qquad C^\gamma_k=JX^\gamma_{\;,k}.
\end{align*}
So from Eq.(2) we get
\begin{align}
\frac{\partial{J}}{\partial{x^k_{\;,\gamma}}}=JX^\gamma_{\;,k}.
\end{align}
Let us now consider the relation $dv=JdV$ between the spatial and material volume elements, and by taking the material time derivative
\begin{align}
\dot{\overline{dv}}=\dot{J}dV,
\end{align}
because $\dot{\overline{dV}}=0,$ by definition. From Eqs.(3)-(4),
\begin{align*}
\dot{\overline{\Big(\frac{dv}{dV}\Big)}}=
\frac{\partial{J}}{\partial{x^i_{\;,\alpha}}}
\;\dot{\overline{x^i_{\;,\alpha}}}=\frac{\partial{J}}{\partial{x^i_{\;,\alpha}}}
\;\dot{x}^i_{\;,\alpha}=JX^\alpha_{\;,i}\;v^i_{\;,\alpha}=Jv^i_{\;,i}
\end{align*}
where $\dot{x}^i_{\;,\alpha}\equiv{v^i_{\;,\alpha}}$ are material gradient components of velocity field, thus arriving to the result due to Euler\cite{cite:Euler}
\begin{align}
\frac{\dot{J}}{J}=v^i_{\;,i}\equiv\nabla_x\cdot\mathbf{v}, \qquad \dot{\overline{\log{J}}}=\nabla_x\cdot\mathbf{v},
\end{align}
expressing the spatial divergence of velocity field.
Also, by substituting $dV=dv/J$ in Eq.(4) we get
\begin{align}
\dot{\overline{\log{dv}}}=\nabla_x\cdot\mathbf{v}.
\end{align}
\paragraph{Physical interpretations}
\begin{itemize}
\item {The time logarithm of dilatation and the first  invariant $I_{\nabla_x\mathbf{v}}$ associate to the tensor of velocity spatial gradient, coincide exactly.} 
\item {If we consider the {\em Lagrangian strain} tensor $E_{ij}=1/2(u_{i,j}+u_{j,i}+u_{i,k}u_{k,j})$ (large strain) for $``small"$ strain, i.e. the initial undistorsioned (material) reference configuration $\chi_\varkappa(X_i,t_0)$ maps to  near distorsioned (spatial) reference configuration $\chi_{\varkappa+\Delta\varkappa}(X_i,\tau)$ (as $\tau\to t_0$) during the motion of continuum $\Re,$ coinciding approximately the spatial coordinates with the material coordinates and therefore, as a consequence, the quadratic displacement gradient $u_{i,k}u_{k,j}\approx 0.$ {\footnote{Indeed for small strain is required that $\nabla_X\mathbf{u}\cdot\mathbf{u}\nabla_X\approx\mathbf{0}$ and $\nabla_X\mathbf{u}\cdot\nabla_x\mathbf{u}\approx\mathbf{0}.$ To see this, we consider the coordinates transformation $\mathbf{x}(\tau)=\chi(\mathbf{X},\tau),$ as $\tau\to {t_0}.$ So $d\mathbf{x}=d\mathbf{X}\cdot\nabla_X\mathbf{x}.$ But $d\mathbf{u}=d\mathbf{X}\cdot\nabla_X\mathbf{u}=
d\mathbf{x}\cdot\nabla_x\mathbf{u},$  then by the first equation, we have $d\mathbf{u}=(d\mathbf{X}\cdot\nabla_X\mathbf{x})\cdot\nabla_x\mathbf{u}=
d\mathbf{X}\cdot (\nabla_X\mathbf{x}\cdot\nabla_x\mathbf{u}),$ and since $d\mathbf{X}$ is arbitrary $\nabla_X\mathbf{u}=\nabla_X\mathbf{x}\cdot\nabla_x\mathbf{u}.$ (The chain rule!) Recalling now $\mathbf{x}=\mathbf{X}+\mathbf{u},$ we get $\nabla_X\mathbf{u}=[\nabla_X(\mathbf{X}+\mathbf{u})]\cdot\nabla_x\mathbf{u}=
(\mathbf{1}+\nabla_X\mathbf{u})\cdot\nabla_x\mathbf{u}=
\nabla_x\mathbf{u}+\nabla_X\mathbf{u}\cdot\nabla_x\mathbf{u},$ which shows that quadratic gradient is approximately equal to {\bf zero} whenever $\nabla_X\mathbf{u}\approx\nabla_x\mathbf{u},$ i.e. the material undistorsioned reference configuration $\varkappa$ be approximately equal to the spatial distorsioned configuration $\varkappa+\Delta\varkappa.$}} In fact elasticity theory defines {\em infinitesimal strain} tensor as $e_{ij}\equiv\ 1/2(u_{i,j}+u_{j,i}),$ i.e. $e_{ij}\approx E_{ij}$ for small strain, but the definition of tensor $\mathbf{e}$ is exact. So, in the vector displacement $u_i=x_i-X_i,$ we take the material rate $\dot{u}_i=\dot{x}_i\equiv{v}_i$ and hence $\dot{u}_{i,j}=v_{i,j}.$ Therefore, according to the mentioned approximation, the material time derivative for tensors $\mathbf{E}$ and $\mathbf{e}$ are given by $\dot{E}_{ij}\approx \dot{e}_{ij}\approx 1/2(v_{i,j}+v_{j,i}).$ {\footnote{The last approximation because the tensors $\mathbf{E}$ and $\mathbf{e}$  are usually defined with respect to material coordinates and not with respect to the spatial ones.}} Now by contracting $j=i$, we get $\dot{E}_{ii}\approx\dot{e}_{i,i}\approx v_{i,i}=\nabla\cdot\mathbf{v}.$ {\footnote{Notice that although this is an approximated result, Eqs.(5)-(6) are exact.}}} 
\item{Considering the infinitesimal strain tensor $\mathbf{e}$ (or $\mathbf{E}$ for small strain), we see that sum of {\em normal strain} $e_{ii}=e_{11}+e_{22}+e_{33}=u_{1,1}+u_{2,2}+u_{3,3}$ represents the {\em trace} or first invariant $I_{\mathbf{e}}$. Thus, in the initial undistorsioned reference configuration $\varkappa,$ we can use principal centered axes (i.e. along the eigenvectors of tensor $\mathbf{e},$ whose representation corresponds to pure normal strains) of a volume element $dV$ in order to measure the induced {\em dilatation} $(dv-dV)/dV$ in the distorsioned reference configuration $\varkappa + \Delta\varkappa$. So, for an elemental rectangular parallelopiped of volume $dV$, we have
\begin{align*}
\Big(\frac{dv}{dV}\Big)\approx (1+e_{11})(1+e_{22})(1+e_{33})=1+e_{ii}+ o(e^2_{nn}). \quad   (n\; not\; summed)
\end{align*}
By taking now the material time derivative,
\begin{align*}
\dot{\overline{\Big(\frac{dv}{dV}\Big)}}\approx \dot{e}_{ii}\approx\dot{E}_{ii}\approx\dot{u}_{i,i}=\dot{x}_{i,i}=v_{i,i}=
\nabla\cdot\mathbf{v},
\end{align*}
thus completing the aimed physical interpretation.}
\item{A volume-preserving motion is said to be {\em isochoric.} Then
\begin{align*}
J=\bigg\vert\frac{\partial x^i}{\partial X^\alpha}\bigg\vert=1, \qquad \operatorname{div}_x\mathbf{v}=0.
\end{align*}
$V$ is called {\em material volume} and $v$ is called {\em control volume.}}
\item{Although we have used Cartesian rectangular systems, if we introduce generalizated tensors, all the results obtained are also valid for curvilinear coordinates . For instance,
\begin{enumerate}
\item Divergence
\begin{align*}
\nabla_x\cdot\mathbf{v}=\mathbf{g}^i\cdot\frac{\partial}{\partial{x^i}}
(v^j\mathbf{g}_j)=\mathbf{g}^i\cdot\mathbf{g}_j v^j\vert_i =v^i\vert_i=v^i_{\;,i}+\Gamma^j_{ji}v^i, 
\end{align*}
\begin{align*} 
\Gamma^j_{ji}=\frac{1}{2g}\frac{\partial{g}}{\partial{x}^i}, \quad
g=\vert{g_{ij}}\vert,
\quad g_{jk}g^{ki}=\delta^i_j,
\end{align*}
where $v^j\vert_i,\,\mathbf{g}^i,\,\mathbf{g}_j,$ stand for covariant 
derivative and contravariant and covariant spatial base vectors, respectively.
\item Gradient
\begin{align*}
\nabla_X\mathbf{u}=\mathbf{G}^\alpha\frac{\partial}{\partial{X}^\alpha}
(u^\beta\mathbf{G}_\beta)=u^\beta\vert_\alpha\mathbf{G}^\alpha\mathbf{G}_\beta=
(u^\beta_{\;,\alpha}+
\Gamma^\beta_{\gamma\alpha}u^\gamma)\mathbf{G}^\alpha\mathbf{G}_\beta,
\end{align*}
\begin{align*}
\Gamma^\beta_{\gamma\alpha}=G^{\beta\delta}\Gamma_{\gamma\alpha\delta},
\quad \Gamma_{\gamma\alpha\delta}=\frac{1}{2}(G_{\gamma\delta,\alpha}+
G_{\alpha\delta,\gamma}-G_{\gamma\alpha,\delta}).
\end{align*}
\begin{align*}
\mathbf{u}\nabla_X=\frac{\partial}{\partial{X}^\beta}(u^\alpha\mathbf{G}_\alpha)
\mathbf{G}^\beta=u^\alpha\vert_\beta\mathbf{G}_\alpha\mathbf{G}^\beta=
(u^\alpha_{\;,\beta}+\Gamma^\alpha_{\gamma\beta}u^\gamma)\mathbf{G}_\alpha
\mathbf{G}^\beta,
\end{align*}
where $\mathbf{G}_\alpha\;,\,\mathbf{G}^\beta$ are the covariant and contravariant 
material base vectors, respectively. {\em mutatis mutandis} for spatial gradient
tensors $\nabla_x\mathbf{u}\;,\,\mathbf{u}\nabla_x.$
\item Material time derivative
\begin{align*}
\dot{\mathbf{v}}=\dot{\overline{v^i\mathbf{g}_i}}=
\frac{\partial{v^i}}{\partial{t}}\mathbf{g}_i+v^i\vert_jv^j\mathbf{g}_i=
\frac{\partial{v^i}}{\partial{t}}\mathbf{g}_i+(v^i_{\;,j}v^j+
\Gamma^i_{kj}v^kv^j)\mathbf{g}_i,
\end{align*}
where the local time derivative  $\partial\mathbf{g}_i/\partial{t}=\mathbf{0},$ by definition.
\end{enumerate}}
\end{itemize}
\begin{thebibliography}{1}
\bibitem{cite:Euler}
L. Euler, {\em Principes généraux du mouvement des fluides,} Hist. Acad. Berlin {\bf 1755,} 274-315, 1757.
\end{thebibliography}

%%%%%
%%%%%
\end{document}
