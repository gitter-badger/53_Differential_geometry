\documentclass[12pt]{article}
\usepackage{pmmeta}
\pmcanonicalname{TriangularwaveFunction}
\pmcreated{2013-03-22 15:18:43}
\pmmodified{2013-03-22 15:18:43}
\pmowner{pahio}{2872}
\pmmodifier{pahio}{2872}
\pmtitle{triangular-wave function}
\pmrecord{26}{37115}
\pmprivacy{1}
\pmauthor{pahio}{2872}
\pmtype{Definition}
\pmcomment{trigger rebuild}
\pmclassification{msc}{53A04}
\pmclassification{msc}{26A06}
\pmsynonym{triangular wave function}{TriangularwaveFunction}
\pmsynonym{saw-blade function}{TriangularwaveFunction}
\pmsynonym{saw blade function}{TriangularwaveFunction}
%\pmkeywords{periodic function}
\pmrelated{CyclometricFunctions}
\pmrelated{CommonFourierSeries}
\pmrelated{GettingTaylorSeriesFromDifferentialEquation}
\pmrelated{PeriodicExtension}
\pmrelated{LissajousCurves}

\endmetadata

% this is the default PlanetMath preamble.  as your knowledge
% of TeX increases, you will probably want to edit this, but
% it should be fine as is for beginners.

% almost certainly you want these
\usepackage{amssymb}
\usepackage{amsmath}
\usepackage{amsfonts}

\usepackage{pstricks}
\usepackage{pst-plot}

% used for TeXing text within eps files
%\usepackage{psfrag}
% need this for including graphics (\includegraphics)
\usepackage{graphicx}
% for neatly defining theorems and propositions
 \usepackage{amsthm}
% making logically defined graphics
%%%\usepackage{xypic}

% there are many more packages, add them here as you need them

% define commands here

\theoremstyle{definition}
\newtheorem*{thmplain}{Theorem}
\begin{document}
\PMlinkescapeword{inner function}

The \PMlinkname{arcsine}{CyclometricFunctions} is the inverse function of the sine.  Therefore the composition function
                    $$f:\,x\mapsto \arcsin(\sin{x})$$
is the identity map \, $x\mapsto x$\, on the interval \, $[-\frac{\pi}{2},\,\frac{\pi}{2}]$.\, On this interval, the inner function $\sin$ \PMlinkescapetext{increases monotonically and continuously from its least value $-1$ to its greatest value 1; then the outer} function $\arcsin$ (\PMlinkname{i.e.}{Ie} the angle corresponding the sine value) and the whole composition correspondingly grows from $-\frac{\pi}{2}$ to $\frac{\pi}{2}$.\, On the next equally long interval\, $[\frac{\pi}{2},\,\frac{3\pi}{2}]$,\, when the inner function decreases from 1 to $-1$, the composition thus decreases from $\frac{\pi}{2}$ to $-\frac{\pi}{2}$,\, evidently again linearly.\, We have now run through a \PMlinkescapetext{period} interval\, $[-\frac{\pi}{2},\,\frac{3\pi}{2}]$\, of the inner function and the composition $f$ and obtained a wedge-formed portion ($\wedge$) of the graph.\, Because of the periodicity, the whole graph of $f$ consists of such successive wedges and thus looks like a saw \PMlinkescapetext{blade}.\, The {\em triangular-wave function} is continuous.\, Its derivative (away from the \PMlinkescapetext{singular points} \, $\frac{\pi}{2}+n\pi,\, n\in\mathbb{Z}$) is a 
\PMlinkname{square-wave function}{CommonFourierSeries}.

\begin{center}
\begin{pspicture}(-6.6,-2.3)(6.7,2.3) 
\psline[linecolor=blue](-6.6,-0.3168)(-4.7124,1.5708)
\psline[linecolor=blue](-4.7124,1.5708)(-1.5708,-1.5708)
\psline[linecolor=blue](-1.5708,-1.5708)(1.5708,1.5708)
\psline[linecolor=blue](1.5708,1.5708)(4.7124,-1.5708)
\psline[linecolor=blue](4.7124,-1.5708)(6.6,0.3168)
\rput[l](-6.6,0){.}
\psaxes[labels=none,Dx=1.5708,Dy=1.5708]{->}(0,0)(-6.6,-1.8)(7.0,2.3)
\rput[a](7.1,-0.25){$x$}
\rput[r](-0.22,2.35){$y$}
\rput[a](0,-2.3){\textbf{Figure:} Graph of $\arcsin(\sin x)$}
\rput[a](-6.2832,-0.4){$2\pi$}
\rput[a](-4.7124,-0.4){$-\frac{3\pi}{2}$}
\rput[a](-3.1416,-0.4){$\pi$}
\rput[a](-1.5708,-0.4){$-\frac{\pi}{2}$}
\rput[a](1.5708,-0.4){$\frac{\pi}{2}$}
\rput[a](3.1416,-0.4){$\pi$}
\rput[a](4.7124,-0.4){$\frac{3\pi}{2}$}
\rput[a](6.2832,-0.4){$2\pi$}
\rput[r](-0.3,-1.5708){$-\frac{\pi}{2}$}
\rput[r](-0.3,1.5708){$\frac{\pi}{2}$}
\end{pspicture}
\end{center}

Sometimes, such a function is called a {\em saw-tooth function}, although this name usually refers to a discontinuous function with graph consisting of either ascending ($/$) or descending ($\backslash$) line segments with jumps.
%%%%%
%%%%%
\end{document}
