\documentclass[12pt]{article}
\usepackage{pmmeta}
\pmcanonicalname{LineInPlane}
\pmcreated{2013-03-22 15:18:29}
\pmmodified{2013-03-22 15:18:29}
\pmowner{matte}{1858}
\pmmodifier{matte}{1858}
\pmtitle{line in plane}
\pmrecord{17}{37108}
\pmprivacy{1}
\pmauthor{matte}{1858}
\pmtype{Definition}
\pmcomment{trigger rebuild}
\pmclassification{msc}{53A04}
\pmclassification{msc}{51N20}
\pmsynonym{y-intercept}{LineInPlane}
\pmsynonym{x-intercept}{LineInPlane}
%\pmkeywords{equation of line}
\pmrelated{LineSegment}
\pmrelated{SlopeAngle}
\pmrelated{LineInSpace}
\pmrelated{Slope}
\pmrelated{AnalyticGeometry}
\pmrelated{FanOfLines}
\pmrelated{PencilOfConics}
\pmdefines{$y$-intercept}
\pmdefines{$x$-intercept}
\pmdefines{slope-intercept form}

% this is the default PlanetMath preamble.  as your knowledge
% of TeX increases, you will probably want to edit this, but
% it should be fine as is for beginners.

% almost certainly you want these
\usepackage{amssymb}
\usepackage{amsmath}
\usepackage{amsfonts}
\usepackage{amsthm}

\usepackage{mathrsfs}

% used for TeXing text within eps files
%\usepackage{psfrag}
% need this for including graphics (\includegraphics)
%\usepackage{graphicx}
% for neatly defining theorems and propositions
%
% making logically defined graphics
%%%\usepackage{xypic}

% there are many more packages, add them here as you need them

% define commands here

\newcommand{\sR}[0]{\mathbb{R}}
\newcommand{\sC}[0]{\mathbb{C}}
\newcommand{\sN}[0]{\mathbb{N}}
\newcommand{\sZ}[0]{\mathbb{Z}}

 \usepackage{bbm}
 \newcommand{\Z}{\mathbbmss{Z}}
 \newcommand{\C}{\mathbbmss{C}}
 \newcommand{\F}{\mathbbmss{F}}
 \newcommand{\R}{\mathbbmss{R}}
 \newcommand{\Q}{\mathbbmss{Q}}



\newcommand*{\norm}[1]{\lVert #1 \rVert}
\newcommand*{\abs}[1]{| #1 |}



\newtheorem{thm}{Theorem}
\newtheorem{defn}{Definition}
\newtheorem{prop}{Proposition}
\newtheorem{lemma}{Lemma}
\newtheorem{cor}{Corollary}
\begin{document}
\subsubsection*{Equation of a line}
Suppose $a,b,c\in \R$. Then the set of points $(x,y)$ in the 
plane that satisf{y}
$$
   ax+by+c \;=\; 0,
$$
where $a$ and $b$ can not be both 0, is an (infinite) \emph{line}.

The value of $y$ when $x=0$, if it exists, is called the \emph{$y$-intercept}.  Geometrically, if $d$ is the $y$-intercept, then $(0,d)$ is the point of intersection of the line and the $y$-axis.  The $y$-intercept exists iff the line is not parallel to the $y$-axis.  The \emph{$x$-intercept} is defined similarly.

If $b\neq0$, then the above equation of the line can be rewritten as
$$
   y = mx + d.
$$
This is called the \emph{slope-intercept form} of a line, because both the slope and the $y$-intercept are easily identifiable in the equation.  The slope is $m$ and the $y$-intercept is $d$.

Three finite points $(x_1,\,y_1)$, $(x_2,\,y_2)$, $(x_3,\,y_3)$ in $\R^2$ are collinear if and only if the following determinant vanishes:
$$\left| \begin{array}{ccc} x_1 & x_2 &x_3 \\ y_1 & y_2 & y_3 \\ 1 & 1& 1\end{array} \right|=0.$$
Therefore, the line through distinct points $(x_1,\,y_1)$ and $(x_2,\,y_2)$ has equation 
$$\left| \begin{array}{ccc} x_1 & x_2 &x \\ y_1 & y_2 & y \\ 1 & 1& 1\end{array} \right|=0,$$ 
or more simply
$$
   (y_1-y_2)x+(x_2 - x_1)y + y_2 x_1-x_2 y_1=0.
$$

\subsubsection*{Line segment}

Let $p_1 = (x_1,\,y_1)$ and $p_2 = (x_2,\,y_2)$ be distinct points in $\R^2$.  The closed line segement generated by these points is the set
$$\{ p\in \R^2 \mid   p=t p_1+(1-t) p_2,\; 0\leq t\leq 1\}.$$
%%%%%
%%%%%
\end{document}
