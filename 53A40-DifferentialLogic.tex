\documentclass[12pt]{article}
\usepackage{pmmeta}
\pmcanonicalname{DifferentialLogic}
\pmcreated{2013-08-22 21:53:01}
\pmmodified{2013-08-22 21:53:01}
\pmowner{Jon Awbrey}{15246}
\pmmodifier{Jon Awbrey}{15246}
\pmtitle{differential logic}
\pmrecord{16}{40394}
\pmprivacy{1}
\pmauthor{Jon Awbrey}{15246}
\pmtype{Definition}
\pmcomment{trigger rebuild}
\pmclassification{msc}{53A40}
\pmclassification{msc}{39A12}
\pmclassification{msc}{34G99}
\pmclassification{msc}{03B44}
\pmclassification{msc}{03B42}
\pmclassification{msc}{03B15}
\pmrelated{DifferentialPropositionalCalculus}
\pmrelated{DifferentialPropositionalCalculusExamples}
\pmrelated{DifferentialPropositionalCalculusAppendices}
\pmrelated{DifferentialPropositionalCalculusAppendix2}
\pmrelated{DifferentialPropositionalCalculusAppendix3}
\pmrelated{DifferentialPropositionalCalculusAppendix4}

\endmetadata

% This is the default PlanetMath preamble.
% as your knowledge of TeX increases, you
% will probably want to edit this, but it
% should be fine as is for beginners.

% Almost certainly you want these:

\usepackage{amssymb}
\usepackage{amsmath}
\usepackage{amsfonts}

% Used for TeXing text within EPS files:

\usepackage{psfrag}

% Need this for including graphics (\includegraphics):

\usepackage{graphicx}

% For neatly defining theorems and propositions:

%\usepackage{amsthm}

% Making logically defined graphics:

%%%\usepackage{xypic}

% There are many more packages, add them here as you need them.

% define commands here

\begin{document}
\textbf{Differential logic} is the component of logic whose object is the description of variation --- for example, the aspects of change, difference, \PMlinkname{distribution}{Distribution}, and diversity --- in universes of discourse that are subject to logical description.  In formal logic, differential logic treats the principles that govern the use of a \textbf{differential logical calculus}, that is, a formal system with the expressive capacity to describe change and diversity in logical universes of discourse.

A simple example of a differential logical calculus is furnished by a \PMlinkname{differential propositional calculus}{DifferentialPropositionalCalculus}.  This extends an ordinary \PMlinkname{propositional calculus}{PropositionalCalculus} in the same way that \PMlinkname{differential calculus}{Calculus} extends analytic geometry.

\section{Readings}

\begin{itemize}
\item
Awbrey, J., ``\PMlinkexternal{Differential Logic : Introduction}{http://intersci.ss.uci.edu/wiki/index.php/Differential_Logic_:_Introduction}".  
\item
Awbrey, J., ``\PMlinkexternal{Differential Logic and Dynamic Systems}{http://intersci.ss.uci.edu/wiki/index.php/Differential_Logic_and_Dynamic_Systems_2.0}".
\end{itemize}

%%%%%
%%%%%
\end{document}
