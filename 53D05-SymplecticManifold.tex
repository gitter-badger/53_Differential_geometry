\documentclass[12pt]{article}
\usepackage{pmmeta}
\pmcanonicalname{SymplecticManifold}
\pmcreated{2013-03-22 13:12:18}
\pmmodified{2013-03-22 13:12:18}
\pmowner{matte}{1858}
\pmmodifier{matte}{1858}
\pmtitle{symplectic manifold}
\pmrecord{11}{33667}
\pmprivacy{1}
\pmauthor{matte}{1858}
\pmtype{Definition}
\pmcomment{trigger rebuild}
\pmclassification{msc}{53D05}
\pmrelated{ContactManifold}
\pmrelated{KahlerManifold}
\pmrelated{HyperkahlerManifold}
\pmrelated{MathbbCIsAKahlerManifold}
\pmdefines{symplectic form}
\pmdefines{symplectomorphism}
\pmdefines{canonical transformation}

\endmetadata

% this is the default PlanetMath preamble.  as your knowledge
% of TeX increases, you will probably want to edit this, but
% it should be fine as is for beginners.

% almost certainly you want these
\usepackage{amssymb}
\usepackage{amsmath}
\usepackage{amsfonts}
\usepackage{amsthm}

\usepackage{mathrsfs}

% used for TeXing text within eps files
%\usepackage{psfrag}
% need this for including graphics (\includegraphics)
%\usepackage{graphicx}
% for neatly defining theorems and propositions
%
% making logically defined graphics
%%%\usepackage{xypic}

% there are many more packages, add them here as you need them

% define commands here

\newcommand{\sR}[0]{\mathbb{R}}
\newcommand{\sC}[0]{\mathbb{C}}
\newcommand{\sN}[0]{\mathbb{N}}
\newcommand{\sZ}[0]{\mathbb{Z}}

 \usepackage{bbm}
 \newcommand{\Z}{\mathbbmss{Z}}
 \newcommand{\C}{\mathbbmss{C}}
 \newcommand{\R}{\mathbbmss{R}}
 \newcommand{\Q}{\mathbbmss{Q}}



\newcommand*{\norm}[1]{\lVert #1 \rVert}
\newcommand*{\abs}[1]{| #1 |}



\newtheorem{thm}{Theorem}
\newtheorem{defn}{Definition}
\newtheorem{prop}{Proposition}
\newtheorem{lemma}{Lemma}
\newtheorem{cor}{Corollary}
\begin{document}
Symplectic manifolds constitute
the mathematical structure for modern Hamiltonian mechanics.
Symplectic manifolds can also be seen as even dimensional 
analogues to contact manifolds. 

\begin{defn}
A {\em symplectic manifold} is a pair $(M,\omega)$ consisting 
of a smooth manifold $M$ and a 
closed \PMlinkname{2-form}{DifferentialForms} 
$\omega\in\Omega^2(M)$, that is non-degenerate
at each point.  
Then $\omega$ is called a {\em symplectic 
form} for $M$.
\end{defn}

\subsubsection*{Properties}
\begin{enumerate}
\item Every symplectic manifold is even dimensional. This is 
easy to understand in view of the physics. In Hamilton
equations, location and momentum vectors always appear in pairs.
\item A form $\omega\in \Omega^2(M)$ on a $2n$-dimensional 
manifold $M$ is non-degenerate if and only if the 
$n$-fold product $\omega^n= \omega\wedge \cdots \wedge \omega$
is non-zero.
\item As a consequence of the last \PMlinkescapetext{property}, every symplectic manifold
is orientable. 
\end{enumerate}

Let $(M,\omega)$ and $(N,\eta)$ be symplectic manifolds.  Then a diffeomorphism $f\colon M\to N$ is
called a {\em symplectomorphism} if $f^*\eta=\omega$, that is, if the symplectic form on $N$
pulls back to the form on $M$.

\subsubsection*{Notes}
A symplectomorphism is also known as a \emph{canonical transformation}.
This \PMlinkescapetext{term} is mostly used in the mechanics literature.
%%%%%
%%%%%
\end{document}
