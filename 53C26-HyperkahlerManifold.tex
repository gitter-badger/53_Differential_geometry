\documentclass[12pt]{article}
\usepackage{pmmeta}
\pmcanonicalname{HyperkahlerManifold}
\pmcreated{2013-03-22 15:50:16}
\pmmodified{2013-03-22 15:50:16}
\pmowner{tiphareth}{13221}
\pmmodifier{tiphareth}{13221}
\pmtitle{hyperk\"ahler manifold}
\pmrecord{4}{37813}
\pmprivacy{1}
\pmauthor{tiphareth}{13221}
\pmtype{Definition}
\pmcomment{trigger rebuild}
\pmclassification{msc}{53C26}
\pmsynonym{"hyper-K\"ahler manifold"}{HyperkahlerManifold}
\pmsynonym{"hyper-K\"ahlerian manifold"}{HyperkahlerManifold}
\pmrelated{Kahlermanifold}
\pmrelated{almostcomplexstructure}
\pmrelated{symplecticmanifold}
\pmrelated{quaternions}
\pmrelated{Quaternions}
\pmrelated{KahlerManifold}
\pmrelated{SymplecticManifold}
\pmrelated{AlmostComplexStructure}
\pmdefines{"hyperk\"ahler manifold"}
\pmdefines{"hypercomplex manifold"}

\endmetadata

% this is the default PlanetMath preamble.  as your knowledge
% of TeX increases, you will probably want to edit this, but
% it should be fine as is for beginners.

% almost certainly you want these
\usepackage{amssymb}
\usepackage{amsmath}
\usepackage{amsfonts}

% used for TeXing text within eps files
%\usepackage{psfrag}
% need this for including graphics (\includegraphics)
%\usepackage{graphicx}
% for neatly defining theorems and propositions
%\usepackage{amsthm}
% making logically defined graphics
%%%\usepackage{xypic}

% there are many more packages, add them here as you need them

% define commands here
\begin{document}
{\bf Definition:}
Let {\it M} be a smooth manifold, and {\it I,J,K $\in$ End(TM)}
endomorphisms of the tangent bundle satisfying the
quaternionic relation
\[
I^2=J^2=K^2=IJK=-Id_{TM}.
\]
The manifold {\it (M,I,J,K)} is called {\bf hypercomplex}
if the almost complex structures {\it I, J, K}
are {\bf integrable}. If, in addition, {\it M}
is equipped with a Riemannian metric {\it g} which
is K\"ahler with respect to {\it I,J,K}, the
manifold {\it (M,I,J,K,g)} is called {\bf hyperk\"ahler}. 

\hfill

Since {\it g} is K\"ahler with respect to
{\it (I,J,K)}, we have
\[
\nabla I = \nabla J = \nabla K=0
\]
where $\nabla$ denotes the Levi-Civita connection.
This means that the holonomy of $\nabla$ lies inside
the group {\it Sp(n)} of quaternionic-Hermitian
endomorphisms. The converse is also true: a
Riemannian manifold is hyperk\"ahler if and only
if its holonomy is contained in {\it Sp(n)}.
This definition is standard in differential
geometry. 

In physics literature, one sometimes assumes 
that  the holonomy of a hyperk\"ahler manifold is 
precisely {\it Sp(n)}, and not its proper
subgroup. In mathematics, such hyperk\"ahler
manifolds are called {\bf simple hyperk\"ahler manifolds}.

The following splitting theorem (due to F. Bogomolov)
is implied by Berger's classification of irreducible 
holonomies.

\hfill

{\bf Theorem:} Any hyperk\"ahler
manifold has a finite covering which is a product
of a hyperk\"ahler torus and several simple
hyperk\"ahler manifolds.

\hfill


Consider the K\"ahler forms $\omega_I, \omega_J, \omega_K$
on {\it M}:
\[
\omega_I(\cdot, \cdot):= g(\cdot, I\cdot), \ \
\omega_J(\cdot, \cdot):= g(\cdot, J\cdot), \ \
\omega_K(\cdot, \cdot):= g(\cdot, K\cdot).
\]
An elementary linear-algebraic calculation implies
that the 2-form $\omega_J+\sqrt{-1}\omega_K$ is of Hodge type (2,0)
on {\it (M,I)}. This form is clearly closed and
non-degenerate, hence it is a holomorphic 
symplectic form.

In algebraic geometry, the word ``hyperk\"ahler''
is essentially synonymous with ``holomorphically
symplectic'', due to the following theorem, which is
implied by Yau's solution of Calabi conjecture
(the famous Calabi-Yau theorem).

\hfill

%%%%%%%%%%%%%%%%%%%%%%%%%%%%%%%%%%%%%%%%%%%%%%%%
{\bf Theorem:} \label{_Calabi-Yau_Theorem_}
Let {\it (M,I)} be a compact, K\"ahler, holomorphically
symplectic manifold. Then there exists a unique
hyperk\"ahler metric on {\it (M,I)} with the same K\"ahler class.

\hfill

%%%%%%%%%%%%%%%%%%%%%%%%%%%%%%%%%%%%%%%%%%%%%%%%
{\bf Remark:}
The hyperk\"ahler metric is unique, but there could
be several hyperk\"ahler structures compatible with
a given hyperk\"ahler metric on {\it (M,I)}.

{\small
\begin{thebibliography}{6666}

\bibitem[Bea]{_Beauville_} 
 Beauville, A. {\em 
Varietes K\"ahleriennes dont la premi\`ere classe de Chern est
nulle.}  J. Diff. Geom. {\bf 18}, pp. 755-782 (1983).

\bibitem[Bes]{_Besse:Einst_Manifo_} 
Besse, 
A., {\em Einstein Manifolds}, Springer-Verlag, New York (1987)

\bibitem[Bo1]{_Bogomolov:decompo_}  
Bogomolov, F. {\em On the decomposition of 
K\"ahler manifolds with trivial canonical class}, Math. USSR-Sb.
{\bf 22} (1974), 580-583.

\bibitem[Y]{_Yau:Calabi-Yau_} 
Yau, S. T., {\em On the Ricci curvature of a compact K\"ahler manifold 
and the complex Monge-Amp\`ere equation I.}  Comm. on Pure and Appl.
Math. 31, 339-411 (1978).

\end{thebibliography}

}% end of small
%%%%%
%%%%%
\end{document}
