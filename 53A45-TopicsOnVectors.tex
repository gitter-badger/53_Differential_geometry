\documentclass[12pt]{article}
\usepackage{pmmeta}
\pmcanonicalname{TopicsOnVectors}
\pmcreated{2013-03-22 15:49:56}
\pmmodified{2013-03-22 15:49:56}
\pmowner{perucho}{2192}
\pmmodifier{perucho}{2192}
\pmtitle{topics on vectors}
\pmrecord{32}{37804}
\pmprivacy{1}
\pmauthor{perucho}{2192}
\pmtype{Topic}
\pmcomment{trigger rebuild}
\pmclassification{msc}{53A45}
\pmsynonym{topic entry on vectors}{TopicsOnVectors}

% this is the default PlanetMath preamble.  as your knowledge
% of TeX increases, you will probably want to edit this, but
% it should be fine as is for beginners.

% almost certainly you want these
\usepackage{amssymb}
\usepackage{amsmath}
\usepackage{amsfonts}

% used for TeXing text within eps files
%\usepackage{psfrag}
% need this for including graphics (\includegraphics)
%\usepackage{graphicx}
% for neatly defining theorems and propositions
%\usepackage{amsthm}
% making logically defined graphics
%%%\usepackage{xypic}

% there are many more packages, add them here as you need them

% define commands here
\begin{document}
\subsection*{I Vector algebra}
\begin{enumerate}
\item definition of vector
\item vector space
\item parallelogram principle, median vector, difference of vectors
\item geometric applications: mid-segment theorem, common point of triangle medians, median of trapezoid
\item system of coordinates
\item basis
\item coordinate vector
\item position vector
\item \PMlinkname{norm}{VectorPNorm} (through Pythagoras)
\item unit vector
\item direction cosines
\item dot product
\item \PMlinkid{parallelism condition}{6178}
\item \PMlinkid{orthogonality condition}{6178}
\item vector components and scalar components
\item cross product
\item \PMlinkname{area}{CrossProduct} of parallelogram 
\item triple scalar product, volume of prism
\item triple cross product
\item distance of non-parallel lines (an application)
\item matrices and determinants
\item matrices and linear mappings
\item linear systems and solution methods
\end{enumerate}

\subsection*{II Vector calculus}
\begin{enumerate}
\item definiton of real valued vector function
\item derivative of vector function
\item properties of derivative of vector function
\item derivative of a vector function with constant norm
\item nabla
\item cylindrical coordinates
\item polar coordinates
\item spherical coordinates
\item \PMlinkname{differential geometry}{ClassicalDifferentialGeometry}
\item \PMlinkname{tangent}{TangentSpace}, \PMlinkname{normal}{NormalVector} and binormal vectors
\item osculating plane, normal plane and binormal planes
\item Frenet frame
\item Frenet-Serret equations
\item kinematic method for calculating the \PMlinkname{radius of curvature}{CurvatureOfACurve}
\item gradient of a scalar function
\item divergence of a vector function
\item solenoidal field
\item vector potential
\item curl of a vector function
\item irrotational field, lamellar field
\item Helmholtz decomposition
\item integration of vector functions
\item line integral
\item tensors and differential forms
\item covariant differentiation
\end{enumerate}

\subsection*{III Integral theorems}
\begin{enumerate}
\item Gauss theorem
\item solid angle
\item Green theorems
\item Stokes theorem
\item circulation and vorticity
\item Kelvin theorem
\item Helmholtz theorems
\end{enumerate}

\subsection*{IV Vector advanced topics}
\begin{enumerate}
\item alternate characterization of curl
\item tensor notation for a vector
\item transformation law for a vector
\item vector fields: Lagrangian and Eulerian description
\item motion of continuum
\item Jacobians connected with transformation of integration regions
\item Reynolds transport theorem
\item rotations
\item linear transformation spaces
\item linear functionals or covectors
\item bivectors
\item exterior or Grassmann algebra
\item Clifford algebra
\item quaternions
\item projective geometry
\item Grassmann-Cayley algebra
\item vector bundles
\item connections
\item spinors
\item twistors
\item spin structures
\item linear programming and the simplex method
\item representation theory
\item linear extension
\item K-theory
\item Category ${\rm Vect}_{\mathbb{R}}$
\end{enumerate}

\subsection*{V Endomorphism decomposition}
\begin{enumerate}
\item eigenvalues, eigenvectors
\item characteristic and minimal polynomials
\item eigen-subspaces and invariant subspaces
\item \PMlinkname{Hamilton-Cayley theorem}{CayleyHamiltonTheorem}
\item Jordan blocks and canonical decomposition
\item singular value decomposition
\end{enumerate}

\subsection*{VI Lie groups and Lie algebras}
\begin{enumerate}
\item the connection between Lie groups and Lie algebras
\item commutators or Lie bracket
\item matrix groups and algebras
\item Pauli matrices
 
\end{enumerate}
%%%%%
%%%%%
\end{document}
