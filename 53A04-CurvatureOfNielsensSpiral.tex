\documentclass[12pt]{article}
\usepackage{pmmeta}
\pmcanonicalname{CurvatureOfNielsensSpiral}
\pmcreated{2015-02-06 12:53:54}
\pmmodified{2015-02-06 12:53:54}
\pmowner{pahio}{2872}
\pmmodifier{pahio}{2872}
\pmtitle{curvature of Nielsen's spiral}
\pmrecord{22}{39350}
\pmprivacy{1}
\pmauthor{pahio}{2872}
\pmtype{Example}
\pmcomment{trigger rebuild}
\pmclassification{msc}{53A04}
\pmsynonym{arc length of Nielsen's spiral}{CurvatureOfNielsensSpiral}
%\pmkeywords{arc length}
\pmrelated{CosineIntegral}
\pmrelated{SineIntegral}
\pmrelated{FamousCurvesInThePlane}
\pmrelated{DerivativeForParametricForm}
\pmdefines{Nielsen's spiral}

\endmetadata

% this is the default PlanetMath preamble.  as your knowledge
% of TeX increases, you will probably want to edit this, but
% it should be fine as is for beginners.

% almost certainly you want these
\usepackage{amssymb}
\usepackage{amsmath}
\usepackage{amsfonts}

% used for TeXing text within eps files
%\usepackage{psfrag}
% need this for including graphics (\includegraphics)
\usepackage{graphicx}
% for neatly defining theorems and propositions
 \usepackage{amsthm}
% making logically defined graphics
%%%\usepackage{xypic}
%\usepackage{pstricks}
%\usepackage{pst-plot}

% there are many more packages, add them here as you need them

% define commands here
\DeclareMathOperator{\ci}{ci}
\DeclareMathOperator{\si}{si}
\theoremstyle{definition}
\newtheorem*{thmplain}{Theorem}

\begin{document}
{\em Nielsen's spiral} is the plane curve defined in the 
parametric form
\begin{align}
x = a\,\mbox{ci}\,{t}, \quad y = a\,\mbox{si}\,{t} 
\end{align}
where $a$ is a non-zero constant, ``$\mbox{ci}$'' and 
``$\mbox{si}$'' are the \PMlinkname{cosine integral}{sineintegral} 
and \PMlinkname{the sine integral}{sineintegral} 
and $t$ is the \PMlinkname{parameter}{Parametre} ($t > 0$).

We determine the \PMlinkname{curvature}{CurvaturePlaneCurve} $\kappa$ of this curve using the expression
\begin{align}
\kappa = \frac{x'y''-y'x''}{{[}(x')^2+(y')^2{]}^{3/2}}.
\end{align}

The first derivatives of (1) are
\begin{align}
x' = \frac{d}{dt}\left(a\int_\infty^t\frac{\cos{u}}{u}\,du\!\right) 
\;=\; \frac{a\cos{t}}{t},
\end{align}
\begin{align}
y' \;=\; \frac{d}{dt}\left(a\int_\infty^t\frac{\sin{u}}{u}\,du\!\right) 
\;=\; \frac{a\sin{t}}{t},
\end{align}
and hence the second derivatives
$$x'' = -a\cdot\frac{t\sin{t}+\cos{t}}{t^2}, \quad y'' = a\cdot\frac{t\cos{t}-\sin{t}}{t^2}.$$
Substituting the derivatives in (2) yields
$$\kappa \;=\; 
a^2\!\cdot\!\frac{(\cos{t})(t\cos{t}-\sin{t})+(\sin{t})(t\sin{t}+\cos{t})}{t\cdot t^2}
\!:\!\left(\frac{a^2\cos^2{t}+a^2\sin^2{t}}{t^2}\right)^{\frac{3}{2}}\!,$$
which is easily simplified to
\begin{align}
\kappa \;=\; \frac{t}{a}.
\end{align}


The \PMlinkname{arc length}{ArcLength} of Nielsen's spiral can also be obtained in a \PMlinkescapetext{simple} \PMlinkname{closed form}{ClosedForm4}; using (3) and (4) we get:
$$s \;=\; \int_1^t\sqrt{x'^2\!+\!y'^2}\,dt \;=\; 
\int_1^t\sqrt{ \frac{a^2\cos^2t}{t^2}+\frac{a^2\sin^2t}{t^2} }\,dt \;=\; \int_1^t\frac{a}{t}\,dt,$$
i.e.
\begin{align}
s \;=\; a\ln{t}.
\end{align}


\textbf{Note.}\, The expressions for $x'$ and $y'$ allow us determine as well
$$\frac{dy}{dx} \;=\; \frac{y'}{x'} \;=\; \frac{\sin{t}}{\cos{t}} \;=\; \tan{t},$$
which says that the sense of the parameter $t$ is the slope angle of the tangent line of the Nielsen's spiral.

\begin{figure}
\begin{center}
\includegraphics{nielsen}
\caption{Plot of Nielsen's spiral for $2 \leq t \leq 50$.  
Axis scaling is in units of $a$. 
{\small 
(Octave / MATLAB \PMlinkexternal{source program}{http://aux.planetmath.org/files/objects/9350/nielsen.m} for plot;
in \PMlinkexternal{PDF format}{http://aux.planetmath.org/files/objects/9350/nielsen.pdf})}}
\end{center}
\end{figure}


%%%%%
%%%%%
\end{document}
