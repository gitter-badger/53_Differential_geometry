\documentclass[12pt]{article}
\usepackage{pmmeta}
\pmcanonicalname{Isocline}
\pmcreated{2013-03-22 18:05:52}
\pmmodified{2013-03-22 18:05:52}
\pmowner{pahio}{2872}
\pmmodifier{pahio}{2872}
\pmtitle{isocline}
\pmrecord{6}{40638}
\pmprivacy{1}
\pmauthor{pahio}{2872}
\pmtype{Definition}
\pmcomment{trigger rebuild}
\pmclassification{msc}{53A25}
\pmclassification{msc}{53A04}
\pmclassification{msc}{51N05}
\pmrelated{OrthogonalCurves}

\endmetadata

% this is the default PlanetMath preamble.  as your knowledge
% of TeX increases, you will probably want to edit this, but
% it should be fine as is for beginners.

% almost certainly you want these
\usepackage{amssymb}
\usepackage{amsmath}
\usepackage{amsfonts}

% used for TeXing text within eps files
%\usepackage{psfrag}
% need this for including graphics (\includegraphics)
%\usepackage{graphicx}
% for neatly defining theorems and propositions
 \usepackage{amsthm}
% making logically defined graphics
%%%\usepackage{xypic}

% there are many more packages, add them here as you need them

% define commands here

\theoremstyle{definition}
\newtheorem*{thmplain}{Theorem}

\begin{document}
Let $\Gamma$ be a family of plane curves.\, The {\em isocline} of $\Gamma$ is the locus of the points, in which all members of $\Gamma$ have an equal slope.

If the family $\Gamma$ has the differential equation
$$F(x,\,y,\,\frac{dy}{dx}) = 0,$$
then the equation of any isocline of $\Gamma$ has the form
$$F(x,\,y,\,K) = 0$$
where $K$ is \PMlinkescapetext{constant}.\\

For example, the family
$$y = e^{Cx}$$
of \PMlinkname{exponential}{ExponentialFunction} curves satisfies the differential equation\, $\frac{dy}{dx} = Ce^{Cx}$\, or\, $\frac{dy}{dx} = Cy$,\, whence the isoclines are\, $Cy =K$,\, i.e. they are horizontal lines.\\


\PMlinkexternal{Wiki}{http://en.wikibooks.org/wiki/Differential_Equations/Isoclines_1}

%%%%%
%%%%%
\end{document}
