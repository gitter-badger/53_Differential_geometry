\documentclass[12pt]{article}
\usepackage{pmmeta}
\pmcanonicalname{PseudoRiemannianManifold}
\pmcreated{2013-03-22 15:44:15}
\pmmodified{2013-03-22 15:44:15}
\pmowner{cvalente}{11260}
\pmmodifier{cvalente}{11260}
\pmtitle{pseudo-Riemannian manifold}
\pmrecord{10}{37688}
\pmprivacy{1}
\pmauthor{cvalente}{11260}
\pmtype{Definition}
\pmcomment{trigger rebuild}
\pmclassification{msc}{53Z05}
\pmrelated{EinsteinFieldEquations}
\pmrelated{SylvestersLaw}
\pmrelated{MinkowskiSpace}
\pmrelated{CategoryOfRiemannianManifolds}
\pmdefines{pseudo-Riemannian geometry}
\pmdefines{pseudo-Riemannian manifold}

% this is the default PlanetMath preamble.  as your knowledge
% of TeX increases, you will probably want to edit this, but
% it should be fine as is for beginners.

% almost certainly you want these
\usepackage{amssymb}
\usepackage{amsmath}
\usepackage{amsfonts}

% used for TeXing text within eps files
%\usepackage{psfrag}
% need this for including graphics (\includegraphics)
%\usepackage{graphicx}
% for neatly defining theorems and propositions
%\usepackage{amsthm}
% making logically defined graphics
%%%\usepackage{xypic}

% there are many more packages, add them here as you need them

% define commands here
\begin{document}
A \emph{pseudo-Riemannian} manifold is a manifold $M$ together with a \PMlinkname{non degenerate}{NonDegenerateBilinearForm}, symmetric section $g$ of $T^0_{2}(M)$ (2-covariant tensor bundle over $M$).

Unlike with a Riemannian manifold, $g$ is not positive definite. That is, there exist vectors $v\in T_{p}M$ such that $g(v,v)\le0$.

A well known \PMlinkname{result from linear algebra}{SylvestersLaw} permits us to make a change of basis such that in the new base $g$ is represented by a diagonal matrix with $-1$ or $1$ elements in the diagonal. If there are $i$, $-1$ elements in the diagonal and $j$, $1$, the tensor is said to have signature $(i,j)$

The signature will be invariant in every connected component of $M$, but usually the restriction that it be a global invariant is added to the definition of a pseudo-Riemannian manifold.

Unlike a Riemannian metric, some manifolds do not admit a pseudo-Riemannian metric.

Pseudo-Riemannian manifolds are crucial in Physics and in particular in General Relativity where space-time is modeled as a 4-pseudo Riemannian manifold with signature (1,3)\footnote{also referred to as $(-+++)$}.

Intuitively pseudo-Riemannian manifolds are generalizations of Minkowski's space just as a Riemannian manifold is a generalization of a vector space with a positive definite metric.
%%%%%
%%%%%
\end{document}
