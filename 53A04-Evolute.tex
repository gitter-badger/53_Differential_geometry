\documentclass[12pt]{article}
\usepackage{pmmeta}
\pmcanonicalname{Evolute}
\pmcreated{2013-03-22 17:35:06}
\pmmodified{2013-03-22 17:35:06}
\pmowner{pahio}{2872}
\pmmodifier{pahio}{2872}
\pmtitle{evolute}
\pmrecord{8}{39997}
\pmprivacy{1}
\pmauthor{pahio}{2872}
\pmtype{Topic}
\pmcomment{trigger rebuild}
\pmclassification{msc}{53A04}
\pmrelated{ConditionOfOrthogonality}
\pmrelated{ArcLength}
\pmrelated{BolzanosTheorem}
\pmrelated{SubstitutionNotation}
\pmdefines{evolute}

% this is the default PlanetMath preamble.  as your knowledge
% of TeX increases, you will probably want to edit this, but
% it should be fine as is for beginners.

% almost certainly you want these
\usepackage{amssymb}
\usepackage{amsmath}
\usepackage{amsfonts}

% used for TeXing text within eps files
%\usepackage{psfrag}
% need this for including graphics (\includegraphics)
%\usepackage{graphicx}
% for neatly defining theorems and propositions
%\usepackage{amsthm}
% making logically defined graphics
%%%\usepackage{xypic}

% there are many more packages, add them here as you need them

% define commands here
\newcommand{\sijoitus}[2]%
{\operatornamewithlimits{\Big/}_{\!\!\!#1}^{\,#2}}
\begin{document}
The locus of the center of curvature of a plane curve is called the {\em evolute} of this curve.

The coordinates of the center of curvature belonging to the point \,$P = (x,\,y)$\, of the curve $\gamma$ are
\begin{align}
\xi \;=\; x-\varrho\,\sin\alpha,\qquad \eta \;=\; y+\varrho\,\cos\alpha,
\end{align}
where $\varrho$ is the radius of curvature in $P$ and $\alpha$ is the slope angle of the tangent line of the curve in $P$.\, So (1) may be regarded as the equations of the evolute of $\gamma$.\\

If the plane curve is given in the parametric form \, $x = x(t),\;\, y = y(t)$,\, the corresponding parametric equations of the evolute are
 $$\xi \;=\; x-\frac{(x'^{\,2}\!+\!y'^{\,2})y'}{x'y''\!-\!x''y'},\qquad
 \eta \;=\; y+\frac{(x'^{\,2}\!+\!y'^{\,2})x'}{x'y''\!-\!x''y'}.$$
In the spexial case that the curve is given in the form\, $y = y(x)$\, these equations can be written
 $$\xi \;=\; x-\frac{(1\!+\!y'^{\,2})y'}{y''},\qquad \eta \;=\; y+\frac{1\!+\!y'^{\,2}}{y''}.$$


For examining the properties of the evolute we choose for parameter the arc length $s$, measured from a certain point of the curve; then in (1) the quantities $x,\,y,\,\varrho,\,\alpha$ and thus $\xi$ and $\eta$ are functions of $s$.  We assume that all needed derivatives exist and are continuous.

Differentiating (1) with respect to $s$, we obtain
 $$\frac{d\xi}{ds} \;=\; \frac{dx}{ds}-\varrho\frac{d\alpha}{ds}\cos\alpha-\frac{d\varrho}{ds}\sin\alpha,\qquad
  \frac{d\eta}{ds} \;=\; \frac{dy}{ds}-\varrho\frac{d\alpha}{ds}\sin\alpha+\frac{d\varrho}{ds}\cos\alpha,$$
and recalling that\, $\frac{dx}{ds} = \cos\alpha$,\, $\frac{dy}{ds} = \sin\alpha$\, and\, $\varrho\frac{d\alpha}{ds} = 1$\, it yields
\begin{align}
\frac{d\xi}{ds} \;=\; -\frac{d\varrho}{ds}\sin\alpha,\qquad \frac{d\eta}{ds} \;=\; \frac{d\varrho}{ds}\cos\alpha.
\end{align}
If\, $\frac{d\varrho}{ds} \neq 0$\, in the point \,$(x,\,y)$\, of $\gamma$, the derivatives $\frac{d\xi}{ds}$ and $\frac{d\eta}{ds}$ do not vanish simultaneously, and so the evolute has in the corresponding point\, $(\xi,\,\eta)$\, a tangent line with the slope
 $$\frac{d\eta}{ds}:\frac{d\xi}{ds} \;=\; -\frac{1}{\tan\alpha}.$$
Since the \PMlinkescapetext{right side} of this is the slope of the normal line of the given curve $\gamma$, we have the

\textbf{Theorem 1.}\, The normal line of the curve in a point\, $(x,\,y)$,\, where\, $\displaystyle\frac{d\varrho}{ds} \neq 0$, is the tangent line of the evolute, having as tangency point the corresponding center of curvature\, $(\xi,\,\eta)$.\, Thus the evolute is the envelope of the normal lines of the curve.\\

We shall calculate the arc length $\sigma$ of the evolute corresponding the arc of the curve $\gamma$ which is passed through when the parameter $s$ grows from $s_1$ to $s_2$; we assume that $\varrho$ and $\frac{d\varrho}{ds}$ are then continuous and distinct from zero.  According the arc length formula,
 $$\sigma \;=\; \int_{s_1}^{s_2}\sqrt{\left(\frac{d\xi}{ds}\right)^{\!2}+\left(\frac{d\eta}{ds}\right)^{\!2}}\,ds.$$
Using the equations (2) and the fact that the sign of $\frac{d\varrho}{ds}$ does not change, we can write
 $$\sigma \;=\; \int_{s_1}^{s_2}\sqrt{\left(\frac{d\varrho}{ds}\right)^{\!2}}\,ds \;=\; 
  \int_{s_1}^{s_2}\left|\frac{d\varrho}{ds}\right|\,ds \;=\; \left|\int_{s_1}^{s_2}\frac{d\varrho}{ds}\,ds\right|
\;=\; \left|\sijoitus{s_1}{\quad s_2}\varrho\right| \;=\; |\varrho_2\!-\!\varrho_1|,$$
where $\varrho_1$ and $\varrho_2$ are the corresponding \PMlinkescapetext{radii of curvature} of $\gamma$.  We have proved the

\textbf{Theorem 2.}\, The \PMlinkescapetext{length} of an arc of the evolute is equal to the difference of the \PMlinkescapetext{radii of curvature} of the given curve touching the arc of the evolute in its end points, provided that $\varrho$ and $\frac{d\varrho}{ds}$ are continuous and do not change their sign on the arc of the curve.

\begin{thebibliography}{8}
\bibitem{lindelof}{\sc Ernst Lindel\"of}: {\em Differentiali- ja integralilasku
ja sen sovellutukset I}.  Toinen painos. WSOY, Helsinki (1950).
\end{thebibliography} 

%%%%%
%%%%%
\end{document}
