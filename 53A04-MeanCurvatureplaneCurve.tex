\documentclass[12pt]{article}
\usepackage{pmmeta}
\pmcanonicalname{MeanCurvatureplaneCurve}
\pmcreated{2013-03-22 15:31:16}
\pmmodified{2013-03-22 15:31:16}
\pmowner{Mathprof}{13753}
\pmmodifier{Mathprof}{13753}
\pmtitle{mean curvature (plane curve)}
\pmrecord{11}{37392}
\pmprivacy{1}
\pmauthor{Mathprof}{13753}
\pmtype{Definition}
\pmcomment{trigger rebuild}
\pmclassification{msc}{53A04}
\pmrelated{MeanCurvatureAtSurfacePoint}
\pmdefines{total curvature}
\pmdefines{mean curvature}

\endmetadata

% this is the default PlanetMath preamble.  as your knowledge
% of TeX increases, you will probably want to edit this, but
% it should be fine as is for beginners.

% almost certainly you want these
\usepackage{amssymb}
\usepackage{amsmath}
\usepackage{amsfonts}

% used for TeXing text within eps files
%\usepackage{psfrag}
% need this for including graphics (\includegraphics)
%\usepackage{graphicx}
% for neatly defining theorems and propositions
%\usepackage{amsthm}
% making logically defined graphics
%%%\usepackage{xypic}

% there are many more packages, add them here as you need them

% define commands here
\begin{document}
Let $ \Gamma $ be a piecewise $C^1$ planar curve. 

The \emph{total curvature}, $\kappa_{total}$ ,  of $\Gamma$ is defined to be $\int_{\Gamma} |\kappa(s)| ds $ where $\Gamma$ is parameterized by arclength $s$
and $\kappa(s)$ is the \PMlinkname{curvature}{CurvatureOfACurve}
of $\Gamma$.

The \emph{mean curvature} of $\Gamma$ is defined to be the ratio of the total curvature to the length of $\Gamma$ : $$ M(\Gamma) = \frac{\kappa_{total} (\Gamma)}{L(\Gamma)}$$
%%%%%
%%%%%
\end{document}
