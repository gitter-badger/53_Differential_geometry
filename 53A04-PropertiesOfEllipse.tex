\documentclass[12pt]{article}
\usepackage{pmmeta}
\pmcanonicalname{PropertiesOfEllipse}
\pmcreated{2013-03-22 18:53:43}
\pmmodified{2013-03-22 18:53:43}
\pmowner{pahio}{2872}
\pmmodifier{pahio}{2872}
\pmtitle{properties of ellipse}
\pmrecord{18}{41742}
\pmprivacy{1}
\pmauthor{pahio}{2872}
\pmtype{Topic}
\pmcomment{trigger rebuild}
\pmclassification{msc}{53A04}
\pmclassification{msc}{51N20}
\pmclassification{msc}{51-00}
\pmrelated{HeronsPrinciple}
\pmrelated{Conic}
\pmrelated{PerimeterOfEllipse}
\pmrelated{PlanarArea}
\pmrelated{KeplersFirstLaw}
\pmrelated{PropertiesOfParabola}
\pmrelated{Hyperbola2}
\pmrelated{AreaOfPlaneRegion}

\endmetadata

% this is the default PlanetMath preamble.  as your knowledge
% of TeX increases, you will probably want to edit this, but
% it should be fine as is for beginners.

% almost certainly you want these
\usepackage{amssymb}
\usepackage{amsmath}
\usepackage{amsfonts}
\usepackage{amsthm}

\usepackage{mathrsfs}
\usepackage{pstricks}
\usepackage{pst-plot}

% used for TeXing text within eps files
%\usepackage{psfrag}
% need this for including graphics (\includegraphics)
%\usepackage{graphicx}
% for neatly defining theorems and propositions
%
% making logically defined graphics
%%%\usepackage{xypic}

% there are many more packages, add them here as you need them

% define commands here

\newcommand{\sR}[0]{\mathbb{R}}
\newcommand{\sC}[0]{\mathbb{C}}
\newcommand{\sN}[0]{\mathbb{N}}
\newcommand{\sZ}[0]{\mathbb{Z}}

 \usepackage{bbm}
 \newcommand{\Z}{\mathbbmss{Z}}
 \newcommand{\C}{\mathbbmss{C}}
 \newcommand{\F}{\mathbbmss{F}}
 \newcommand{\R}{\mathbbmss{R}}
 \newcommand{\Q}{\mathbbmss{Q}}



\newcommand*{\norm}[1]{\lVert #1 \rVert}
\newcommand*{\abs}[1]{| #1 |}



\newtheorem{thm}{Theorem}
\newtheorem{defn}{Definition}
\newtheorem{prop}{Proposition}
\newtheorem{lemma}{Lemma}
\newtheorem{cor}{Corollary}
\begin{document}
The ellipse is defined as the locus of the points $P$ of the plane such that the distances of $P$ from two fixed points (the foci) have a constant sum ($2a$).\, The ellipse gets the simplest equation, when the foci are on a coordinate axis equally distant ($= c < a$) from the origin.\, So, if the foci are\, $(\pm c,\,0)$,\, the equation of the ellipse may be written first
\begin{align}
\sqrt{(x\!-\!c)^2\!+\!y^2}+\sqrt{(x\!+\!c)^2\!+\!y^2} \;=\; 2a.
\end{align}
After two squarings it is simplified to
\begin{align}
(a^2\!-\!c^2)x^2\!+\!a^2y^2 \,=\, a^2(a^2\!-\!c^2).
\end{align}
Denoting\, $a^2\!-\!c^2 := b^2$ (where\, $b > 0$) the equation of the ellipse attains the form
\begin{align}
\frac{x^2}{a^2}+\frac{y^2}{b^2} \;=\; 1.
\end{align}
From the equation one sees that\, $x = \pm a$\, give the rightmost and the leftmost points of the ellipse (on the $x$-axis); the segment between them is the major axis, with length $2a$.\, Similarly,\, $y = \pm b$\, give the highest and the lowest points (on the $y$-axis); the segment between them is the minor axis, with length $2b$.\, The distance between a focus and an end point of the minor axis is $a$ (see the yellow triangle).\\

\begin{center}
\begin{pspicture}(-4.7,-3.5)(4.7,3.5)
\psaxes[Dx=9,Dy=9]{->}(0,0)(-4.2,-3.4)(4.5,3.2)
\rput(-4.7,-3.5){.}
\rput(4.7,3.5){.}
\rput[b](4.63,0.1){$x$}
\rput[r](0.15,3.5){$y$}
\psellipse[linecolor=blue](0,0)(3.5,2)
\psdots[linecolor=red](-2.87,0)(2.87,0)
\psdot[linecolor=blue](2,1.61)
\psline(2,1.61)(1.3,0)
\rput(1.3,-0.25){$N$}
\rput(2.75, 1.85){$P = (x,\,y)$}
\pspolygon[linecolor=yellow](-2.87,0)(0,-2)(0,0)
\psline[linecolor=red](-2.87,0)(2,1.61)(2.87,0)
\rput(3.6,-0.2){$a$}
\rput(-3.78,-0.22){$-a$}
\rput(2.87,-0.23){$c$}
\rput(-3,-0.23){$-c$}
\rput(0.15,2.2){$b$}
\rput(2.7,0.8){$r_1$}
\rput(-1,0.8){$r_2$}
\rput(-1.2,-0.2){$c$}
\rput(-0.2,-1){$b$}
\rput(-1.6,-1.1){$a$}
\end{pspicture}
\end{center}

The eccentricity $\varepsilon$ is the ratio of the ``focal length'' $2c$ and the major axis $2a$.\, The focal radii $r_1$ and $r_2$ satisfy
$$r_1^2 = (c-x)^2+y^2, \quad r_2^2 = (-c-x)^2+y^2,$$
whence the subtraction yields
$$r_2^2\!-\!r_1^2 \;=\; 4cx.$$
This gives, since\, $r_2\!+\!r_1 = 2a$,\, that 
$$r_2\!-\!r_1 \;=\; 2\varepsilon x,$$
and the both last equations imply the expressions
\begin{align}
r_1 \;=\; a\!-\!\varepsilon x, \quad r_2 \;=\; a\!+\!\varepsilon x
\end{align}

By the entry conjugate diameters of ellipse, the slope of the tangent of the ellipse in the point \,$(x_0,\,y_0)$\, is
\begin{align}
m_t \;=\; -\frac{b^2x_0}{a^2y_0}.
\end{align}
This result is easily obtained also by implicit differentiation of (3):
$$\frac{2x}{a^2}+\frac{2yy'}{b^2} \;=\; 0$$
Substituting here\, $x = x_0$,\; $y = y_0$\, and solving\, $m_t = y'$,\, one gets (5).
Then we can write the equation of the normal line of the ellipse (3) in\, $(x_0,\,y_0)$:
$$y-y_0 \;=\; \frac{a^2y_0}{b^2x_0}(x-x_0)$$
The normal cuts the $x$-axis in the point $N$ with
$$x \;=\; \frac{a^2-b^2}{a^2}x_0 \;=\; \frac{c^2}{a^2}x_0 \;=\; \varepsilon^2x_0.$$
Thus the distances of $N$ from the foci are by (4) equal to
$$d_1 \;=\; c-\varepsilon^2x_0 = a\varepsilon-\varepsilon^2x_0 = \varepsilon(a-\varepsilon x_0) = \varepsilon r_1,$$
$$d_2 \;=\; c+\varepsilon^2x_0 = a\varepsilon+\varepsilon^2x_0 = \varepsilon(a+\varepsilon x_0) = \varepsilon r_2.$$
Consequently,\, $d_1\!:\!d_2 \,=\, r_1\!:\!r_2$,\, which means by the bisectors theorem that the normal is the angle bisector of the angle between $r_1$ and $r_2$.\\

Accordingly, we have derived the

\textbf{Theorem 1.}\, The normal of ellipse bisects the angle between the focal radii.\, This may also be stated so that the tangent of the ellipse forms equal angles with both focal radii.\\

\textbf{Corollary.}\, If the circumference of an ellipse reflects the light rays meeting it, then the rays emanating from one of the foci meet after the reflection in the other focus.\\

One can prove also the

\textbf{Theorem 2.}\, The locus of the intersection point of a normal of an ellipse and the line through a focus and perpendicular to the normal is the circumscribed circle of the ellipse.

\begin{center}
\begin{pspicture}(-4.7,-4)(4.7,4)
\psaxes[Dx=9,Dy=9]{->}(0,0)(-4.2,-3.7)(4.5,3.8)
\rput(-4.7,-4){.}
\rput(4.7,4){.}
\rput[b](4.63,0.1){$x$}
\rput[r](0.15,4){$y$}
\psellipse[linecolor=blue](0,0)(3.5,2)
\pscircle(0,0){3.5}
\psdots[linecolor=red](-2.87,0)(2.87,0)
\psdot[linecolor=blue](1.5,1.81)
\psline(0.3,2.15)(3.8,1.2)
\psline(2.87,0)(3.2,1.42)
\psdot[linecolor=green](3.2,1.42)
\psline(1.5,1.81)(0.9,-0.41)
\rput(3.62,-0.2){$a$}
\rput(2.87,-0.25){$c$}
\rput(-3,-0.25){$-c$}
\rput(-0.15,2.18){$b$}
\end{pspicture}
\end{center}

%%%%%
%%%%%
\end{document}
