\documentclass[12pt]{article}
\usepackage{pmmeta}
\pmcanonicalname{RelatedRates}
\pmcreated{2013-03-22 12:44:59}
\pmmodified{2013-03-22 12:44:59}
\pmowner{rmilson}{146}
\pmmodifier{rmilson}{146}
\pmtitle{related rates}
\pmrecord{6}{33054}
\pmprivacy{1}
\pmauthor{rmilson}{146}
\pmtype{Definition}
\pmcomment{trigger rebuild}
\pmclassification{msc}{53A45}
\pmclassification{msc}{53A17}
\pmclassification{msc}{26A24}
\pmrelated{Derivative2}

\newcommand{\bx}{\mathbf{x}}
\newcommand{\by}{\mathbf{y}}
\newcommand{\rT}{\mathrm{T}}


\usepackage{amsmath}
\usepackage{amsfonts}
\usepackage{amssymb}
\newcommand{\reals}{\mathbb{R}}
\newcommand{\natnums}{\mathbb{N}}
\newcommand{\cnums}{\mathbb{C}}
\newcommand{\znums}{\mathbb{Z}}
\newcommand{\lp}{\left(}
\newcommand{\rp}{\right)}
\newcommand{\lb}{\left[}
\newcommand{\rb}{\right]}
\newcommand{\supth}{^{\text{th}}}
\newtheorem{proposition}{Proposition}
\newtheorem{definition}[proposition]{Definition}
\newcommand{\nl}[1]{\PMlinkescapetext{{#1}}}
\newcommand{\pln}[2]{\PMlinkname{{#1}}{#2}}
\begin{document}
The notion of a derivative has numerous interpretations and
applications.  A well-known geometric interpretation is that of a
slope, or more generally that of a linear approximation to a mapping
between linear spaces (see \PMlinkid{here}{2975}).  Another useful
interpretation comes from physics and is based on the idea of related
rates.  This second point of view is quite general, and sheds light on
the definition of the derivative of a manifold mapping (the latter is
described in the pushforward entry).

Consider two physical quantities $x$ and $y$ that are somehow coupled.
For example:
\begin{itemize}
\item the quantities $x$ and $y$ could be the coordinates of a point as it
  moves along the unit circle; 
\item the quantity $x$ could be the radius of a sphere
  and $y$ the sphere's surface area; 
\item the quantity $x$ could be the horizontal position of a point on
  a given curve and $y$ the distance traversed by that point as it
  moves from some fixed starting position;
\item the quantity $x$ could be depth of water in a conical tank and
  $y$ the rate at which the water flows out the bottom.
\end{itemize}
Regardless of the application, the situation is such that a change in
the value of one quantity is accompanied by a change in the value of
the other quantity.  So let's imagine that we take control of one of
the quantities, say $x$, and change it in any way we like.  As we do
so, quantity $y$ follows suit and changes along with $x$.  Now the
analytical relation between the values of $x$ and $y$ could be quite
complicated and non-linear, but the relation between the instantaneous
rates of change of $x$ and $y$ {\em is linear}. 

It does not matter how we vary the two quantities, the ratio of the
rates of change depends only on the values of $x$ and $y$.  This ratio
is, of course, the derivative of the function that maps the values of
$x$ to the values of $y$.  Letting $\dot{x}, \dot{y}$ denote the rates
of change of the two quantities, we describe this conception of the
derivative as
$$  \frac{dy}{dx} = \frac{\dot{y}}{\dot{x}},$$
or equivalently as
\begin{equation}
  \label{eq:deriv}
  \dot{y} = \frac{dy}{dx}\,  \dot{x}.
\end{equation}

Next, let us generalize the discussion and suppose that the two
quantities $\bx$ and $\by$ represent physical states with multiple
degrees of freedom.  For example, $\bx$ could be a point on the
earth's surface, and $\by$ the position of a point 1 kilometer to the
north of $\bx$.  Again, the dependence of $\by$ and $\bx$ is, in
general, non-linear, but the rate of change of $\by$ does have a
linear dependence on the rate of change of $\bx$.  We would like to
say that the derivative is precisely this linear relation, but we must
first contend with the following complication. The rates of change are
no longer scalars, but rather velocity vectors, and therefore the
derivative must be regarded as a linear transformation that changes
one vector into another.

In order to formalize this generalized notion of the derivative we
must consider $\bx$ and $\by$ to be points on manifolds $X$ and $Y$,
and the relation between them a manifold mapping $\phi:X\rightarrow
Y$.  A varying $\bx$ is formally described by a parameterized curve
$$\gamma:I\rightarrow X,\quad I\subset\reals.$$
The corresponding
velocities take their value in the tangent spaces of $X$:
$$\gamma'(t) \in \rT_{\gamma(t)} X.$$
The ``coupling'' of the two quantities is described by the composition
$$\phi\circ\gamma:I\rightarrow Y.$$
The derivative of $\phi$ at any given $\bx\in X$ is a linear mapping
$$\phi_*(\bx): \rT_{\bx} X \rightarrow \rT_{\phi(\bx)} Y,$$
called the
pushforward of $\phi$ at $\bx$, with the property that for every
trajectory $\gamma$ passing through $\bx$ at time $t$, we have
$$(\phi\circ\gamma)'(t) = \phi_*(\bx)\gamma'(t).$$
The above is the multi-dimensional and coordinate-free generalization
of the related rates relation \eqref{eq:deriv}.

All of the above has a perfectly rigorous presentation in terms of
manifold theory.  The approach of the present entry is more informal;
our ambition was merely to motivate the notion of a derivative by
describing it as a linear transformation between velocity vectors.
%%%%%
%%%%%
\end{document}
