\documentclass[12pt]{article}
\usepackage{pmmeta}
\pmcanonicalname{Sinusoid}
\pmcreated{2015-02-04 11:23:26}
\pmmodified{2015-02-04 11:23:26}
\pmowner{matte}{1858}
\pmmodifier{pahio}{2872}
\pmtitle{sinusoid}
\pmrecord{12}{37101}
\pmprivacy{1}
\pmauthor{matte}{2872}
\pmtype{Definition}
\pmcomment{trigger rebuild}
\pmclassification{msc}{53A04}
\pmrelated{Trigonometry}
\pmrelated{DefinitionsInTrigonometry}

\endmetadata

% this is the default PlanetMath preamble.  as your knowledge
% of TeX increases, you will probably want to edit this, but
% it should be fine as is for beginners.

% almost certainly you want these
\usepackage{amssymb}
\usepackage{amsmath}
\usepackage{amsfonts}
\usepackage{amsthm}

\usepackage{mathrsfs}

% used for TeXing text within eps files
%\usepackage{psfrag}
% need this for including graphics (\includegraphics)
\usepackage{graphicx}
% for neatly defining theorems and propositions
%
% making logically defined graphics
%%%\usepackage{xypic}

% there are many more packages, add them here as you need them

% define commands here

\newcommand{\sR}[0]{\mathbb{R}}
\newcommand{\sC}[0]{\mathbb{C}}
\newcommand{\sN}[0]{\mathbb{N}}
\newcommand{\sZ}[0]{\mathbb{Z}}

 \usepackage{bbm}
 \newcommand{\Z}{\mathbbmss{Z}}
 \newcommand{\C}{\mathbbmss{C}}
 \newcommand{\F}{\mathbbmss{F}}
 \newcommand{\R}{\mathbbmss{R}}
 \newcommand{\Q}{\mathbbmss{Q}}



\newcommand*{\norm}[1]{\lVert #1 \rVert}
\newcommand*{\abs}[1]{| #1 |}



\newtheorem{thm}{Theorem}
\newtheorem{defn}{Definition}
\newtheorem{prop}{Proposition}
\newtheorem{lemma}{Lemma}
\newtheorem{cor}{Corollary}
\begin{document}
A \emph{sinusoid} is a curve of the form
\begin{eqnarray*}
  \mathbb{R}&\to& \mathbb{R}^2 \\
   t&\mapsto & (t,\,\sin{kt}),
\end{eqnarray*}
where $k>0$ is a parameter determining the oscillation.

The basic sinusoid, the curve 
          $$y  = \sin{x}$$
in the $xy$-plane, oscillates periodically with the \PMlinkname{period of sine}{ComplexSineAndCosine}, $2\pi$, as $x$ increases.
\begin{center}
\includegraphics{sinusoid}
\end{center}
\begin{itemize}
\item On the interval \,$[0,\,\frac{\pi}{2}]$,\, the curve is ascending because the \PMlinkname{derivative of sine}{ComplexSineAndCosine}, $\cos{x}$, is positive for acute angles $x$. 
\item Consequently, on the interval \,$[\frac{\pi}{2},\,\pi]$,\,the supplement formula \,$\sin{(\pi-x)} = \sin{x}$\, tells that the sinusoid is descending.
\item Thus we get on the whole interval \,$[0,\,\pi]$\, a cap-formed ($\smallfrown$) arc. 
\item Because \PMlinkname{sine is an odd function}{ComplexSineAndCosine}, we have on the interval \,$[-\pi,\,0]$\, the \PMlinkescapetext{mirror image} of the cap, a cup-formed ($\smallsmile$) arc.
\item All in all, on the period interval \,$[-\pi,\,\pi]$\, the sinusoid consists of the consecutive cup and cap, together a lying-S formed ($\backsim$) arc. 
\item The same is repeated on each other period interval \,$[(2n\!-\!1)\pi,\,(2n\!+\!1)\pi]$\, where \,$n\in\mathbb{Z}$.
\end{itemize}
%%%%%
%%%%%
\end{document}
