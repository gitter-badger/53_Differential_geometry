\documentclass[12pt]{article}
\usepackage{pmmeta}
\pmcanonicalname{TotalDifferential}
\pmcreated{2013-03-22 19:11:24}
\pmmodified{2013-03-22 19:11:24}
\pmowner{pahio}{2872}
\pmmodifier{pahio}{2872}
\pmtitle{total differential}
\pmrecord{10}{42101}
\pmprivacy{1}
\pmauthor{pahio}{2872}
\pmtype{Definition}
\pmcomment{trigger rebuild}
\pmclassification{msc}{53A04}
\pmclassification{msc}{26B05}
\pmclassification{msc}{01A45}
\pmsynonym{exact differential}{TotalDifferential}
\pmsynonym{differential}{TotalDifferential}
\pmrelated{ExactDifferentialForm}
\pmrelated{ExactDifferentialEquation}
\pmrelated{Differential}
\pmrelated{DifferntiableFunction}
\pmdefines{differentiable}

% this is the default PlanetMath preamble.  as your knowledge
% of TeX increases, you will probably want to edit this, but
% it should be fine as is for beginners.

% almost certainly you want these
\usepackage{amssymb}
\usepackage{amsmath}
\usepackage{amsfonts}

% used for TeXing text within eps files
%\usepackage{psfrag}
% need this for including graphics (\includegraphics)
%\usepackage{graphicx}
% for neatly defining theorems and propositions
 \usepackage{amsthm}
% making logically defined graphics
%%%\usepackage{xypic}

% there are many more packages, add them here as you need them

% define commands here

\theoremstyle{definition}
\newtheorem*{thmplain}{Theorem}

\begin{document}
\PMlinkescapeword{consistent}  \PMlinkescapeword{independent}


There is the generalisation of the theorem in \PMlinkname{the parent entry}{Differential} concerning the real functions of several variables; here we formulate it for three variables:\\

\textbf{Theorem.}\, Suppose that $S$ is a ball in $\mathbb{R}^3$, the function \,$f\!:S\to \mathbb{R}$\, is continuous and has partial derivatives $f_x',\,f_y',\,f_z'$ in $S$ and the partial derivatives are continuous in a point 
\,$(x,\,y,\,z)$\, of $S$.\, Then the increment
$$\Delta f \;:=\; f(x\!+\!\Delta x,\,y\!+\!\Delta y,\,z\!+\!\Delta z)-f(x,\,y,\,z),$$
which $f$ gets when one moves from\, $(x,\,y,\,z)$\, to another point 
\,$(x\!+\!\Delta x,\,y\!+\!\Delta y,\,z\!+\!\Delta z)$\, of $S$, can be split into two parts as follows:
\begin{align}
\Delta f \;=\; [f_x'(x,\,y,\,z)\Delta x+f_y'(x,\,y,\,z)\Delta y+f_z'(x,\,y,\,z)\Delta z]+\langle\varrho\rangle\varrho.
\end{align}
Here,\, $\varrho := \sqrt{\Delta x^2\!+\!\Delta y^2\!+\!\Delta z^2}$\, and $\langle\varrho\rangle$ is a quantity tending to 0 along with $\varrho$.\\


The former part of $\Delta x$ is called the (\emph{total}) \emph{differential} or the \emph{exact differential} of the function $f$ in the point \,$(x,\,y,\,z)$\, and it is denoted by\, $df(x,\,y,\,z)$\, of briefly $df$.\, In the special case \,$f(x,\,y,\,z) \equiv x$,\, we see that\, $df = \Delta x$\, and thus\, $\Delta x = dx$;\, similarly\, 
$\Delta y = dy$\, and $\Delta z = dz$.\, Accordingly, we obtain for the general case the more consistent notation
\begin{align}
df \;=\; f_x'(x,\,y,\,z)dx+f_y'(x,\,y,\,z)dy+f_z'(x,\,y,\,z)dz,
\end{align}
where $dx,\,dy,\,dz$ may be thought as independent variables.\\


We now assume conversely that the increment of a function $f$ in $\mathbb{R}^3$ can be split into two parts as follows:
\begin{align}
f(x\!+\!\Delta x,\,y\!+\!\Delta y,\,z\!+\!\Delta z)-f(x,\,y,\,z) 
\;=\; [A\Delta x+B\Delta y+C\Delta z]+\langle\varrho\rangle\varrho
\end{align}
where the coefficients $A,\,B,\,C$ are independent on the quantities $\Delta x,\,\Delta y,\,\Delta z$ and 
$\varrho,\,\langle\varrho\rangle$ are as in the above theorem.\, Then one can infer that the partial derivatives 
$f_x',\,f_y',\,f_z'$ exist in the point\, $(x,\,y,\,z)$\, and have the values $A,\,B,\,C$, respectively.\, In fact, if we choose\, $\Delta y = \Delta z = 0$, then\, $\varrho = |\Delta x|$\, whence (3) attains the form
$$f(x\!+\!\Delta x,\,y\!+\!\Delta y,\,z\!+\!\Delta z)-f(x,\,y,\,z) \,=\, A\Delta x+\langle\Delta x\rangle\Delta x$$
and therefore 
$$A \;=\; \lim_{\Delta x \to 0}\frac{f(x\!+\!\Delta x,\,y\!+\!\Delta y,\,z\!+\!\Delta z)-f(x,\,y,\,z)}{\Delta x}
\;=\; f_x'(x,\,y,\,z).$$
Similarly we see the values of $f_y'$ and $f_z'$.

The last consideration showed the uniqueness of the total differential.\\

\textbf{Definition.}\, A function $f$ in $\mathbb{R}^3$, satisfying the conditions of the above theorem is said to be \emph{differentiable} in the point\, $(x,\,y,\,z)$.\\


\textbf{Remark.}\, The differentiability of a function $f$ of two variables in the point\, $(x,\,y)$\, means that the surface \;$z \,=\, f(x,\,y)$\; has a tangent plane in this point.

\begin{thebibliography}{8}
\bibitem{lindelof}{\sc Ernst Lindel\"of}: {\em Differentiali- ja integralilasku
ja sen sovellutukset II}.\, Mercatorin Kirjapaino Osakeyhti\"o, Helsinki (1932).
\end{thebibliography} 

%%%%%
%%%%%
\end{document}
