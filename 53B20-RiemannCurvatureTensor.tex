\documentclass[12pt]{article}
\usepackage{pmmeta}
\pmcanonicalname{RiemannCurvatureTensor}
\pmcreated{2013-03-22 16:26:17}
\pmmodified{2013-03-22 16:26:17}
\pmowner{juanman}{12619}
\pmmodifier{juanman}{12619}
\pmtitle{Riemann curvature tensor}
\pmrecord{10}{38592}
\pmprivacy{1}
\pmauthor{juanman}{12619}
\pmtype{Definition}
\pmcomment{trigger rebuild}
\pmclassification{msc}{53B20}
\pmclassification{msc}{53A55}
\pmrelated{Curvature}
\pmrelated{Connection}
\pmrelated{FormalLogicsAndMetaMathematics}

\endmetadata

% this is the default PlanetMath preamble.  as your knowledge
% of TeX increases, you will probably want to edit this, but
% it should be fine as is for beginners.

% almost certainly you want these
\usepackage{amssymb}
\usepackage{amsmath}
\usepackage{amsfonts}

% used for TeXing text within eps files
%\usepackage{psfrag}
% need this for including graphics (\includegraphics)
%\usepackage{graphicx}
% for neatly defining theorems and propositions
%\usepackage{amsthm}
% making logically defined graphics
%%%\usepackage{xypic}

% there are many more packages, add them here as you need them

% define commands here

\begin{document}
Let 
$\mathcal{X}$ denote the vector space of smooth vector fields on a
smooth Riemannian manifold $(M,g)$.  Note that $\mathcal{X}$ is actually a
$\mathcal{C}^\infty(M)$ module because we can multiply a vector field
by a function to obtain another vector field.
The \emph{Riemann curvature tensor} is the tri-linear
$\mathcal{C}^\infty$ mapping  
$$R:{\mathcal{X}}\times{\mathcal{X}}\times{\mathcal{X}}\to{\mathcal{X}},$$ 
which is defined by
$$R(X,Y)Z=\nabla_X\nabla_YZ-\nabla_Y\nabla_XZ-\nabla_{[X,Y]}Z$$
where $X,Y,Z\in\mathcal{X}$ are vector fields,  where $\nabla$ is
the Levi-Civita connection attached to the metric tensor $g$, and
where the square brackets denote the Lie bracket of two vector fields.
The tri-linearity means that for every smooth $f\colon M\to\mathbb{R}$
we have
$$fR(X,Y)Z=R(fX,Y)Z=R(X,fY)Z=R(X,Y)fZ.$$

In components this tensor is classically denoted by a set of
four-indexed components ${R^i}_{jkl}$. This means that given a
basis of linearly independent vector fields $X_i$ we have
$$R(X_j,X_k)X_l=\sum_s {R^s}_{jkl}X_s.$$

In a two dimensional manifold it is known that the Gaussian curvature
it is given by
$$K_g=\frac{R_{1212}}{g_{11}g_{22}-{g_{12}}^2}$$
%%%%%
%%%%%
\end{document}
