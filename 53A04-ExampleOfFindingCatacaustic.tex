\documentclass[12pt]{article}
\usepackage{pmmeta}
\pmcanonicalname{ExampleOfFindingCatacaustic}
\pmcreated{2013-03-22 18:52:59}
\pmmodified{2013-03-22 18:52:59}
\pmowner{pahio}{2872}
\pmmodifier{pahio}{2872}
\pmtitle{example of finding catacaustic}
\pmrecord{17}{41731}
\pmprivacy{1}
\pmauthor{pahio}{2872}
\pmtype{Example}
\pmcomment{trigger rebuild}
\pmclassification{msc}{53A04}
\pmclassification{msc}{51N20}
\pmclassification{msc}{26B05}
\pmclassification{msc}{26A24}
\pmsynonym{example of catacaustic}{ExampleOfFindingCatacaustic}
\pmsynonym{catacaustic of exponential curve}{ExampleOfFindingCatacaustic}
\pmrelated{Catacaustic}
\pmrelated{ConditionOfOrthogonality}

% this is the default PlanetMath preamble.  as your knowledge
% of TeX increases, you will probably want to edit this, but
% it should be fine as is for beginners.

% almost certainly you want these
\usepackage{amssymb}
\usepackage{amsmath}
\usepackage{amsfonts}

% used for TeXing text within eps files
%\usepackage{psfrag}
% need this for including graphics (\includegraphics)
%\usepackage{graphicx}
% for neatly defining theorems and propositions
 \usepackage{amsthm}
% making logically defined graphics
%%%\usepackage{xypic}
\usepackage{pstricks}
\usepackage{pst-plot}

% there are many more packages, add them here as you need them

% define commands here

\theoremstyle{definition}
\newtheorem*{thmplain}{Theorem}

\begin{document}
Let us find the catacaustic when the family of vertical rays \,$x = t$\, coming from above \PMlinkname{reflects}{HeronsPrinciple} from the curve \,$y = e^x$.\\

The line \,$x = t$\, meets the curve in the point \,$(t,\,e^t)$\, where the slope angle $\alpha$ of the normal line of the curve satisfies
$$\tan\alpha = -e^{-t}$$
and the slope angle of the reflected ray is $2\alpha\!+\!\frac{\pi}{2}$; so its slope is
$$\tan(2\alpha\!+\!\frac{\pi}{2}) \;=\; -\frac{1}{\tan2\alpha} \;=\; \frac{\tan^2\alpha\!-\!1}{2\tan{\alpha}} \;=\;
\frac{e^{-2t}-1}{-2e^{-t}} \;=\; \frac{e^t-e^{-t}}{2} \;=\; \sinh{t}$$
(N.B.\, $\tan(2\alpha\!+\!\frac{\pi}{2})$ cannot be determined with the addition formula of \PMlinkname{tangent}{Trigonometry}, but by using the complement formula:\, 
$\tan(2\alpha\!+\!\frac{\pi}{2}) = \tan(\frac{\pi}{2}-(-2\alpha)) = \cot(-2\alpha) = -\frac{1}{\tan2\alpha}$.)
Thus the equation of the reflected ray is
$$y-e^t \;=\; (\sinh{t})(x-t),$$
i.e.
\begin{align}
F(x,\,y,\,t) \;:=\; (x-t)\sinh{t}-y+e^t \;=\; 0.
\end{align}
We have the partial derivative
\begin{align}
F'_t(x,\,y,\,t) \;=\; -\sinh{t}+(x-t)\cosh{t}+e^t.
\end{align}
Then the points of the catacaustic, i.e. the envelope of all reflected rays satisfy (see determining envelope)
\begin{align}
\begin{cases}
F(x,\,y,\,t) \;=\; 0,\\ 
F'_t(x,\,y,\,t) \;=\; 0.
\end{cases}
\end{align}
For eliminating the parametre $t$ from (3) solve first its latter equation for $x\!-\!t$:
\begin{align}
x\!-\!t \;=\; \frac{\sinh{t}-e^t}{\cosh{t}} \;\equiv\; -1
\end{align}
By (1) and (4), we obtain 
$$y \;=\; e^t+(x\!-\!t)\sinh{t} \;=\; e^t-\sinh{t} \;\equiv\; \cosh{t},$$
whence the equation of the catacaustic reads simply
\begin{align}
y \;=\; \cosh(x\!+\!1).
\end{align}
This \PMlinkescapetext{represents} a catenary with the \PMlinkescapetext{apex} in the point\, $(-1,\,1)$ .

\begin{center}
\begin{pspicture}(-5,-2)(3.5,7)
\psaxes[Dx=10,Dy=10]{->}(0,0)(-3.5,-0.5)(3.5,6.5)
\rput(3.6,-0.2){$x$}
\rput(0.2,6.6){$y$}
\psplot{-3}{1.9}{2.71828 x exp}
\psplot[linecolor=red]{-3.5}{1.5}{2.71828 x 1 add exp 2.71828 x 1 add neg exp add 2 div}
\psline[linecolor=blue](1,2.71828)(1,6)
\psline[arrows=->,linecolor=blue](1,2.71828)(-0.5,0.88)
\psline[linecolor=blue](-0.7,0.49659)(-0.7,6)
\psline[arrows=->,linecolor=blue](-0.7,0.49659)(-3.2,2.38)
\rput(-3.7,5.5){$y = \cosh(x\!+\!1)$}
\rput(2.4, 5.5){$y = e^x$}
\rput(-1,-1){Two vertical rays (blue) reclecting from the exponential curve}
\end{pspicture}
\end{center}

%%%%%
%%%%%
\end{document}
