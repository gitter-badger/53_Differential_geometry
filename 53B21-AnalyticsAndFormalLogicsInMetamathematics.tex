\documentclass[12pt]{article}
\usepackage{pmmeta}
\pmcanonicalname{AnalyticsAndFormalLogicsInMetamathematics}
\pmcreated{2013-03-22 18:24:21}
\pmmodified{2013-03-22 18:24:21}
\pmowner{bci1}{20947}
\pmmodifier{bci1}{20947}
\pmtitle{analytics and formal logics in meta-mathematics}
\pmrecord{46}{41054}
\pmprivacy{1}
\pmauthor{bci1}{20947}
\pmtype{Topic}
\pmcomment{trigger rebuild}
\pmclassification{msc}{53B21}
\pmclassification{msc}{03G30}
\pmclassification{msc}{03G20}
\pmclassification{msc}{03B22}
\pmclassification{msc}{01A60}
\pmclassification{msc}{18-00}
\pmclassification{msc}{03G30}
\pmclassification{msc}{46M15}
\pmclassification{msc}{03B22}
\pmclassification{msc}{03B50}
\pmclassification{msc}{03B15}
\pmsynonym{meta-logic}{AnalyticsAndFormalLogicsInMetamathematics}
\pmsynonym{metamathematics}{AnalyticsAndFormalLogicsInMetamathematics}
\pmsynonym{mathematical foundations}{AnalyticsAndFormalLogicsInMetamathematics}
\pmsynonym{axiomatics and formal logics}{AnalyticsAndFormalLogicsInMetamathematics}
%\pmkeywords{meta-logic}
%\pmkeywords{metamathematics}
%\pmkeywords{mathematical foundations}
%\pmkeywords{axiomatics and formal logics}
%\pmkeywords{meta-mathematics}
\pmrelated{Supercategory}
\pmrelated{JanLukasiewicz}
\pmrelated{Logicism}
\pmrelated{FOUNDATIONSOFMATHEMATICSOVERVIEW}
\pmrelated{AnOutlineOfHilbertsProgramme}
\pmrelated{SupercategoriesOfComplexSystems}
\pmrelated{BibliographyOfManyValuedLogicsAndApplications}
\pmrelated{PlatosMathematics}
\pmrelated{AnOutlineOfHilbertsProgramme}
\pmrelated{RiemannCurvatureTensor}
\pmrelated{GWLe}
\pmdefines{meta-theory}
\pmdefines{metatheory}
\pmdefines{metalogic}
\pmdefines{metamathematics}
\pmdefines{meta-theorems}
\pmdefines{metatheorems}

\endmetadata

% this is the default PlanetMath preamble. 

\usepackage{amssymb}
\usepackage{amsmath}
\usepackage{amsfonts}

% define commands here
\usepackage{amsmath, amssymb, amsfonts, amsthm, amscd, latexsym}
%%\usepackage{xypic}
\usepackage[mathscr]{eucal}

\setlength{\textwidth}{6.5in}
%\setlength{\textwidth}{16cm}
\setlength{\textheight}{9.0in}
%\setlength{\textheight}{24cm}

\hoffset=-.75in     %%ps format
%\hoffset=-1.0in     %%hp format
\voffset=-.4in

\theoremstyle{plain}
\newtheorem{lemma}{Lemma}[section]
\newtheorem{proposition}{Proposition}[section]
\newtheorem{theorem}{Theorem}[section]
\newtheorem{corollary}{Corollary}[section]

\theoremstyle{definition}
\newtheorem{definition}{Definition}[section]
\newtheorem{example}{Example}[section]
%\theoremstyle{remark}
\newtheorem{remark}{Remark}[section]
\newtheorem*{notation}{Notation}
\newtheorem*{claim}{Claim}

\renewcommand{\thefootnote}{\ensuremath{\fnsymbol{footnote%%@
}}}
\numberwithin{equation}{section}

\newcommand{\Ad}{{\rm Ad}}
\newcommand{\Aut}{{\rm Aut}}
\newcommand{\Cl}{{\rm Cl}}
\newcommand{\Co}{{\rm Co}}
\newcommand{\DES}{{\rm DES}}
\newcommand{\Diff}{{\rm Diff}}
\newcommand{\Dom}{{\rm Dom}}
\newcommand{\Hol}{{\rm Hol}}
\newcommand{\Mon}{{\rm Mon}}
\newcommand{\Hom}{{\rm Hom}}
\newcommand{\Ker}{{\rm Ker}}
\newcommand{\Ind}{{\rm Ind}}
\newcommand{\IM}{{\rm Im}}
\newcommand{\Is}{{\rm Is}}
\newcommand{\ID}{{\rm id}}
\newcommand{\GL}{{\rm GL}}
\newcommand{\Iso}{{\rm Iso}}
\newcommand{\Sem}{{\rm Sem}}
\newcommand{\St}{{\rm St}}
\newcommand{\Sym}{{\rm Sym}}
\newcommand{\SU}{{\rm SU}}
\newcommand{\Tor}{{\rm Tor}}
\newcommand{\U}{{\rm U}}

\newcommand{\A}{\mathcal A}
\newcommand{\Ce}{\mathcal C}
\newcommand{\D}{\mathcal D}
\newcommand{\E}{\mathcal E}
\newcommand{\F}{\mathcal F}
\newcommand{\G}{\mathcal G}
\newcommand{\Q}{\mathcal Q}
\newcommand{\R}{\mathcal R}
\newcommand{\cS}{\mathcal S}
\newcommand{\cU}{\mathcal U}
\newcommand{\W}{\mathcal W}

\newcommand{\bA}{\mathbb{A}}
\newcommand{\bB}{\mathbb{B}}
\newcommand{\bC}{\mathbb{C}}
\newcommand{\bD}{\mathbb{D}}
\newcommand{\bE}{\mathbb{E}}
\newcommand{\bF}{\mathbb{F}}
\newcommand{\bG}{\mathbb{G}}
\newcommand{\bK}{\mathbb{K}}
\newcommand{\bM}{\mathbb{M}}
\newcommand{\bN}{\mathbb{N}}
\newcommand{\bO}{\mathbb{O}}
\newcommand{\bP}{\mathbb{P}}
\newcommand{\bR}{\mathbb{R}}
\newcommand{\bV}{\mathbb{V}}
\newcommand{\bZ}{\mathbb{Z}}

\newcommand{\bfE}{\mathbf{E}}
\newcommand{\bfX}{\mathbf{X}}
\newcommand{\bfY}{\mathbf{Y}}
\newcommand{\bfZ}{\mathbf{Z}}

\renewcommand{\O}{\Omega}
\renewcommand{\o}{\omega}
\newcommand{\vp}{\varphi}
\newcommand{\vep}{\varepsilon}

\newcommand{\diag}{{\rm diag}}
\newcommand{\grp}{{\mathbb G}}
\newcommand{\dgrp}{{\mathbb D}}
\newcommand{\desp}{{\mathbb D^{\rm{es}}}}
\newcommand{\Geod}{{\rm Geod}}
\newcommand{\geod}{{\rm geod}}
\newcommand{\hgr}{{\mathbb H}}
\newcommand{\mgr}{{\mathbb M}}
\newcommand{\ob}{{\rm Ob}}
\newcommand{\obg}{{\rm Ob(\mathbb G)}}
\newcommand{\obgp}{{\rm Ob(\mathbb G')}}
\newcommand{\obh}{{\rm Ob(\mathbb H)}}
\newcommand{\Osmooth}{{\Omega^{\infty}(X,*)}}
\newcommand{\ghomotop}{{\rho_2^{\square}}}
\newcommand{\gcalp}{{\mathbb G(\mathcal P)}}

\newcommand{\rf}{{R_{\mathcal F}}}
\newcommand{\glob}{{\rm glob}}
\newcommand{\loc}{{\rm loc}}
\newcommand{\TOP}{{\rm TOP}}

\newcommand{\wti}{\widetilde}
\newcommand{\what}{\widehat}

\renewcommand{\a}{\alpha}
\newcommand{\be}{\beta}
\newcommand{\ga}{\gamma}
\newcommand{\Ga}{\Gamma}
\newcommand{\de}{\delta}
\newcommand{\del}{\partial}
\newcommand{\ka}{\kappa}
\newcommand{\si}{\sigma}
\newcommand{\ta}{\tau}
\newcommand{\lra}{{\longrightarrow}}
\newcommand{\ra}{{\rightarrow}}
\newcommand{\rat}{{\rightarrowtail}}
\newcommand{\oset}[1]{\overset {#1}{\ra}}
\newcommand{\osetl}[1]{\overset {#1}{\lra}}
\newcommand{\hr}{{\hookrightarrow}}

\begin{document}
This is a topic on meta-theories, metalogic and metamathematics founded in formal logic.

\subsection{Introduction: Formal Logic, Meta-Logic and Meta-Mathematics}

A {\em methatheory} or {\em meta-theory} can be described as a theory about theories belonging to a theory class $\mathbb{M}$.
With this meaning, a theory $\mathcal{T}$ of the domain $\mathcal{D}$ is a meta-theory if $\mathcal{D}$ is a theory belonging to a class
$\mathbb{M}$ of (lower-level, or first level) theories. A general theory is not a meta-theory because its domain $\mathcal{D}$ does not contain any other theories. Valid statements made in a meta-theory are called {\em meta-theorems} or {\em metatheorems}.


A {\em metalogic} is then a meta-theory of various types of logic.

{\em Meta-mathematics} is concerned with the study of metatheories containing mathematical metatheorems.

As an example of a metatheory, the theory of super-categories $\mathcal{S}$ is concerned with metatheorems about categories of categories. On the other hand, an example of a metatheory of supercategories $\S$, such as organismic supercategories $OS$, is the metatheory of the higher dimensional supercategory of supercategories. Higher dimensional
algebra (HDA) is a metatheory of algebraic categories and other algebraic structures; good examples are double groupoids, double algebroids and their categories, as well as double categories. Further specific examples of HDA
are 2-Lie groups and 2-Lie algebras, as well as their categories of 2-Lie groups and 2-Lie algebras.

In the perspective of the development of mathematics, advances in logic --and over the last century in logics and meta-logics -- have played, and are playing, very important roles both in the foundations of mathematics, as well as in related areas such as: categorical logics, many-valued logic algebras, model theory and many specific fields of mathematics including, but not limited to, number theory/arithmetics. The following is only a brief outline of the connection between Husserl's `Formal Logics' or `Analytics', model theory and the long-debated logical foundations of number theory.

\subsection{On the Logical Foundations for Arithmetic}
A real argument occurred between Husserl and \PMlinkname{Frege}{Logicism} over the possibility of employing formal logic to completely formalize arithmetic in mathematics. Husserl's negation of such a possibility seems to have been completely validated by subsequent developments, as for example by $G\"odel$'s theorem.

``The culmination of the new approach to logic lay in its capacity to illuminate the nature of the mathematical reasoning. While the idealists sought to reveal the internal coherence of absolute reality and the pragmatists offered to account for human inquiry as a loose pattern of investigation, the new logicians hoped to show that the most significant relations among things could be understood as `purely formal and external'. Mathematicians like Richard Dedekind realized that on this basis it might be possible to establish mathematics firmly on logical grounds. Giuseppe Peano had demonstrated in 1889 that \emph{all of arithmetic could be reduced to an axiomatic system with a carefully restricted set of preliminary postulates}. Frege promptly sought to express these postulates in the
symbolic notation of his own invention. By 1913, Russell and Whitehead had completed the monumental ``Principia Mathematica'' (1913), taking \emph{three massive volume}s to move from a few logical axioms through a definition of number to a proof that ``1 + 1 = 2 .'' Although the work of $G\"odel$ (less than two decades later) made clear the
\emph{inherent limitations of this approach}, \emph{its significance} for our understanding of logic
\emph{and mathematics remains}''.

\subsection{Formal Analytics}

Furthermore, in mathematics, as in the case of analysis --such as in functional analysis and analytical geometry (with the latter sometimes being said to have been initiated by Descartes, or `Cartesius')-- the central concepts are those of functions and variables; one can proceed to define the term `mathematical function' , or `mapping of sets', as in the `N. Bourbaki' school of mathematics, in the terms of formal logic.
Perhaps, Goethe--as a philosopher, rather than a poet-- was the first to react negatively to the `reductionist' (or analytical/analytic) methodology strongly pursued by Descartes, whereas Newton may have been the first to apply it with amazing success in classical physics, including classical mechanics, `celestial mechanics' and optics. The latter fitted well either Occam's razor dictum of the simplest explanation being the winner, or Newton's statement that he \emph{`does not make hypotheses}', although he made several implicit, or hidden, ones, thus giving in to Descartes's `demon of deception'. One must also recognize that at the earlier, beginning stages in natural sciences, mathematics, or any other field of knowledge, one should, and indeed, must make major simplifying assumptions in order to be able to present a comprehensible theory of any kind. The other side of the coin is, however, that once past such an initial stage one must re-consider all `hidden' or implicit incorrect assumptions that were previously made and then remove them from the theory. A good example, is that of Einstein's removal of the concept of an undefinable `ether' from all Physics, as well as the elimination of the concepts of an `absolute space' and `absolute time' as valid physical descriptions of both space and time; he replaced them with the fundamental concept of \emph{space-time}, and proposed at first, in special relativity (SR) theory, that the structure of space-time be formally represented by a \emph{homogeneous, four-dimensional (4D}), mathematical, Minkowski space without
\PMlinkname{curvature}{RiemannCurvatureTensor}. Subsequently, in his general relativity (GR) theory, Einstein retracted that the space-time structure is `flat', but proposed instead that it is curved, with a
\PMlinkname{curvature}{RiemannCurvatureTensor} specified by the \PMlinkname{Riemannian metric tensor}{RiemannianMetric}, as it is in a (continuous) Riemann manifold or space; thus, Einstein's physical, 4D, Riemannian space-time has a curvature that is caused by the presence of both energy fields and massive bodies in the Universe. Along with this fundamental hypothesis about space-time came Einstein's famous equation that relates energy, mass and the speed of light , $E= mc^2$, with the speed of light, $c$, postulated in GR to be an \emph{universal constant} for all electromagnetic radiation, or indeed all form of radiation, and all reference frames; the latter equation was arrived at within Einstein's own theory of photon interactions with matter utilizing Planck's concept of quanta as a description for the light photons.

\subsection{Husserl's `Analytics' and Functorial Meta-Mathematics }
After, this short detour into basic physics and mathematical physics, let us briefly return to the subject of Husserl's \emph{`Analytics'}. Thus, according to Husserl: ``{\em since every science has its own field, scientific knowledge is directed towards a thematic object, and in this case analytics, being a formal doctrine of science, has, as all sciences have, a real direction, and because of its a priori generality, it may also be said to have an ontological direction. It is thus a `Formal Ontology' (\textit{op. cit.}, p. 107). Its a priori truths enunciate what is valid and therefore endowed with formal generality for objects-in-general, for domains of objects in general. They enunciate in what form these objects in general exist or may exist; these enunciations are themselves judgements, for it is in judgements alone that objects-in-general ``exist'' in the form of categories.}'' (Quoted from: Anton Dumitriu - \emph{``History of logic''} - Volume 3, Tubridge Wells, Abacus Press - 1977, pp. 362-366). On the other hand, according to Kant, the synthetic \emph{a priori} judgments are the crucial case, since only they could provide \emph{new information that is necessarily true} (but neither Leibniz nor Hume considered the possibility of any such case).
(See also {\em ``The Ontological Argument, from St. Anselm to Contemporary Philosophers''}, ed. by Alvin Plantinga, Anchor, 1989).

According to the following quote, available from the \PMlinkexternal{website}{http://www.philosophypages.com/hy/6h.htm}:

``George Boole completed this transformation by explicitly interpreting
\PMlinkname{categorical logic}{AlgebraicCategoryOfLMnLogicAlgebras} (as we now do) by reference to classes of things. The logical/set-theoretical/mathematical relations that hold among such classes can be expressed at least as well in a ``Boolean algebra'' as in traditional Aristotelean terms. What is more, as Leonhard Euler and John Venn showed, these relations can be represented perspicuously in purely topographical diagrams whose features model formal validity. All of these developments encouraged philosophers to examine''... the \PMlinkname{functors between the categories of logic and mathematics}{AlgebraicCategoryOfLMnLogicAlgebras} more closely, as indeed should also both modern logicians and mathematicians.

\begin{thebibliography}{99}

\bibitem{CA1956}
Church, A. {\em Introduction to Mathematical Logic}, Princeton,1956.

\bibitem{FG1884}
Frege, G, {\em Grundlagen der Arithmetik}= ``Fundamentals of Arithmetics'', Breslau, 1884.

\bibitem{GLK1944}
$G\"odel$, L, K., Russell's Mathematical Logic, in {\em The Philosophy of Bertrand Russell}, ed. P. Schilpp, The Library of Living Philosophers, 1944.

\bibitem{HE1887}
Husserl, E., \"Uber den Begriff der Zahl. {\em Psychologische Analysen}, 1887.

\bibitem{HE1891}
Husserl, E., Philosophie der Arithmetik. {\em Psychologische und logische Untersuchungen}, 1891.

\bibitem{HE1900}
Husserl, E., Logische Untersuchungen. Erste Teil: Prolegomena zur reinen Logik, 1900; reprinted 1913.

\bibitem{HE1913}
Logische Untersuchungen. Zweite Teil: Untersuchungen zur Ph\"anomenologie und Theorie der Erkenntnis, 1901; second edition 1913 (for part one); second edition 1921 (for part two).

\bibitem{HE1911}
Husserl, E., Philosophie als strenge Wissenschaft, {\em Logos} {\bf 1}, (1911): 289-341.

\bibitem{QW1955}
Quine, W. {\em Mathematical Logic}, Cambridge, MA, 1955.

\bibitem{RB1993}
Russell, B., {\em Introduction to Mathematical Philosophy}, London, 1993.

\bibitem{ICB-RP2k8}
Baianu, I.C. and Roberto Poli. 2008, Categorical Theory of Ontology Levels for Super-- and Ultra-- Complex Systems.,
in {\em Theory and applications of Ontology,}, vol. 2, M. Healy et al. eds., Springer.

\bibitem{Husserl's, translated}
{\bf Husserl's Publications translated into English; publication dates of the German originals are in square
$[~~]$ brackets.}

\bibitem{HE1910a}
Edmund Husserl. 1910. {\em Philosophy as Rigorous Science,} transl. in Q. Lauer (ed.),

\bibitem{HE1910b}
Edmund Husserl. 1910b, {\em Phenomenology and the Crisis of Philosophy}, New York: Harper $[1910]$, 1965.

\bibitem{HE1929}
Edmund Husserl. {\em Formal and Transcendental Logic}, transl. D. Cairns. The Hague: Nijhoff $[1929]$, 1969.

\bibitem{HE1936}
Edmund Husserl. {\em The Crisis of European Sciences and Transcendental Philosophy}, trans. D. Carr. Evanston: Northwestern University Press $[1936/54]$, 1970.

\bibitem{HE1939}
Edmund Husserl. {\em Logical Investigation}, transl. J. N. Findlay, London: Routledge $[1900/01]$; 2nd, revised edition 1913], 1973.
Experience and Judgement, transl. J. S. Churchill and K. Ameriks, London: Routledge [1939], 1973.

\bibitem{HE1913b}
Edmund Husserl. {\em Ideas Pertaining to a Pure Phenomenology and to a Phenomenological Philosophy--Third Book: Phenomenology and the Foundations of the Sciences}, transl. T. E. Klein and W. E. Pohl, Dordrecht: Kluwer, 1980.

\bibitem{HE1913a}
Edmund Husserl. {\em Ideas Pertaining to a Pure Phenomenology and to a Phenomenological Philosophy--First Book: General Introduction to a Pure Phenomenology}, transl. F. Kersten. The Hague: Nijhoff (= Ideas) $[1913]$, 1982.

\bibitem{HE1931}
Edmund Husserl. {\em Cartesian Meditations}, transl. D. Cairns, Dordrecht: Kluwer [1931], 1988.

\bibitem{HE1931}
Edmund Husserl. {\em Ideas Pertaining to a Pure Phenomenology and to a Phenomenological Philosophy - Second Book: Studies in the Phenomenology of Constitution}, transl. R. Rojcewicz and A. Schuwer, Dordrecht: Kluwer, 1989.

\bibitem{HE1928}
Edmund Husserl. {\em On the Phenomenology of the Consciousness of Internal Time (1893-1917)}, transl. J. B. Brough, Dordrecht: Kluwer $[1928]$, 1990.

\bibitem{HE1994}
Edmund Husserl. {\em Early Writings in the Philosophy of Logic and Mathematics.}, transl. D. Willard, Dordrecht: Kluwer, 1994.

\bibitem{WD1999}
D. Welton, editor. {\em The Essential Husserl}, Bloomington: Indiana University Press, 1999.
\end{thebibliography}

%%%%%
%%%%%
\end{document}
