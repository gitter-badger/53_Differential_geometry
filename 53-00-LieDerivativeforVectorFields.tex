\documentclass[12pt]{article}
\usepackage{pmmeta}
\pmcanonicalname{LieDerivativeforVectorFields}
\pmcreated{2013-03-22 14:09:59}
\pmmodified{2013-03-22 14:09:59}
\pmowner{matte}{1858}
\pmmodifier{matte}{1858}
\pmtitle{Lie derivative (for vector fields)}
\pmrecord{9}{35590}
\pmprivacy{1}
\pmauthor{matte}{1858}
\pmtype{Definition}
\pmcomment{trigger rebuild}
\pmclassification{msc}{53-00}
\pmdefines{Lie derivative}

\endmetadata

% this is the default PlanetMath preamble.  as your knowledge
% of TeX increases, you will probably want to edit this, but
% it should be fine as is for beginners.

% almost certainly you want these
\usepackage{amssymb}
\usepackage{amsmath}
\usepackage{amsfonts}

% used for TeXing text within eps files
%\usepackage{psfrag}
% need this for including graphics (\includegraphics)
%\usepackage{graphicx}
% for neatly defining theorems and propositions
%\usepackage{amsthm}
% making logically defined graphics
%%%\usepackage{xypic}

% there are many more packages, add them here as you need them

\newtheorem{thm}{Theorem}

% define commands here

\newcommand{\sR}[0]{\mathbb{R}}
\newcommand{\sC}[0]{\mathbb{C}}
\newcommand{\sN}[0]{\mathbb{N}}
\newcommand{\sZ}[0]{\mathbb{Z}}

\newcommand*{\norm}[1]{\lVert #1 \rVert}
\newcommand*{\abs}[1]{| #1 |}
\begin{document}
\newcommand{\cbra}[1]{\left( #1 \right)}
\newcommand{\qbra}[1]{\left[ #1 \right]}
\newcommand{\gbra}[1]{\left\{ #1 \right\}}
\newcommand{\abra}[1]{\left\langle #1 \right\rangle}



\newcommand{\TTT}{\mathcal{T}}
\newcommand{\UUU}{\mathcal{U}}
\newcommand{\VVV}{\mathcal{V}}
\newcommand{\R}{\mathbb{R}}
\newcommand{\LLL}{\mathcal{L}}

Let $M$ be a smooth manifold, and $X,Y\in\TTT(M)$ smooth vector fields
on $M$. Let $\Theta:\UUU\rightarrow M$  be the flow  of $X$, where
$\UUU\subseteq \R\times M$ is an open neighborhood of
$\gbra{0}\times M$. We make use of the following notation:
$$\UUU^p=\gbra{t\in\R\,|\,(t,p)\in\UUU},\ \ \forall p\in M,$$
$$\UUU_t=\gbra{p\in M\,|\,(t,p)\in\UUU},\ \ \forall t\in\R,$$ and we introduce the
auxiliary maps $\theta_t:\UUU_t\rightarrow M$ and
$\theta^p:\UUU^p\rightarrow M$ defined as
$$\Theta(t,p)=\theta_t(p)=\theta^p(t),\ \ \forall (t,p)\in\UUU.$$


The \emph{Lie derivative} of $Y$ along $X$ is the vector field
$\LLL_XY\in\TTT(M)$ defined by
$$(\LLL_XY)_p=\left.      \frac{d}{dt}     \cbra{         d(\theta_{-t})_{\theta_t(p)}  (Y_{\theta_t(p)})     }     \right|_{t=0}
=\lim_{t\rightarrow0}\frac{d(\theta_{-t})_{\theta_t(p)}
(Y_{\theta_t(p)}) - Y_p}{t},\ \ \forall p\in M,$$
where $d(\theta_{-t})_{\theta_t(p)}\in\mathrm{Hom}(T_{\theta_{t}(p)}M,T_pM)$ if the push-forward of $\theta_{-t}$, i.e. 
$$d(\theta_{-t})_{\theta_t(p)}(v)(f)=v(f\circ\theta_{-t}),\ \ \ \forall v\in T_{\theta_{-t}(p)}M,\ f\in C^\infty(p).$$

The following result is not immediate at all.
\begin{thm}
$\LLL_XY=[X,Y]$, where $[X,Y]=XY-YX$ is the Lie bracket of $X$ and
$Y$.
\end{thm}

%%%%%
%%%%%
\end{document}
