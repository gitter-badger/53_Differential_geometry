\documentclass[12pt]{article}
\usepackage{pmmeta}
\pmcanonicalname{Codifferential}
\pmcreated{2013-03-22 18:37:11}
\pmmodified{2013-03-22 18:37:11}
\pmowner{whm22}{2009}
\pmmodifier{whm22}{2009}
\pmtitle{codifferential}
\pmrecord{5}{41354}
\pmprivacy{1}
\pmauthor{whm22}{2009}
\pmtype{Definition}
\pmcomment{trigger rebuild}
\pmclassification{msc}{53B21}
\pmrelated{DifferentialForms}
\pmrelated{Laplacian}

% this is the default PlanetMath preamble.  as your knowledge
% of TeX increases, you will probably want to edit this, but
% it should be fine as is for beginners.

% almost certainly you want these
\usepackage{amssymb}
\usepackage{amsmath}
\usepackage{amsfonts}

% used for TeXing text within eps files
%\usepackage{psfrag}
% need this for including graphics (\includegraphics)
%\usepackage{graphicx}
% for neatly defining theorems and propositions
%\usepackage{amsthm}
% making logically defined graphics
%%%\usepackage{xypic}

% there are many more packages, add them here as you need them

% define commands here

\begin{document}
The codifferential $\delta$ of a $k$-form on an $n$-dimensional
Riemannian manifold is given by:

$$(-1)^{n(k+1)+1}\ast d \ast$$

where $\ast$ is the Hodge star operator and $d$ is the exterior
derivative.

\bigskip
Let $g$ denote the matrix locally representing the metric with
respect to co-ordinates $x_1,\cdots,x_n$. Then for a 1-form $w$ we
have:

$$
\delta w = \frac{-1}{\surd{({\rm Det } g)}} \frac{\partial}{\partial
x_i}\left[\surd{({\rm Det } g)} \{g^{-1}\}_{ij}  w_j \right]
$$
%%%%%
%%%%%
\end{document}
