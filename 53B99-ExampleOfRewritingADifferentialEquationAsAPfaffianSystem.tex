\documentclass[12pt]{article}
\usepackage{pmmeta}
\pmcanonicalname{ExampleOfRewritingADifferentialEquationAsAPfaffianSystem}
\pmcreated{2013-03-22 14:38:49}
\pmmodified{2013-03-22 14:38:49}
\pmowner{rspuzio}{6075}
\pmmodifier{rspuzio}{6075}
\pmtitle{example of rewriting a differential equation as a Pfaffian system}
\pmrecord{6}{36235}
\pmprivacy{1}
\pmauthor{rspuzio}{6075}
\pmtype{Example}
\pmcomment{trigger rebuild}
\pmclassification{msc}{53B99}

\endmetadata

% this is the default PlanetMath preamble.  as your knowledge
% of TeX increases, you will probably want to edit this, but
% it should be fine as is for beginners.

% almost certainly you want these
\usepackage{amssymb}
\usepackage{amsmath}
\usepackage{amsfonts}

% used for TeXing text within eps files
%\usepackage{psfrag}
% need this for including graphics (\includegraphics)
%\usepackage{graphicx}
% for neatly defining theorems and propositions
%\usepackage{amsthm}
% making logically defined graphics
%%%\usepackage{xypic}

% there are many more packages, add them here as you need them

% define commands here
\begin{document}
To show how one may reformulate a differential equation as Pfaff's problem for a set of differential forms, consider the wave equation
 $${\partial^2 u \over \partial t^2} = {\partial^2 u \over \partial x^2} + {\partial^2 u \over \partial y^2}$$

The first step is to rewrite the equation as a system of first-order equations
 $${\partial a \over \partial t} - {\partial b \over \partial x} - {\partial c \over \partial y} = 0$$
 $${\partial u \over \partial t} - a = 0$$
 $${\partial u \over \partial x} - b = 0$$
 $${\partial u \over \partial y} - c = 0$$

To translate these equations into the language of differential forms, we shall use the fact that
 $$du = {\partial u \over \partial t} \, dt + {\partial u \over \partial x} \, dx + {\partial u \over \partial y} \, dy$$
from which it follows that
 $$du \wedge dx \wedge dy = {\partial u \over \partial t} \, dt \wedge dx \wedge dy$$
 $$du \wedge dy \wedge dt = {\partial u \over \partial x} \, dt \wedge dx \wedge dy$$
 $$du \wedge dt \wedge dx = {\partial u \over \partial y} \, dt \wedge dx \wedge dy$$
We can do likewise with $a$ or $b$ or $c$ in the place of $u$; there is no point in repeating the formulas for each of these variables.

Multiplying the differential equations through by the form $dt \wedge dx \wedge dy$ and using the above identities to eliminate partial derivatives, we obtain the following system of differential forms:
 $$da \wedge dx \wedge dy - db \wedge dy \wedge dt - dc \wedge dt \wedge dx$$
 $$du \wedge dx \wedge dy - a \, dt \wedge dx \wedge dy$$
 $$du \wedge dy \wedge dt - b \, dt \wedge dx \wedge dy$$
 $$du \wedge dt \wedge dx - c \, dt \wedge dx \wedge dy$$
From the way these forms were constructed, it is clear that a three dimensional surface in the seven dimensional space with coordinates $x,y,t,a,b,c,u$ which solves Pfaff's problem and can be parameterized by $x,y,t$ corresponds to the graph of a solution to the system of differential equations, and hence to a solution of the wave equation.

Note: These considerations are purely local.  The global topology of the seven-dimensional space will depend on the domain on which the original wave equation was formulated and on the boundary conditions.
%%%%%
%%%%%
\end{document}
