\documentclass[12pt]{article}
\usepackage{pmmeta}
\pmcanonicalname{SectionalCurvatureDeterminesRiemannCurvatureTensor}
\pmcreated{2013-03-22 15:55:09}
\pmmodified{2013-03-22 15:55:09}
\pmowner{kerwinhui}{11200}
\pmmodifier{kerwinhui}{11200}
\pmtitle{sectional curvature determines Riemann curvature tensor}
\pmrecord{10}{37924}
\pmprivacy{1}
\pmauthor{kerwinhui}{11200}
\pmtype{Theorem}
\pmcomment{trigger rebuild}
\pmclassification{msc}{53B21}
\pmclassification{msc}{53B20}

\usepackage{amsmath}
\usepackage{amsthm}

\DeclareMathOperator{\SPAN}{span}

\newtheorem{theorem}{Theorem}
\newtheorem*{notation}{Notation}
\begin{document}
\begin{theorem}\label{thm:thm1}
The sectional curvature operator $\Pi\mapsto K(\Pi)$ completely determines the Riemann curvature tensor.
\end{theorem}

In fact, a more general result is true.  Recall the Riemann $(1,3)$-curvature tensor $R\colon TM\otimes TM\otimes TM\to TM$ satisfies
\begin{align}
(x,y,z,t)+(y,z,x,t)+(z,x,y,t)&=0\quad\text{First Bianchi identity}\label{eqn:bianchi}\\
(x,y,z,t)+(y,x,z,t)&=0\label{eqn:anti}\\
(x,y,z,t)-(z,t,x,y)&=0,\label{eqn:symm}
\end{align}
where $(x,y,z,t):=g(R(x,y,z),t)$, and the sectional curvature is defined by
\begin{equation}
K(\Pi=\SPAN\{x,y\})=\frac{g(R(x,y,x),y)}{g(x,x)g(y,y)-g(x,y)^2}.\label{eqn:sectcurv}
\end{equation}
Thus Theorem~\ref{thm:thm1} is implied by

\begin{theorem}\label{thm:thm2}
Let $V$ be a real inner product space, with inner product $\langle-,-\rangle$.  Let $R$ and $R'$ be linear maps $V^{\otimes 3}\to V$.  Suppose $R$ and $R'$ satisfies
\begin{itemize}
\item Equations \eqref{eqn:bianchi},\eqref{eqn:anti},\eqref{eqn:symm}, and
\item $K(\sigma)=K'(\sigma)$ for all $2$-planes $\sigma$, where $K,K'$ are defined by \eqref{eqn:sectcurv} using $\langle -,-\rangle$ in \PMlinkescapetext{place} of $g(-,-)$.
\end{itemize}
Then $R=R'$.
\end{theorem}

Write
\begin{align*}
(x,y,z,t)&:=\langle R(x,y,z),t\rangle\\
(x,y,z,t)'&:=\langle R'(x,y,z),t\rangle.
\end{align*}

\begin{proof}[Proof of Theorem~\ref{thm:thm2}]
We need to prove, for all $x,y,z,t\in V$,
\[
(x,y,z,t)=(x,y,z,t)'.
\]

From $K=K'$, we get $(x,y,x,y)=(x,y,x,y)'$ for all $x,y\in V$.  The first step is to use polarization identity to change this quadratic form (in $x$) into its associated symmetric bilinear form.  Expand $(x+z,y,x+z,y)=(x+z,y,x+z,y)'$ and use \eqref{eqn:symm}, we get
\[
(x,y,x,y)+2(x,y,z,y)+(z,y,z,y)=(x,y,x,y)'+2(x,y,z,y)'+(z,y,z,y)'.
\]
So $(x,y,z,y)=(x,y,z,y)'$ for all $x,y,z\in V$.

Unfortunately, the form $(x,y,z,t)$ is not symmetric in $y$ and $t$, so we need to work harder.  Expand $(x,y+t,z,y+t)=(x,y+t,z,y+t)'$, we get
\[
(x,y,z,t)+(x,t,z,y)=(x,y,z,t)'+(x,t,z,y)'.
\]
Now use \eqref{eqn:anti} and \eqref{eqn:symm}, we get
\begin{align*}
(x,y,z,t)-(x,y,z,t)'&=(x,t,z,y)'-(x,t,z,y)\\
&=(z,y,x,t)'-(z,y,x,t)\\
&=(y,z,x,t)-(y,z,x,t)'.
\end{align*}
So $(x,y,z,t)-(x,y,z,t)'$ is invariant under cyclic permutation of $x,y,z$.  But the cyclic sum is zero by \eqref{eqn:bianchi}.  So
\[
(x,y,z,t)=(x,y,z,t)'\quad\forall x,y,z,t\in V
\]
as desired.
\end{proof}
%%%%%
%%%%%
\end{document}
