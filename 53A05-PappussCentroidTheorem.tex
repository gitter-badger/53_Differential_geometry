\documentclass[12pt]{article}
\usepackage{pmmeta}
\pmcanonicalname{PappussCentroidTheorem}
\pmcreated{2013-03-22 15:28:06}
\pmmodified{2013-03-22 15:28:06}
\pmowner{stevecheng}{10074}
\pmmodifier{stevecheng}{10074}
\pmtitle{Pappus's centroid theorem}
\pmrecord{7}{37321}
\pmprivacy{1}
\pmauthor{stevecheng}{10074}
\pmtype{Theorem}
\pmcomment{trigger rebuild}
\pmclassification{msc}{53A05}
\pmsynonym{Guldin's rule}{PappussCentroidTheorem}
\pmsynonym{Guldinus theorem}{PappussCentroidTheorem}
\pmsynonym{Guldin's theorem}{PappussCentroidTheorem}
\pmsynonym{Pappus's theorem for solids of revolution}{PappussCentroidTheorem}
\pmsynonym{Pappus's theorem for surfaces of revolution}{PappussCentroidTheorem}
\pmrelated{Centroid2}
\pmrelated{CentreOfMass}
\pmrelated{SurfaceOfRevolution2}
\pmrelated{VolumeOfSolidOfRevolution}

\usepackage{amssymb}
\usepackage{amsmath}
\usepackage{amsfonts}
\usepackage{amsthm}
\usepackage{enumerate}

% used for TeXing text within eps files
%\usepackage{psfrag}
% need this for including graphics (\includegraphics)
%\usepackage{graphicx}
% making logically defined graphics
%%%\usepackage{xypic}

% define commands here
\newcommand{\complex}{\mathbb{C}}
\newcommand{\real}{\mathbb{R}}
\newcommand{\rat}{\mathbb{Q}}
\newcommand{\nat}{\mathbb{N}}

\providecommand{\abs}[1]{\lvert#1\rvert}
\providecommand{\absW}[1]{\left\lvert#1\right\rvert}
\providecommand{\absB}[1]{\Bigl\lvert#1\Bigr\rvert}
\providecommand{\norm}[1]{\lVert#1\rVert}
\providecommand{\normW}[1]{\left\lVert#1\right\rVert}
\providecommand{\normB}[1]{\Bigl\lVert#1\Bigr\rVert}
\providecommand{\defnterm}[1]{\emph{#1}}

\DeclareMathOperator{\D}{D}
\DeclareMathOperator{\linspan}{span}

\newtheorem{thm}{Theorem}
\newtheorem{ex}{Example}
\begin{document}
\begin{thm}
The surface of revolution generated by a smooth curve $\gamma$
in the xz-plane (with $x \geq 0$), rotated about the z axis,
has surface area
\[
A = sd\,,
\]
where $s$ is the arc length of $\gamma$, and $d$ is the distance travelled
by the centroid $\mu$ of $\gamma$ under a full rotation.
(The centroid is also called the centre of mass, assuming the curve
has uniform line density.)
\end{thm}

\begin{thm}
The solid of revolution generated by a region (Lebesgue-measurable set) $\mathcal{S}$ in the xz-plane (with $x \geq 0$)
rotated about the z axis,
has volume
\[
V = Ad\,,
\]
where $A$ is the area of $\mathcal{S}$, and $d$ is the distance travelled by the centroid
$\mu$ of $\mathcal{S}$ under a full rotation.  
\end{thm}

In English-speaking countries, these two theorems are known as Pappus's theorems,
after the ancient Greek geometer Pappus of Alexandria.
In continental Europe, these theorems are more commonly associated with the name of 
Paul Guldin (who rediscovered them):
e.g. in German ``die guldinsche Regeln''; in Finnish ``Guldinin s\"a\"ann\"ot'';
in French ``le th\'eor\`eme de Guldin''.

\bigskip

\begin{ex}
The surface area of the torus,
with the generating circle having radius $r$, and ring ``radius'' $R$ (measured from the centre of the torus to the centre of the generating circle),
is $A = (2\pi r)(2\pi R) = 4\pi^2 rR$.
We used here the obvious fact that the centroid of a circle is at its centre.
\end{ex}

\begin{ex}
The volume of the (solid) torus, with the same parameters as above, is $V = (\pi r^2) (2\pi R) = 2\pi^2 r^2 R$.
\end{ex}

\begin{ex}
We compute the volume of the three-dimensional ball in $\real^3$.
The ball can be considered to be the solid of revolution generated
by a half-disk. 
So we will need to know the centroid of the upper-half disk $B^2_+(r)$, radius $r$, in the plane.
By symmetry, this centroid has only a vertical component and no horizontal component.
The vertical component is calculated by:
\begin{align*}
\frac{1}{\pi r^2/2} \int_{B^2_+(r)} y \, dx dy &= 
\frac{r}{\pi/2} \int_{B^2_+(1)} y \, dx dy \\
&= 
\frac{r}{\pi/2} \int_0^1 \int_{-\sqrt{1-y^2}}^{\sqrt{1-y^2}} y \, dx \, dy \\
&=
\frac{r}{\pi/2} \int_0^1 2y \sqrt{1-y^2} \, dy \\
&=
\frac{4r}{3\pi}\,.
\end{align*}
Then the volume of the ball of radius $r$ is
\[
V = \left(\frac{4r}{3\pi} \cdot 2 \pi\right) \left( \frac{\pi r^2}{2} \right) = \frac{4\pi}{3} r^3\,.
\]
\end{ex}

\begin{ex}
Similarly we can compute the surface area of the sphere of radius $r$,
generated by revolving a half-circular arc.
The centroid of the upper half-circle $S^1_+(r)$ in the plane only has
the vertical component:
\[
\frac{1}{\pi r} \int_0^\pi y \, r dt = \frac{1}{\pi r} \int_0^\pi (r \sin t) \, r dt = \frac{r}{\pi} \int_0^\pi \sin t \, dt = \frac{2r}{\pi}\,.
\]
Thus the surface area of the sphere is given by
\[
A = \left( \frac{2r}{\pi} \cdot 2\pi  \right) \pi r = 4\pi r^2\,.
\]
\end{ex}
%%%%%
%%%%%
\end{document}
