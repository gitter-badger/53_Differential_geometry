\documentclass[12pt]{article}
\usepackage{pmmeta}
\pmcanonicalname{GeodesicTriangle}
\pmcreated{2013-03-22 17:11:31}
\pmmodified{2013-03-22 17:11:31}
\pmowner{Wkbj79}{1863}
\pmmodifier{Wkbj79}{1863}
\pmtitle{geodesic triangle}
\pmrecord{7}{39510}
\pmprivacy{1}
\pmauthor{Wkbj79}{1863}
\pmtype{Definition}
\pmcomment{trigger rebuild}
\pmclassification{msc}{53C22}

\usepackage{amssymb}
\usepackage{amsmath}
\usepackage{amsfonts}
\usepackage{pstricks}
\usepackage{psfrag}
\usepackage{graphicx}
\usepackage{amsthm}
%%\usepackage{xypic}

\begin{document}
Let $M$ be a differentiable manifold (at least two times differentiable) and $A,B,C \in M$ (not necessarily distinct).  Let $x_1,x_2,x_3\in [0,\infty)$.  Let $\gamma_1 \colon [0,x_1] \to M$, $\gamma_2 \colon [0,x_2] \to M$, and $\gamma_3 \colon [0,x_3] \to M$ be geodesics such that all of the following hold:

\begin{itemize}
\item $\gamma_1(0)=A$;
\item $\gamma_1(x_1)=B$;
\item $\gamma_2(0)=A$;
\item $\gamma_2(x_2)=C$;
\item $\gamma_3(0)=B$;
\item $\gamma_3(x_3)=C$.
\end{itemize}

Then the figure determined by $\gamma_1$, $\gamma_2$, and $\gamma_3$ is a \emph{geodesic triangle}.

Note that a geodesic triangle need not be a triangle.  For example, in $\mathbb{R}^2$, if $A=(0,0)$, $B=(1,2)$, and $C=(3,6)$, then the geodesic triangle determined by $A$, $B$, and $C$ is $\{(x,2x): x\in[0,3]\}$, which is not a triangle.

\begin{center}
\begin{pspicture}(-1,0)(3,6)
\rput[a](3,6){.}
\psline(0,0)(3,6)
\psdots(0,0)(1,2)(3,6)
\rput[r](-0.2,0){$A$}
\rput[r](0.8,2){$B$}
\rput[r](2.8,6){$C$}
\end{pspicture}
\end{center}

\PMlinkescapetext{This entry is not yet complete, as a} \PMlinkname{geodesic metric space}{GeodesicMetricSpace} \PMlinkescapetext{has not yet been defined on PlanetMath.  If the words ``geodesic metric space'' are clickable in the previous sentence, please let me know right away.  Thanks.}
%%%%%
%%%%%
\end{document}
