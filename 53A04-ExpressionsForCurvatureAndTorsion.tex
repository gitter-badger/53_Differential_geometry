\documentclass[12pt]{article}
\usepackage{pmmeta}
\pmcanonicalname{ExpressionsForCurvatureAndTorsion}
\pmcreated{2013-03-22 15:38:14}
\pmmodified{2013-03-22 15:38:14}
\pmowner{Simone}{5904}
\pmmodifier{Simone}{5904}
\pmtitle{expressions for curvature and torsion}
\pmrecord{7}{37567}
\pmprivacy{1}
\pmauthor{Simone}{5904}
\pmtype{Theorem}
\pmcomment{trigger rebuild}
\pmclassification{msc}{53A04}
\pmrelated{Torsion}
\pmrelated{CurvatureOfACurve}

% this is the default PlanetMath preamble.  as your knowledge
% of TeX increases, you will probably want to edit this, but
% it should be fine as is for beginners.

% almost certainly you want these
\usepackage{amssymb}
\usepackage{amsmath}
\usepackage{amsfonts}

% used for TeXing text within eps files
%\usepackage{psfrag}
% need this for including graphics (\includegraphics)
%\usepackage{graphicx}
% for neatly defining theorems and propositions
%\usepackage{amsthm}
% making logically defined graphics
%%%\usepackage{xypic}

% there are many more packages, add them here as you need them

% define commands here
\begin{document}
For a \PMlinkname{regular}{Curve}, parameterized curve $\alpha\colon (a,b)\to\mathbb R^3$, not necessarily unit speed, the curvature $\kappa(t)$ and torsion $\tau(t)$ are given, respectively,  by
\begin{align*}
\kappa(t)&=\frac{\|\alpha'(t)\times\alpha''(t)\|}{\|\alpha'(t)\|^3};\\
\tau(t)&=\frac{(\alpha'(t)\times\alpha''(t))\cdot\alpha'''(t)}{\|\alpha'(t)\times\alpha''(t)\|^2}.
\end{align*}

\begin{thebibliography}
{} John McCleary, \emph{Geometry from a Differentiable Viewpoint}, Cambridge University Press, 1994.
\end{thebibliography}
%%%%%
%%%%%
\end{document}
