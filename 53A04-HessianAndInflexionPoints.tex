\documentclass[12pt]{article}
\usepackage{pmmeta}
\pmcanonicalname{HessianAndInflexionPoints}
\pmcreated{2013-03-22 18:22:26}
\pmmodified{2013-03-22 18:22:26}
\pmowner{rspuzio}{6075}
\pmmodifier{rspuzio}{6075}
\pmtitle{Hessian and inflexion points}
\pmrecord{14}{41015}
\pmprivacy{1}
\pmauthor{rspuzio}{6075}
\pmtype{Theorem}
\pmcomment{trigger rebuild}
\pmclassification{msc}{53A04}
\pmclassification{msc}{26A51}

\endmetadata

% this is the default PlanetMath preamble.  as your knowledge
% of TeX increases, you will probably want to edit this, but
% it should be fine as is for beginners.

% almost certainly you want these
\usepackage{amssymb}
\usepackage{amsmath}
\usepackage{amsfonts}

% used for TeXing text within eps files
%\usepackage{psfrag}
% need this for including graphics (\includegraphics)
%\usepackage{graphicx}
% for neatly defining theorems and propositions
\usepackage{amsthm}
% making logically defined graphics
%%%\usepackage{xypic}

% there are many more packages, add them here as you need them

% define commands here
\newtheorem{thm}{Theorem}
\begin{document}
\begin{thm}
Suppose that $C$ is a curve in the real projective plane 
$\mathbb{R}\mathbb{P}^2$ given by a homogeneous
equation $F(x,y,z) = 0$ of 
\PMlinkname{degree of homogeneity}{HomogeneousFunction} $n$.
If $F$ has continuous first derivatives in a neighborhood of 
a point $P$ and the gradient of $F$ is non-zero at $P$ and $P$ 
is an inflection point of $C$, then $H(P) = 0$, where $H$ is 
the Hessian determinant:
\[
 H = \left| \begin{matrix}
   {\partial^2 F \over \partial x^2} &
   {\partial^2 F \over \partial x \partial y} &
   {\partial^2 F \over \partial x \partial z} \\
   {\partial^2 F \over \partial y \partial x} &
   {\partial^2 F \over \partial y^2} &
   {\partial^2 F \over \partial y \partial z} \\
   {\partial^2 F \over \partial z \partial x} &
   {\partial^2 F \over \partial z \partial y} &
   {\partial^2 F \over \partial z^2}
 \end{matrix} \right|
\]
\end{thm}


\begin{proof}
We may choose a system $x,y,z$ of homogenous coordinates 
such that the point $P$ lies at $(0,0,1)$ and the equation 
of the tangent to $C$ at $P$ is $y = 0$.  Using the 
implicit function theorem, we may conclude that there
exists an interval $(-\epsilon,\epsilon)$ and a function
$f \colon (-\epsilon,\epsilon) \to \mathbb{R}$ such that
$F(t, f(t), 1) = 0$ when $-\epsilon < t < \epsilon$.  In
other words, the portion of curve near $P$ may be described
in non-homogenous coordinates by $y = f(x)$.  By the way
the coordinates were chosen, $f(0) = 0$ and $f'(0) = 0$.
Because $P$ is an inflection point, we also have $f''(0) = 0$.

Differentiating the equation $F(t, f(t), 1) = 0$ twice,
we obtain the following:
\begin{align*}
 0 = {d \over dt} F(t, f(t), 1) &=
   {\partial F \over \partial x} (t, f(t), 1) +
   f'(t) {\partial F \over \partial y} (t, f(t), 1)
\\
 0 = {d^2 \over dt^2} F(t, f(t), 1) &=
   {\partial^2 F \over \partial x^2} (t, f(t), 1) +
   f'(t) {\partial^2 F \over \partial x \partial y} (t, f(t), 1) \\ &\quad+
   \left(f'(t)\right)^2 {\partial^2 F \over \partial y^2} (t, f(t), 1) +
   f''(t) {\partial F \over \partial y} (t, f(t), 1)
\end{align*}
We will now put $t=0$ but, for reasons which will be 
explained later, we do not yet want to make use of the
fact that $f''(0) = 0$:
\begin{align*}
 {\partial F \over \partial x} (0,0,1) &= 0 \\
 {\partial^2 F \over \partial x^2} (0,0,1) &= 
 -f''(0) {\partial F \over \partial y} (0,0,1) 
\end{align*}

Since $F$ is homogenous, Euler's formula holds:
\[
 x {\partial F \over \partial x} +
 y {\partial F \over \partial y} +
 z {\partial F \over \partial z} = n F
\]
Taking partial derivatives, we obtain the following:
\begin{align*}
 x {\partial^2 F \over \partial x^2} +
 y {\partial^2 F \over \partial x \partial y} +
 z {\partial^2 F \over \partial x \partial z} = 
 (n - 1) {\partial F \over \partial x} \\
 x {\partial^2 F \over \partial x \partial y} +
 y {\partial^2 F \over \partial y^2} +
 z {\partial^2 F \over \partial y \partial z} = 
 (n - 1) {\partial F \over \partial y} \\
 x {\partial^2 F \over \partial x \partial z} +
 y {\partial^2 F \over \partial y \partial z} +
 z {\partial^2 F \over \partial z^2} = 
 (n - 1) {\partial F \over \partial z}
\end{align*}
Evaluating at $(0,0,1)$ and making use of the
equations deduced above, we obtain the following:
\begin{align*}
 {\partial F \over \partial z} (0,0,1) &= 0 \\
 {\partial^2 F \over \partial x \partial z} (0,0,1) &= 0 \\
 {\partial^2 F \over \partial y \partial z} (0,0,1) &= 
    (n - 1) {\partial F \over \partial y} (0,0,1) \\
 {\partial^2 F \over \partial z^2} (0,0,1) &= 0
\end{align*} 
Making use of these facts, we may now evaluate the determinant:
\begin{align*}
 H (0,0,1) &= \left| \begin{matrix}
   - f''(0) {\partial F \over \partial y} (0,0,1) &
   {\partial^2 F \over \partial x \partial y} (0,0,1) &
   0 \\
   {\partial^2 F \over \partial x \partial y} (0,0,1) &
   {\partial^2 F \over \partial^2 y} (0,0,1) & 
   (n - 1) {\partial F \over \partial y} (0,0,1) \\
   0 &
   (n - 1) {\partial F \over \partial y} (0,0,1) &
   0
  \end{matrix} \right| \\ &=
 (n - 1)^2 
 \left( {\partial F \over \partial y} (0,0,1) \right)^2
 f'' (0)
\end{align*}
Since $P$ is an inflection point, $f''(0) = 0$, so we
have $H(0,0,1) = 0$.
\end{proof}

Actually, we proved slightly more than what was stated.
Because the gradient is assumed not to vanish at $P$,
but $\partial F / \partial x = 0$ and $\partial F
/ \partial z = 0$ by the way we set up our coordinate
system, we must have $\partial F / \partial y \neq 0$.
Thus, we see that, if $n \neq 1$, then $H(0,0,1) = 0$ 
if and only if $f''(0)$.  However, note that this does
not mean that the Hessian vanishes if and only if $P$ is
an inflection point since the definition of inflection
point not only requires that $f''(0) = 0$ but that the
sign of $f''(t)$ change as $t$ passes through $0$.

This result is used quite often in algebraic geometry,
where $F$ is a homogenous polynomial.  In such a context,
it is desirable to keep demonstrations purely
algebraic and avoid introducing analysis where possible, 
so a variant of this result is preferred.  The theorem 
may be restated as follows:

\begin{thm}
Suppose that $C$ is a curve in the real projective plane 
$\mathbb{R}\mathbb{P}^2$ given by an equation $F(x,y,z) = 0$ 
where $F$ is a homogenous polynomial of degree $n$.
If $C$ is regular at a point $P$ and $P$ is an inflection point 
of $C$, then $H(P) = 0$, where $H$ is the Hessian determinant.
\end{thm}

To make our proof purely algebraic, we replace the use of
the implicit function theorem to obtain $f$ with an expansion
in a formal power series.  As above, we choose our $x,y,z$ 
coordinates so as to place $P$ at $(0,0,1)$ and make $C$
tangent to the line $y=0$ at $P$.  Then, since $P$ is a regular
point of $C$, we may parameterize $C$ by a formal power series
$f(t) = \sum_{k=0}^\infty c_k t^k$ such that $F(t,f(t),1) = 0$.
Then, if we 
\PMlinkname{define derivatives algebraically}{DerivativeOfPolynomial},
we may proceed with the rest of the proof exactly as above.

%%%%%
%%%%%
\end{document}
