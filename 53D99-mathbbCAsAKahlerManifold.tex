\documentclass[12pt]{article}
\usepackage{pmmeta}
\pmcanonicalname{mathbbCAsAKahlerManifold}
\pmcreated{2013-03-22 15:46:32}
\pmmodified{2013-03-22 15:46:32}
\pmowner{cvalente}{11260}
\pmmodifier{cvalente}{11260}
\pmtitle{$\mathbb{C}$ as a K\"ahler manifold}
\pmrecord{16}{37733}
\pmprivacy{1}
\pmauthor{cvalente}{11260}
\pmtype{Example}
\pmcomment{trigger rebuild}
\pmclassification{msc}{53D99}
\pmrelated{KahlerManifold}
\pmrelated{AlmostComplexStructure}
\pmrelated{SymplecticManifold}

\endmetadata

% this is the default PlanetMath preamble.  as your knowledge
% of TeX increases, you will probably want to edit this, but
% it should be fine as is for beginners.

% almost certainly you want these
\usepackage{amssymb}
\usepackage{amsmath}
\usepackage{amsfonts}

% used for TeXing text within eps files
%\usepackage{psfrag}
% need this for including graphics (\includegraphics)
%\usepackage{graphicx}
% for neatly defining theorems and propositions
%\usepackage{amsthm}
% making logically defined graphics
%%%\usepackage{xypic}

% there are many more packages, add them here as you need them

% define commands here
\begin{document}
$\mathbb{C}$ can be interpreted as $\mathbb{R}^2$ with a \PMlinkname{complex structure}{AlmostComplexStructure} $J$.

Parametrize $\mathbb{R}^2$ via the usual coordinates $(x,y)$.

A point $z$ in the complex plane can thus be written $z=x+iy$.

The tangent space at each point is generated by the $\mathrm{span}_{\mathbb{R}} \left\{ \frac{\partial}{\partial x}, \frac{\partial}{\partial y}\right\}$ and the \PMlinkname{complex structure}{AlmostComplexStructure} $J$ is defined by\footnote{notice $J$ acts as a counterclockwise rotation by $\frac{\pi}{2}$, just as expected}

\begin{align}
J\left( \frac{\partial}{\partial x} \right) = \frac{\partial}{\partial y} \\
J\left( \frac{\partial}{\partial y} \right) = - \frac{\partial}{\partial x}
\end{align}

The metric can be the usual metric $g = dx\otimes dx + dy\otimes dy$.
This is a flat metric and therefore all the covariant derivatives are plain partial derivatives in the $(x,y)$ coordinates\footnote{the Christoffel symbols on these coordinates vanish}.

So lets verify all the points in the definition.

\begin{itemize}
\item $\mathbb{C}$ is a Riemannian Manifold
\item $g$ is Hermitian.
$$
g\left(J\frac{\partial}{\partial x}, J\frac{\partial}{\partial y}\right) = g\left(\frac{\partial}{\partial y},-\frac{\partial}{\partial x} \right)=0=g\left(\frac{\partial}{\partial x}, \frac{\partial}{\partial y}\right) \\
$$
$$
g\left(J\frac{\partial}{\partial x}, J\frac{\partial}{\partial x}\right) = g\left(\frac{\partial}{\partial y}, \frac{\partial}{\partial y}\right)=1 = g\left(\frac{\partial}{\partial x}, \frac{\partial}{\partial x}\right)
$$

$$
g\left(J\frac{\partial}{\partial y}, J\frac{\partial}{\partial y}\right) = g\left(-\frac{\partial}{\partial x}, -\frac{\partial}{\partial x}\right)=1 = g\left(\frac{\partial}{\partial y}, \frac{\partial}{\partial y}\right)
$$

\item $J$ is covariantly constant because its components in the $(x,y)$ coordinates are constant and as previously stated, the covariant derivatives are just partial derivatives in this example.

\end{itemize}

$\mathbb{C}$ is therefore a K\"ahler manifold.

The symplectic form for this example is

$$\omega = dx \wedge dy $$

This is the simplest example of a K\"ahler manifold and can be seen as a template for other less trivial examples. Those are generalizations of this example just as Riemannian manifolds are generalizations of $\mathbb{R}^n$  seen as a metric space.
%%%%%
%%%%%
\end{document}
