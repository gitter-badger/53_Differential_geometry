\documentclass[12pt]{article}
\usepackage{pmmeta}
\pmcanonicalname{EvoluteOfCycloid}
\pmcreated{2013-03-22 19:18:06}
\pmmodified{2013-03-22 19:18:06}
\pmowner{pahio}{2872}
\pmmodifier{pahio}{2872}
\pmtitle{evolute of cycloid}
\pmrecord{7}{42237}
\pmprivacy{1}
\pmauthor{pahio}{2872}
\pmtype{Derivation}
\pmcomment{trigger rebuild}
\pmclassification{msc}{53A04}
\pmrelated{Cycloid}
\pmrelated{LogarithmicSpiral}

\endmetadata

\usepackage{amssymb}
\usepackage{amsmath}
\usepackage{amsfonts}
\usepackage{amsthm}
\usepackage{pstricks}
\usepackage{pst-plot}

\usepackage{mathrsfs}

\newcommand{\sR}[0]{\mathbb{R}}
\newcommand{\sC}[0]{\mathbb{C}}
\newcommand{\sN}[0]{\mathbb{N}}
\newcommand{\sZ}[0]{\mathbb{Z}}

 \usepackage{bbm}
 \newcommand{\Z}{\mathbbmss{Z}}
 \newcommand{\C}{\mathbbmss{C}}
 \newcommand{\F}{\mathbbmss{F}}
 \newcommand{\R}{\mathbbmss{R}}
 \newcommand{\Q}{\mathbbmss{Q}}



\newcommand*{\norm}[1]{\lVert #1 \rVert}
\newcommand*{\abs}[1]{| #1 |}



\newtheorem{thm}{Theorem}
\newtheorem{defn}{Definition}
\newtheorem{prop}{Proposition}
\newtheorem{lemma}{Lemma}
\newtheorem{cor}{Corollary}

\begin{document}
\PMlinkescapeword{distance}

We shall determine the evolute of the cycloid
\begin{align}
x \;=\; a(u-\sin{u}), \qquad y \;=\; a(1-\cos{u}),
\end{align}
where the parametre $u$ is the rolling angle of the circle with radius $a$ forming the cycloid.\\

The parametric equations of the evolute of a curve\, $x = x(u),\; y = y(u)$\, are
\begin{align}
\xi \;=\; x-\varrho\sin{\alpha}, \qquad \eta \;=\; y+\varrho\cos{\alpha}
\end{align}
with $\alpha$ the slope angle of the tangent and $\varrho$ the radius of curvature of the given curve in the point \,$(x,\,y)$\,:
$$\varrho \;=\; \frac{(x'^{\,2}+y'^{\,2})^{3/2}}{x'y''-x''y'}$$

In the case of the cycloid (1), we have
$$x' \;=\; a(1-\cos{u}), \qquad y' \;=\; a\sin{u}, \qquad x'' \;=\; a\sin{u}, \qquad y'' \;=\; a\cos{u}.$$
Now we get
$$x'^{\,2}+y'^{\,2} \;=\; 2a^2(1-\cos{u}) \;=\; 4a^2\sin^2\frac{u}{2},$$
$$x'y''-x''y' \;=\; a^2(\cos{u}-1) \;=\; -2a^2\sin^2\frac{u}{2},$$
and thus the radius of curvature (red in the diagram) is
\begin{align}
\varrho \;=\; -4a\sin\frac{u}{2}.
\end{align}
We utilised the \PMlinkescapetext{identity}
\begin{align}
1-\cos{u} \;=\; 2\sin^2\frac{u}{2}
\end{align}
(see the half angle formula of sine in the goniometric formulas).\, It is easy to show that the point, where the circle touches the $x$-axis, bisects the radius of curvature (which lies on the normal line at the point\, $(x,\,y)$\, of the cycloid).

Using then the derivative for parametric form, we obtain
$$\tan\alpha \;=\; \frac{dy}{dx} \;=\; \frac{y'}{x'} \;=\; \frac{\sin{u}}{1-\cos{u}}.$$
which implies
\begin{align}
\sin\alpha \;=\; \cos\frac{u}{2}, \quad \cos\alpha \;=\; \sin\frac{u}{2},
\end{align}
Substituting all needed expressions (1), (3), (5) into (2) and simplifying, we arrive at the result
\begin{align}
\xi \;=\; a(u+\sin{u}), \qquad \eta \;=\; -a(1-\cos{u}).
\end{align}
The equations (6) \PMlinkescapetext{represent} \textbf{the evolute of the given cycloid.\, But it is also a cycloid}, congruent to the original one, which has been \PMlinkname{translated}{Translate} the distance $\pi a$ to the left and the distance $2a$ downwards; this is seen when one performs in (6) the substitution \,$u = v-\pi$;\, then (6) reads
$$\xi \;=\; a(v+\sin{v})-\pi a, \qquad \eta \;=\; a(1-\cos{v})-2a.$$\\


\begin{center}

\begin{pspicture}(-2,-2)(7.3,3)
\psaxes[Dx=9,Dy=9]{->}(0,0)(-2,-1.9)(7.2,2.3)
\pscurve[linecolor=blue] (-0.0783,0.2929)
                         (-0.0335,0.16853)(-0.01,0.07612)(-0.00126,0.019215)(0,0)
                        (0,0)(0.00126,0.019215)(0.01,0.07612)(0.0335,0.16853)(0.0783,0.2929)(0.1503,0.44443)
                        (0.25422,0.61732)(0.39366,0.80491)(0.5708,1)(0.78636,1.19509)(1.04,1.3827)(1.3284,1.5556)
                        (1.649,1.7071)(2,1.83147)(2.36621,1.924)(2.75,1.98)(3.1416,2)(3.533,1.98)(3.917,1.924)
                        (4.2862,1.83147)(4.6341,1.7071)(4.955,1.5556)(5.24357,1.3827)(5.4969,1.19509)(5.7124,1)
                        (5.89,0.80491)(6.03,0.61732)(6.133,0.44443)(6.205,0.2929)(6.25,0.16853)(6.2732,0.07612)
                        (6.282,0.019215)
                         (6.2832,0)(6.2844,0.019215)(6.2932,0.07612)(6.317,0.16853)(6.3615,0.2929)(6.4335,0.44443)
                         (6.5374,0.61732)
\pscircle(1.1,1){1}
\psdots(1.1,1)(1.1,0)
\psdot(0.23,0.55)
\rput(-0.2,0.55){$(x,y)$}
\psline(0.23,0.55)(1.1,1)
\psline[linestyle=dashed](1.1,1)(1.1,0)
\psline(1.19,0)(1.19,0.09)
\psline(1.1,0.09)(1.19,0.09)
\rput(2.34,-0.43){$(\xi,\eta)$}
\rput(0.96,0.8){$u$}
\rput[a](7.25,-0.16){$x$}
\rput[r](-0.1,2.36){$y$}

\pscurve[linecolor=cyan](-0.7754,-0.076)(-0.3914,-0.02)(0,0)(0.3914,-0.02)(0.7754,-0.076)
(1.1446,-0.16853)(1.4925,-0.2929)(1.8134,-0.4444)(2.102,-0.6173)(2.2553,-0.741000)(2.5708,-1)
(2.7484,-1.19509)(2.8884,-1.38268)(2.9914,-1.55557)(3.06341,-1.7071)(3.10841,-1.83147)(3.1316,-1.92388)
(3.1404,-1.980785)
(3.14159,-2)(3.1429,-1.980785)(3.15159,-1.92388)(3.17509,-1.83147)(3.2199,-1.7071)
(3.292,-1.55557)(3.3958,-1.38268)(3.5353,-1.19509)(3.7124,-1)(3.928,-0.80491)(4.1816,-0.6173)(4.47,-0.4444)
(4.7906,-0.2929)(5.14159,-0.16853)(5.5078,-0.076)(5.8916,-0.02)(6.2832,0)(6.6748,-0.02)(7.0586,-0.076)
\psline[linecolor=red](0.23,0.55)(1.97,-0.54)
\psdot[linecolor=red](1.97,-0.54)
\rput(-2,-2){.}
\rput(7.3,3){.}
\end{pspicture}
\end{center}



%%%%%
%%%%%
\end{document}
