\documentclass[12pt]{article}
\usepackage{pmmeta}
\pmcanonicalname{SerretFrenetEquationsInmathbbR2}
\pmcreated{2013-03-22 15:16:57}
\pmmodified{2013-03-22 15:16:57}
\pmowner{rspuzio}{6075}
\pmmodifier{rspuzio}{6075}
\pmtitle{Serret-Frenet equations in $\mathbb{R}^2$}
\pmrecord{8}{37072}
\pmprivacy{1}
\pmauthor{rspuzio}{6075}
\pmtype{Definition}
\pmcomment{trigger rebuild}
\pmclassification{msc}{53A04}
\pmrelated{SerretFrenetFormulas}

\endmetadata

% this is the default PlanetMath preamble.  as your knowledge
% of TeX increases, you will probably want to edit this, but
% it should be fine as is for beginners.

% almost certainly you want these
\usepackage{amssymb}
\usepackage{amsmath}
\usepackage{amsfonts}
\usepackage{amsthm}

\usepackage{mathrsfs}

% used for TeXing text within eps files
%\usepackage{psfrag}
% need this for including graphics (\includegraphics)
%\usepackage{graphicx}
% for neatly defining theorems and propositions
%
% making logically defined graphics
%%%\usepackage{xypic}

% there are many more packages, add them here as you need them

% define commands here

\newcommand{\sR}[0]{\mathbb{R}}
\newcommand{\sC}[0]{\mathbb{C}}
\newcommand{\sN}[0]{\mathbb{N}}
\newcommand{\sZ}[0]{\mathbb{Z}}

 \usepackage{bbm}
 \newcommand{\Z}{\mathbbmss{Z}}
 \newcommand{\C}{\mathbbmss{C}}
 \newcommand{\F}{\mathbbmss{F}}
 \newcommand{\R}{\mathbbmss{R}}
 \newcommand{\Q}{\mathbbmss{Q}}



\newcommand*{\norm}[1]{\lVert #1 \rVert}
\newcommand*{\abs}[1]{| #1 |}



\newtheorem{thm}{Theorem}
\newtheorem{defn}{Definition}
\newtheorem{prop}{Proposition}
\newtheorem{lemma}{Lemma}
\newtheorem{cor}{Corollary}

\newcommand{\bT}[0]{\mathbf{T}}
\newcommand{\bN}[0]{\mathbf{N}}
\begin{document}
Given a plane curve, we may associate to each point on the curve an
orthonormal basis consisting of the unit normal tangent vector and
the unit normal.  In general, different points will have different 
bases associated to them, so we may ask how the basis depends upon
the choice of point.  The Serret-Frenet equations answer this
question by relating the rte of change of the basis vectors to
the curvature of the curve.

Suppose $I$ is an open interval and $c\colon I \to \R^2$ is a twice
continuously differentiable curve such that $\Vert c'\Vert = 1$. 
Let us then
define the \emph{tangent vecto{r}} and \emph{normal vecto{r}} as
\begin{eqnarray*}
   \bT &=& c', \\
   \bN &=& J \cdot \bT,
\end{eqnarray*}
where $J=\begin{pmatrix} 0 & -1 \\ 1 & 0 \end{pmatrix}$ is the 
rotational matrix that rotates the plane $90$ degrees counterclockwise.

\subsubsection*{Curvature}
Differentiating $\langle c',c'\rangle=1$ yields 
  $\langle \bT',\bT\rangle=0$, 
   so $\bT'$ is in the orthogonal complement of $\bT$,
  which is $1$-dimensional. Since $J\cdot \bT$ is also in 
    the orthogonal complement, 
  it follows that there exists a function $\kappa\colon I \to \R$ such that 
  $$
     \bT'=\kappa J \cdot \bT.
  $$
Furthermore, $\kappa$ is uniquely determined by this equation. 
We define this unique $\kappa$ to 
   be the \emph{curvatur{e}} of $c$. Explicitly,
$$
   \kappa = \langle \bT',J \cdot \bT \rangle.
$$

\subsubsection*{Serret-Frenet equations}
By the definition of curvature 
\begin{eqnarray*}
  \bT'&=& \kappa J\cdot \bT 
      = \kappa \bN
\end{eqnarray*}
and so
\begin{eqnarray*}
  \bN'&=& J\cdot \bT' 
      = \kappa J \bN 
      = -\kappa \bT 
\end{eqnarray*}
since $J^2=-\operatorname{I}$. These are the \emph{Serret-Frenet}
equations
\begin{eqnarray*}
   \begin{pmatrix} \bT \\ \bN\end{pmatrix}' = 
   \begin{pmatrix} 0 & \kappa \\ -\kappa & 0 \end{pmatrix}
\begin{pmatrix} \bT \\ \bN\end{pmatrix}.
\end{eqnarray*}
%%%%%
%%%%%
\end{document}
