\documentclass[12pt]{article}
\usepackage{pmmeta}
\pmcanonicalname{LineThroughAnIntersectionPoint}
\pmcreated{2013-03-22 17:30:28}
\pmmodified{2013-03-22 17:30:28}
\pmowner{pahio}{2872}
\pmmodifier{pahio}{2872}
\pmtitle{line through an intersection point}
\pmrecord{8}{39897}
\pmprivacy{1}
\pmauthor{pahio}{2872}
\pmtype{Topic}
\pmcomment{trigger rebuild}
\pmclassification{msc}{53A04}
\pmclassification{msc}{51N20}

\endmetadata

% this is the default PlanetMath preamble.  as your knowledge
% of TeX increases, you will probably want to edit this, but
% it should be fine as is for beginners.

% almost certainly you want these
\usepackage{amssymb}
\usepackage{amsmath}
\usepackage{amsfonts}

% used for TeXing text within eps files
%\usepackage{psfrag}
% need this for including graphics (\includegraphics)
%\usepackage{graphicx}
% for neatly defining theorems and propositions
 \usepackage{amsthm}
% making logically defined graphics
%%%\usepackage{xypic}

% there are many more packages, add them here as you need them

% define commands here

\theoremstyle{definition}
\newtheorem*{thmplain}{Theorem}

\begin{document}
\PMlinkescapeword{represents}
Suppose that the lines
\begin{align}
       Ax+By+C = 0\;\; \mbox{and}\;\; A'x+B'y+C' = 0
\end{align}
have an intersection point.  Then for any real value of $k$, the equation
\begin{align}
             Ax+By+C+k(A'x+B'y+C') = 0
\end{align}
represents a line passing through that point.

In fact, the \PMlinkescapetext{degree} of the equation (2) is 1, and therefore it represents a line; secondly, (2) is satisfied if both equations (1) are satisfied, and therefore the line passes through that intersection point.\\


\textbf{Example.}  Determine the equation of the line passing through the point \,$(-5,\,2)$\, and the intersection point of the lines\, $6x-7y+9 = 0$\, and\, $5x+9y-3 = 0$.

The equation of a line through the common point of those lines is
\begin{align}            
           6x-7y+9 +k(5x+9y-3) = 0.
\end{align}
We have to find such a value for $k$ that also\, $(-5,\,2)$\, lies on the line, i.e. that the equation (3) is satisfied by the values\, $x = -5$,\, $y = 2$.  So we get for determining $k$ the equation
                   $$-35-10k = 0,$$
whence\, $k = -\frac{7}{2}$.  Using this value in (3), multiplying the equation by 2 and simplifying, we obtain the sought equation
                  $$23x+77y-39 = 0.$$
This result would be obtained, of course, by first calculating the intersection point of the two given lines (it is\, $(-\frac{60}{89},\,\frac{63}{89})$) and then forming the equation of the line passing this point and the point\, $(-5,\,2)$, but then the calculations would have been substantially longer.

\textbf{Note.}  It is apparent that no value of $k$ allows the equation (2) to \PMlinkescapetext{represent} the line\,\\ $A'x+B'y+C' = 0$\, itself.  Thus, if we had in the example instead the point\, $(-5,\,2)$\, e.g. the point\, $(6,\,-3)$\, of the line\, $5x+9y-3 = 0$, then we had the condition\, $66+0k = 0$\, which gives no value of $k$.

\begin{thebibliography}{9}
\bibitem{VA}{\sc K. V\"ais\"al\"a:} {\em Algebran oppi- ja esimerkkikirja II}. Nelj\"as painos. \, Werner S\"oderstr\"om osakeyhti\"o, Porvoo \& Helsinki (1956).
\end{thebibliography}
%%%%%
%%%%%
\end{document}
