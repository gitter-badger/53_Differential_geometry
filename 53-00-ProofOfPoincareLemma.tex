\documentclass[12pt]{article}
\usepackage{pmmeta}
\pmcanonicalname{ProofOfPoincareLemma}
\pmcreated{2013-03-22 14:24:36}
\pmmodified{2013-03-22 14:24:36}
\pmowner{pbruin}{1001}
\pmmodifier{pbruin}{1001}
\pmtitle{proof of Poincar\'e lemma}
\pmrecord{4}{35912}
\pmprivacy{1}
\pmauthor{pbruin}{1001}
\pmtype{Proof}
\pmcomment{trigger rebuild}
\pmclassification{msc}{53-00}
\pmclassification{msc}{55N05}

% this is the default PlanetMath preamble.  as your knowledge
% of TeX increases, you will probably want to edit this, but
% it should be fine as is for beginners.

% almost certainly you want these
\usepackage{amssymb}
\usepackage{amsmath}
\usepackage{amsfonts}

% used for TeXing text within eps files
%\usepackage{psfrag}
% need this for including graphics (\includegraphics)
%\usepackage{graphicx}
% for neatly defining theorems and propositions
%\usepackage{amsthm}
% making logically defined graphics
%%%\usepackage{xypic}

% there are many more packages, add them here as you need them

% define commands here
\begin{document}
Let $X$ be a smooth manifold, and let $\omega$ be a closed differential form of degree $k>0$ on $X$.  For any $x\in X$, there exists a contractible neighbourhood $U\subset X$ of $x$ (i.e. $U$ is homotopy equivalent to a single point), with inclusion map
$$\iota\colon U\hookrightarrow X.$$
To construct such a neighbourhood, take for example an open ball in a coordinate chart around $x$.  Because of the homotopy invariance of de Rham cohomology, the $k$th de Rham cohomology group ${\rm H}^k(U)$ is isomorphic to that of a point; in particular,
$$
{\rm H}^k(U)=0\quad\hbox{for all $k>0$}.
$$
Since $d(\iota^*\omega)=\iota^*(d\omega)=0$, this implies that there exists a $(k-1)$-form $\eta$ on $U$ such that $d\eta=\iota^*\omega$.  In the case where $X$ is a contractible manifold, such an $\eta$ exists globally since we can choose $U=X$ above.
%%%%%
%%%%%
\end{document}
