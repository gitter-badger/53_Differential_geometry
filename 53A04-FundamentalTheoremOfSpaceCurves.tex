\documentclass[12pt]{article}
\usepackage{pmmeta}
\pmcanonicalname{FundamentalTheoremOfSpaceCurves}
\pmcreated{2013-03-22 13:23:28}
\pmmodified{2013-03-22 13:23:28}
\pmowner{rmilson}{146}
\pmmodifier{rmilson}{146}
\pmtitle{fundamental theorem of space curves}
\pmrecord{5}{33929}
\pmprivacy{1}
\pmauthor{rmilson}{146}
\pmtype{Theorem}
\pmcomment{trigger rebuild}
\pmclassification{msc}{53A04}
\pmrelated{SpaceCurve}

\endmetadata

\usepackage{amsmath}
\usepackage{amsfonts}
\usepackage{amssymb}
\newcommand{\reals}{\mathbb{R}}
\newcommand{\natnums}{\mathbb{N}}
\newcommand{\cnums}{\mathbb{C}}
\newcommand{\znums}{\mathbb{Z}}
\newcommand{\lp}{\left(}
\newcommand{\rp}{\right)}
\newcommand{\lb}{\left[}
\newcommand{\rb}{\right]}
\newcommand{\supth}{^{\text{th}}}
\newtheorem{proposition}{Proposition}
\newtheorem{definition}[proposition]{Definition}

\newtheorem{theorem}[proposition]{Theorem}

\newcommand{\bg}{\boldsymbol{\gamma}}
\newcommand{\dbg}{\bg'}
\newcommand{\ddbg}{\bg''}
\newcommand{\dddbg}{\bg'''}
\newcommand{\der}[1]{#1{}'}
\newcommand{\bT}{\mathbf{T}}
\newcommand{\bN}{\mathbf{N}}
\newcommand{\bB}{\mathbf{B}}
\begin{document}
\paragraph{Informal summary.} The curvature and torsion of a space
curve are invariant with respect to Euclidean motions.  Conversely, a
given space curve is determined up to a Euclidean motion, by its
curvature and torsion,  expressed as functions of the arclength.

\paragraph{Theorem.}
Let $\bg:I\to\reals$ be a regular, parameterized space curve, without
points of inflection. Let $\kappa(t), \tau(t)$ be the
corresponding curvature and torsion functions. Let
$T:\reals^3\to\reals^3$ be a Euclidean isometry. The curvature and
torsion of the transformed curve 
$T(\bg(t))$ are given by $\kappa(t)$ and $\tau(t)$, respectively.


Conversely, let $\kappa,\tau:I\to\reals$ be continuous functions,
defined on an interval $I\subset\reals$, and suppose that $\kappa(t)$
never vanishes. Then, there exists an arclength parameterization
$\bg:I\to\reals$ of a regular, oriented space curve, without points of
inflection, such that $\kappa(t)$ and $\tau(t)$ are the corresponding
curvature and torsion functions.  If $\hat{\bg}:I\to\reals$ is another
such space curve, then there exists a Euclidean isometry
$T:\reals^3\to\reals^3$ such that $\hat{\bg}(t) = T(\bg(t)).$
%%%%%
%%%%%
\end{document}
