\documentclass[12pt]{article}
\usepackage{pmmeta}
\pmcanonicalname{NashIsometricEmbeddingTheorem}
\pmcreated{2013-03-22 15:38:17}
\pmmodified{2013-03-22 15:38:17}
\pmowner{Simone}{5904}
\pmmodifier{Simone}{5904}
\pmtitle{Nash isometric embedding theorem}
\pmrecord{7}{37569}
\pmprivacy{1}
\pmauthor{Simone}{5904}
\pmtype{Theorem}
\pmcomment{trigger rebuild}
\pmclassification{msc}{53C20}
\pmclassification{msc}{53C42}
\pmclassification{msc}{57R40}
\pmclassification{msc}{58A05}

% this is the default PlanetMath preamble.  as your knowledge
% of TeX increases, you will probably want to edit this, but
% it should be fine as is for beginners.

% almost certainly you want these
\usepackage{amssymb}
\usepackage{amsmath}
\usepackage{amsfonts}

% used for TeXing text within eps files
%\usepackage{psfrag}
% need this for including graphics (\includegraphics)
%\usepackage{graphicx}
% for neatly defining theorems and propositions
%\usepackage{amsthm}
% making logically defined graphics
%%%\usepackage{xypic}

% there are many more packages, add them here as you need them

% define commands here
\begin{document}
Every compact $n$-dimensional Riemannian manifold $M$ of class $C^k$ 
($3\le k\le\infty$) can be $C^k$-isometrically imbedded in any small 
portion of a Euclidean space $\mathbb R^N$, where 
$$
  N=\frac 12 n(3n+11).
$$ 
Every non-compact $n$-dimensional Riemannian manifold $M$ of class $C^k$ ($3\le k\le\infty$) can be $C^k$-isometrically imbedded in any small portion of a Euclidean space $\mathbb R^N$, where 
$$
  N=(n+1)\frac 12 n(3n+11).
$$

The original proof due to Nash relying on an iteration scheme has been considerably simplified. For an overview, see \cite{nashsim}. 


\begin{thebibliography}{9}
\bibitem{nash} Nash, J. F., \emph{The imbedding problem for Riemannian manifold}, Ann. of Math. 63 (1956), 20--63 (MR 17, 782)
\bibitem{nashsim} D. Yang, \emph{Gunther's proof of Nash's isometric embedding theorem}, \PMlinkexternal{online}{http://www.math.poly.edu/~yang/papers/gunther.pdf}
\end{thebibliography}
%%%%%
%%%%%
\end{document}
