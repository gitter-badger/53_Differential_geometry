\documentclass[12pt]{article}
\usepackage{pmmeta}
\pmcanonicalname{ConjugatePoints}
\pmcreated{2013-03-22 14:35:40}
\pmmodified{2013-03-22 14:35:40}
\pmowner{rspuzio}{6075}
\pmmodifier{rspuzio}{6075}
\pmtitle{conjugate points}
\pmrecord{5}{36159}
\pmprivacy{1}
\pmauthor{rspuzio}{6075}
\pmtype{Definition}
\pmcomment{trigger rebuild}
\pmclassification{msc}{53B05}

% this is the default PlanetMath preamble.  as your knowledge
% of TeX increases, you will probably want to edit this, but
% it should be fine as is for beginners.

% almost certainly you want these
\usepackage{amssymb}
\usepackage{amsmath}
\usepackage{amsfonts}

% used for TeXing text within eps files
%\usepackage{psfrag}
% need this for including graphics (\includegraphics)
%\usepackage{graphicx}
% for neatly defining theorems and propositions
%\usepackage{amsthm}
% making logically defined graphics
%%%\usepackage{xypic}

% there are many more packages, add them here as you need them

% define commands here
\begin{document}
Let $M$ be a manifold on which a notion of geodesic is defined.  (For instance, $M$ could be a Riemannian manifold, $M$ could be a manifold with affine connection, or $M$ could be a Finsler space.)

Two distinct points, $P$ and $Q$ of $M$ are said to be conjugate points if there exist two or more distinct geodesic segments having $P$ and $Q$ as endpoints.

A simple example of conjugate points are the north and south poles of a sphere (endowed with the usual metric of constant curvature) --- every meridian is a geodesic segment having the poles as endpoints.
%%%%%
%%%%%
\end{document}
