\documentclass[12pt]{article}
\usepackage{pmmeta}
\pmcanonicalname{CharacteristicPolynomialOfASymplecticMatrixIsAReciprocalPolynomial}
\pmcreated{2013-03-22 15:33:18}
\pmmodified{2013-03-22 15:33:18}
\pmowner{matte}{1858}
\pmmodifier{matte}{1858}
\pmtitle{characteristic polynomial of a symplectic matrix is a reciprocal polynomial}
\pmrecord{7}{37455}
\pmprivacy{1}
\pmauthor{matte}{1858}
\pmtype{Theorem}
\pmcomment{trigger rebuild}
\pmclassification{msc}{53D05}
\pmrelated{ReciprocalPolynomial}
\pmrelated{CharacteristicPolynomialOfAOrthogonalMatrixIsAReciprocalPolynomial}

\endmetadata

% this is the default PlanetMath preamble.  as your knowledge
% of TeX increases, you will probably want to edit this, but
% it should be fine as is for beginners.

% almost certainly you want these
\usepackage{amssymb}
\usepackage{amsmath}
\usepackage{amsfonts}
\usepackage{amsthm}

\usepackage{mathrsfs}

% used for TeXing text within eps files
%\usepackage{psfrag}
% need this for including graphics (\includegraphics)
%\usepackage{graphicx}
% for neatly defining theorems and propositions
%
% making logically defined graphics
%%%\usepackage{xypic}

% there are many more packages, add them here as you need them

% define commands here

\newcommand{\sR}[0]{\mathbb{R}}
\newcommand{\sC}[0]{\mathbb{C}}
\newcommand{\sN}[0]{\mathbb{N}}
\newcommand{\sZ}[0]{\mathbb{Z}}

 \usepackage{bbm}
 \newcommand{\Z}{\mathbbmss{Z}}
 \newcommand{\C}{\mathbbmss{C}}
 \newcommand{\F}{\mathbbmss{F}}
 \newcommand{\R}{\mathbbmss{R}}
 \newcommand{\Q}{\mathbbmss{Q}}



\newcommand*{\norm}[1]{\lVert #1 \rVert}
\newcommand*{\abs}[1]{| #1 |}



\newtheorem{thm}{Theorem}
\newtheorem{defn}{Definition}
\newtheorem{prop}{Proposition}
\newtheorem{lemma}{Lemma}
\newtheorem{cor}{Corollary}
\begin{document}
\begin{thm}
The characteristic polynomial of a symplectic matrix is a reciprocal polynomial.
\end{thm}

\begin{proof}
Let $A$ be the symplectic matrix, and let
$p(\lambda) = \det(A-\lambda I)$ be
its characteristic polynomial. We wish to prove that
$$
p(\lambda) = \pm \lambda^n p(1/\lambda).
$$
By definition, $AJA^T=J$ where $J$ is the matrix
$$
J=\left( \begin{array}{cc}
0 & I \\
-I & 0
\end{array} \right).
$$
Since $A$ and $J$ are symplectic matrices, their determinants are $1$, and
\begin{eqnarray*}
p(\lambda) &=& \det (AJ - \lambda J) \\
           &=&\det (AJ - \lambda AJA^T) \\
           &=&\det (-\lambda A) \det (J) \det (-\frac{1}{\lambda} J + JA^T) \\
           &=&\pm \lambda^n \det (A-\frac{1}{\lambda} I ).
\end{eqnarray*}
as claimed.
\end{proof}
%%%%%
%%%%%
\end{document}
