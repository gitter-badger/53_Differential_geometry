\documentclass[12pt]{article}
\usepackage{pmmeta}
\pmcanonicalname{Loxodrome}
\pmcreated{2013-03-22 19:11:02}
\pmmodified{2013-03-22 19:11:02}
\pmowner{pahio}{2872}
\pmmodifier{pahio}{2872}
\pmtitle{loxodrome}
\pmrecord{11}{42092}
\pmprivacy{1}
\pmauthor{pahio}{2872}
\pmtype{Definition}
\pmcomment{trigger rebuild}
\pmclassification{msc}{53A05}
\pmclassification{msc}{53A04}
\pmclassification{msc}{26B05}
\pmclassification{msc}{26A24}
\pmdefines{meridian}

% this is the default PlanetMath preamble.  as your knowledge
% of TeX increases, you will probably want to edit this, but
% it should be fine as is for beginners.

% almost certainly you want these
\usepackage{amssymb}
\usepackage{amsmath}
\usepackage{amsfonts}

% used for TeXing text within eps files
%\usepackage{psfrag}
% need this for including graphics (\includegraphics)
%\usepackage{graphicx}
% for neatly defining theorems and propositions
 \usepackage{amsthm}
% making logically defined graphics
%%%\usepackage{xypic}
\usepackage{pstricks}
\usepackage{pst-plot}

% there are many more packages, add them here as you need them

% define commands here

\theoremstyle{definition}
\newtheorem*{thmplain}{Theorem}

\begin{document}
\PMlinkescapeword{constant}

Think in $\mathbb{R}^3$ a sphere with radius $r$ and two antipodal points of it wich we call the North pole and the South pole.\, \emph{Meridians} are great circles passing through the \PMlinkescapetext{poles}.\, A \emph{loxodrome} is a curve on the sphere intersecting all meridians at the same angle.\\


Let 
\begin{align*}
\begin{cases}
x \;=\; r\sin{u}\cos{v}\\
y \;=\; r\sin{u}\sin{v}\\
z \;=\; r\cos{u}
\end{cases}
\end{align*}
be a parametric presentation of the sphere (cf. the spherical coordinates).\, We will show that
\begin{align}
bv \;=\; \ln\tan\frac{u}{2}+c,
\end{align}
where $b$ and $c$ are constants, is an equation of loxodromes in the Gaussian coordinates $u,\,v$.\\

We denote\, $\displaystyle\frac{1}{b} := a$,\, whence the equation of the family (1) in the parameter plane reads
\begin{align}
v \;=\; a\ln\tan\frac{u}{2}+ac \;:=\; v(u).
\end{align}
When we denote also the position vector of a point of the sphere by
$$\vec{k} \;=\; \vec{k}(u,\,v) \;:=\; (r\sin{u}\cos{v},\,r\sin{u}\sin{v},\,r\cos{u}),$$
we have the tangent vector of a curve (1) on the sphere:
$$\vec{t} \;:=\; \frac{d}{du}\vec{k}(u,\,v(u)) \;=\; \vec{k}'_u+\vec{k}'_v\!\cdot\!v'(u).$$
Since
$$\vec{k}'_u \;=\; (r\cos{u}\cos{v},\,r\cos{u}\sin{v},\,-r\sin{u}), \quad 
\vec{k}'_v \;=\; (-r\sin{u}\sin{v},\,r\sin{u}\cos{v},\,0)$$
and since
$$v'(u) \;=\; \frac{a}{\tan{\frac{u}{2}}}\cdot\frac{1}{\cos^2\frac{u}{2}}\cdot\frac{1}{2} 
\;=\; \frac{a}{2\sin\frac{u}{2}\cos\frac{u}{2}} \;=\; \frac{a}{\sin{u}},$$
we can write the tangent vector of the curve as
$$\vec{t} \;=\; r\cdot(\cos{u}\cos{v}-a\sin{v},\,\cos{u}\sin{v}+a\cos{v},\,-\sin{u}).$$
For a tangent vector of a meridian, the partial derivative $\vec{k}'_u$ may be taken.\, 
Thus we obtain the value
$$\cos(\vec{t},\,\vec{k}'_u) \;=\; \frac{\vec{t}\cdot\vec{k}'_u}{\left|\vec{t}\right||\vec{k}'_u|} 
\;=\; \frac{1}{\sqrt{1\!+\!a^2}} \;=\; \frac{b}{\sqrt{b^2\!+\!1}},$$
which is a constant.\, It means that the angle $\alpha$ between the curve (1) and a meridian is constant.\\

\begin{center}
\begin{pspicture}(-1,-1)(5,4.5)
\psline{->}(0,0)(0,3.5)
\psline{->}(-0.9,0)(4.5,0)
\psline{->}(0,0)(4,2.67)
\psline(-0.05,2)(3.07,2)
\psline(0,1.85)(0.15,1.85)(0.15,2)
\psarc(0,0){0.25}{40}{90}
\rput(1.5,2.2){$1$}
\rput(-0.2,1.1){$b$}
\rput(2.25,0.9){$\sqrt{b^2\!+\!1}$}
\rput(0.34,3.6){$\vec{k}'_u$}
\rput(4.7,0.25){$\vec{k}'_v$}
\rput(4.1,2.9){$\vec{t}$}
\rput(0.25,0.4){$\alpha$}
\rput(-1,-1){.}
\rput(5,4.3){.}
\end{pspicture}
\end{center}



Pictures in \PMlinkexternal{Wiki}{http://hu.wikipedia.org/wiki/Loxodroma}





%%%%%
%%%%%
\end{document}
