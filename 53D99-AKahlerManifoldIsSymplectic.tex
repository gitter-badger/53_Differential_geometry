\documentclass[12pt]{article}
\usepackage{pmmeta}
\pmcanonicalname{AKahlerManifoldIsSymplectic}
\pmcreated{2013-03-22 16:07:54}
\pmmodified{2013-03-22 16:07:54}
\pmowner{cvalente}{11260}
\pmmodifier{cvalente}{11260}
\pmtitle{a K\"ahler manifold is symplectic}
\pmrecord{15}{38203}
\pmprivacy{1}
\pmauthor{cvalente}{11260}
\pmtype{Result}
\pmcomment{trigger rebuild}
\pmclassification{msc}{53D99}
\pmrelated{KahlerManifold}

\endmetadata

% this is the default PlanetMath preamble.  as your knowledge
% of TeX increases, you will probably want to edit this, but
% it should be fine as is for beginners.

% almost certainly you want these
\usepackage{amssymb}
\usepackage{amsmath}
\usepackage{amsfonts}

% used for TeXing text within eps files
%\usepackage{psfrag}
% need this for including graphics (\includegraphics)
%\usepackage{graphicx}
% for neatly defining theorems and propositions
%\usepackage{amsthm}
% making logically defined graphics
%%%\usepackage{xypic}

% there are many more packages, add them here as you need them

% define commands here

\begin{document}
Let $\omega(X,Y) = g(JX,Y)$ on a K\"ahler manifold. We will prove that $\omega$ is a symplectic form.

\begin{itemize}

\item $\omega$ is anti-symmetric

$\omega(X,Y) = g(JX,Y) = g(Y,JX) = g(JY, J^2 X) = g(JY,-X) = -g(JY,X) = -\omega(Y,X)$. Here we used the fact that $g$ is an Hermitian tensor on a K\"ahler manifold ($g(X,Y) = g(JX, JY)$)

\item $\omega$ is linear

Due to anti-symmetry, we just need to check linearity on the second slot. Since $g(JX,\cdot)$ is by definition linear, $\omega$ will also be linear.

\item $\omega$ is non degenerate

On a given point on the manifold, pick a non null vector $X$, $\alpha_X(\cdot) = \omega(X, \cdot ) = g(JX, \cdot)$. Since $g$ is non-degenerate\footnote{no vector but the null vector is orthogonal to every other vector}, $\alpha$ is also non-degenerate (for all $X$). $\omega$ is thus non degenerate.

\item $\omega$ is closed

First note that 
\begin{eqnarray}
X(\omega(Y,Z)) &=& \nabla_X (\omega(Y,Z)) \nonumber \\
 &=& \nabla_X (g(JY,Z)) \nonumber \\
 &=& g(\nabla_X(JY),Z) + g(JY, \nabla_X Z) \nonumber \\
 &=& g(J\nabla_X Y, Z) + g(JY, \nabla_X Z) \nonumber \\
 &=& \omega(\nabla_X Y, Z) + \omega(Y, \nabla_X Z) \nonumber
\end{eqnarray}

Here we used the fact that both $g$ and $J$ are covariantly constant ($\nabla g = 0$ and $\nabla J = 0$)

We aim to prove that $d \omega = 0$ which is equivalent to proving $(d \omega)(X,Y,Z) = 0$ for all vector fields $X,Y,Z$.

Since this is a tensorial identity, WLOG we can assume that at a specific point $p$ in the K\"ahler manifold $[X,Y]_p = [Y,Z]_p = [Z,X]_p =0$ and prove the indentity for these vector fields\footnote{in particular this works for the canonical base of $T_p M$ associated with a local coordinate system}.

Consider $X,Y,Z$ with the previous commutation relations at $p$, using the formulas for differential forms of small valence:

\begin{eqnarray}
(d \omega)(X,Y,Z) &=& X(\omega(Y,Z)) + Y(\omega(Z,X) + Z(\omega(X,Y)) ) \nonumber \\
        &=& \omega(\nabla_X Y,Z) + \omega(Y,\nabla_X Z) + \nonumber \\
        &&  \omega(\nabla_Y Z,X) + \omega(Z,\nabla_Y X) + \nonumber \\
        && \omega(\nabla_Z X,Y) + \omega(X,\nabla_Z Y) \nonumber \\
        &=& \omega(\nabla_X Y - \nabla_Y X, Z) + \omega(\nabla_Y Z - \nabla_Z Y, X) + \omega(\nabla_Z X - \nabla_X Z, Y) \nonumber
\end{eqnarray}
The Levi-Civita connection is torsion-free, $\nabla_X Y - \nabla_Y X = [X,Y]$ thus:

$$(d \omega)(X,Y,Z) = \omega([X,Y],Z) + \omega([Y,Z],X) + \omega([Z,X],Y)$$

And since all the commutators are null at $p$ (by assumption) we get that:

$$ (d \omega)(X,Y,Z) = 0 $$

$\omega$ is therefore closed.

\end{itemize}

%%%%%
%%%%%
\end{document}
