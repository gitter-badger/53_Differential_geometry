\documentclass[12pt]{article}
\usepackage{pmmeta}
\pmcanonicalname{Ellipse}
\pmcreated{2013-03-22 15:18:10}
\pmmodified{2013-03-22 15:18:10}
\pmowner{matte}{1858}
\pmmodifier{matte}{1858}
\pmtitle{ellipse}
\pmrecord{33}{37102}
\pmprivacy{1}
\pmauthor{matte}{1858}
\pmtype{Definition}
\pmcomment{trigger rebuild}
\pmclassification{msc}{53A04}
\pmclassification{msc}{51N20}
\pmclassification{msc}{51-00}
\pmrelated{SqueezingMathbbRn}
\pmrelated{Ellipsoid}
\pmdefines{major axis}
\pmdefines{minor axis}
\pmdefines{major semi-axis}
\pmdefines{minor semi-axis}
\pmdefines{focus}
\pmdefines{foci}
\pmdefines{aphelium}
\pmdefines{perihelium}
\pmdefines{eccentric anomaly}
\pmdefines{focal radius}
\pmdefines{focal radii}

% this is the default PlanetMath preamble.  as your knowledge
% of TeX increases, you will probably want to edit this, but
% it should be fine as is for beginners.

% almost certainly you want these
\usepackage{amssymb}
\usepackage{amsmath}
\usepackage{amsfonts}
\usepackage{amsthm}

\usepackage{mathrsfs}
\usepackage{pstricks}
\usepackage{pst-plot}

% used for TeXing text within eps files
%\usepackage{psfrag}
% need this for including graphics (\includegraphics)
%\usepackage{graphicx}
% for neatly defining theorems and propositions
%
% making logically defined graphics
%%%\usepackage{xypic}

% there are many more packages, add them here as you need them

% define commands here

\newcommand{\sR}[0]{\mathbb{R}}
\newcommand{\sC}[0]{\mathbb{C}}
\newcommand{\sN}[0]{\mathbb{N}}
\newcommand{\sZ}[0]{\mathbb{Z}}

 \usepackage{bbm}
 \newcommand{\Z}{\mathbbmss{Z}}
 \newcommand{\C}{\mathbbmss{C}}
 \newcommand{\F}{\mathbbmss{F}}
 \newcommand{\R}{\mathbbmss{R}}
 \newcommand{\Q}{\mathbbmss{Q}}



\newcommand*{\norm}[1]{\lVert #1 \rVert}
\newcommand*{\abs}[1]{| #1 |}



\newtheorem{thm}{Theorem}
\newtheorem{defn}{Definition}
\newtheorem{prop}{Proposition}
\newtheorem{lemma}{Lemma}
\newtheorem{cor}{Corollary}
\begin{document}
\PMlinkescapeword{equation}
\PMlinkescapeword{terms}

An \emph{ellipse} that is centered at the origin is the curve in the plane determined by
\begin{equation}
\left(\frac{x}{a}\right)^2 + \left(\frac{y}{b}\right)^2 = 1,
\end{equation}
where $a,b>0$.

Below is a graph of the ellipse $\displaystyle \left(\frac{x}{3}\right)^2+\left(\frac{y}{2}\right)^2=1$:

\begin{center}
\begin{pspicture}(-3.2,-2.2)(3.5,2.5)
\psaxes{->}(0,0)(-3.2,-2.2)(3.5,2.5)
\psellipse(0,0)(3,2)
\rput[b](3.5,-0.5){$x$}
\rput[r](0,2.5){$y$}
\end{pspicture}
\end{center}

The \emph{major axis} of an ellipse is the longest line segment whose endpoints are on the ellipse.  The \emph{minor axis} of an ellipse is the shortest line segment through the midpoint of the ellipse whose endpoints are on the ellipse.

In the first equation given above, if $a=b$, the ellipse reduces to a circle of radius $a$, whereas if $a>b$ (as in the graph above), $a$ is said to be the {\em major semi-axis} length and $b$ the {\em minor semi-axis} length; \PMlinkname{i.e.}{Ie}, the lengths of the major axis and minor axis are $2a$ and $2b$, respectively.

More generally, given any two points $p_1$ and $p_2$ in the (Euclidean) plane and any real number $r$, let $E$ be the set of points $p$ having the property that the sum of the distances from $p$ to $p_1$ and $p_2$ is $r$; i.e.,
$$E = \left\{ p\, |\, r=\lvert p-p_1\rvert + \vert p-p_2\rvert\right\}.$$
In terms of the geometric look of $E$, there are three possible scenarios for $E$: $E=\varnothing$, $E=\overline{p_1p_2}$, the line segment with end-points $p_1$ and $p_2$, or $E$ is an ellipse.  Points $p_1$ and $p_2$ are called \emph{foci} of the ellipse; the line segments connecting a point of the ellipse to the foci are the {\em focal radii} belonging to that point.  When $p_1=p_2$ and $r>0$, $E$ is a circle.  Under appropriate linear transformations (a translation followed by a rotation), $E$ has an algebraic appearance expressed in (1).  

In polar coordinates, the ellipse is parametrized as
\begin{eqnarray*}
   x(t) &=& a\cos t, \\
   y(t) &=& b\sin t,  \quad t\in[0,\,2\pi).
\end{eqnarray*}
If\, $a>b$,\, then $t$ is the {\em eccentric anomaly}; i.e., the polar angle of the point on the circumscribed circle having the same abscissa as the point of the ellipse.

\subsubsection*{Properties}
\begin{enumerate}
\item If\, $a > b$,\, the foci of the ellipse (1) are on the $x$-axis with distances $\sqrt{a^2-b^2}$ from the origin.\, The constant sum of the \PMlinkescapetext{focal radii} of a point $p$ is equal to $2a$.
\item The normal line of the ellipse at its point $p$ halves the angle between the focal radii drawn from $p$.
\item The area of an ellipse is $\pi a b$.  (See \PMlinkname{this page}{AreaOfPlaneRegion}.)
\item The length of the perimeter of an ellipse can be expressed using an elliptic integral. 
\end{enumerate}

\subsubsection*{Eccentricity}
By definition, the {\em eccentricity} $\epsilon$ ($0\leq\epsilon<1$) of the ellipse is given by 
\begin{equation*}
\epsilon=\frac{\sqrt{a^2-b^2}}{a}\cdot
\end{equation*}
For $\epsilon=0$, the ellipse reduces to a circle. Further, $b=a\sqrt{1-\epsilon^2}$, and by assuming that foci are located on $x$-axis, $p_1$ on $x<0$ and $p_2$ on $x>0$, then $|O-p_1|=|O-p_2|=\epsilon a$, where $O(0,0)$ is the origin of the rectangular coordinate system.  
\subsubsection*{Polar equation of the ellipse}
By translating the $y$-axis towards the focus $p_1$, we have
\begin{eqnarray*}
x' &=& x+\epsilon a,  \\
y' &=& y,
\end{eqnarray*}
but from (1) we get
\begin{equation}
\left(\frac{x'-\epsilon a}{a}\right)^2 + \left(\frac{y'}{b}\right)^2 = 1.
\end{equation}
By using the transformation equations to polar coordinates
\begin{eqnarray*}
x' &=& r\cos\theta, \\
y' &=& r\sin\theta,
\end{eqnarray*}
and through (2) we arrive at the polar equation
\begin{equation}
r(\theta)=\frac{(1-\epsilon^2)a}{1-\epsilon\cos\theta}\cdot
\end{equation}
This equation allows us to determine some additional properties about the ellipse: 
\begin{align*}
r_{max}:=r(0)=(1+\epsilon)a, \qquad \text{which is called the {\em aphelium}}; \\
r_{min}:=r(\pi)=(1-\epsilon)a, \qquad \text{which is called the {\em perihelium}}.
\end{align*}
Hence, the general definition of the ellipse expressed above shows that $r_{min}+r_{max}=2a$ and also that the arithmetic mean $\displaystyle \frac{r_{min}+r_{max}}{2}=a$ corresponds to the major semi-axis, while the geometric mean $\sqrt{r_{min}r_{max}}=b$ corresponds to the minor semi-axis of the ellipse. Likewise, if $\theta_\epsilon$ is the angle between the {\em polar axis} $x'$ and the radial distance $|B-p_1|$, where $B(0,b)$ is the point of the ellipse over the $y$-axis, then we get the useful equation $\cos\theta_\epsilon=\epsilon$.
%%%%%
%%%%%
\end{document}
