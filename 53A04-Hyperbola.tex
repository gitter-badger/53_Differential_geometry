\documentclass[12pt]{article}
\usepackage{pmmeta}
\pmcanonicalname{Hyperbola}
\pmcreated{2013-03-22 17:07:28}
\pmmodified{2013-03-22 17:07:28}
\pmowner{pahio}{2872}
\pmmodifier{pahio}{2872}
\pmtitle{hyperbola}
\pmrecord{34}{39427}
\pmprivacy{1}
\pmauthor{pahio}{2872}
\pmtype{Topic}
\pmcomment{trigger rebuild}
\pmclassification{msc}{53A04}
\pmclassification{msc}{51N20}
\pmclassification{msc}{51-00}
\pmrelated{ConicSection}
\pmrelated{EuclideanDistance}
\pmrelated{TransitionToSkewAngledCoordinates}
\pmrelated{TangentLine}
\pmrelated{HyperbolicRotation}
\pmrelated{GraphOfEquationXyConstant}
\pmrelated{PropertiesOfEllipse}
\pmrelated{PropertiesOfParabola}
\pmrelated{DecomposableCurve}
\pmdefines{hyperbola}
\pmdefines{focus}
\pmdefines{foci}
\pmdefines{focal radius}
\pmdefines{focal radii}
\pmdefines{apex of hyperbola}
\pmdefines{apices of hyperbola}
\pmdefines{transversal axis}
\pmdefines{conjugate axis}
\pmdefines{rectangular hyperbola}

\endmetadata

% this is the default PlanetMath preamble.  as your knowledge
% of TeX increases, you will probably want to edit this, but
% it should be fine as is for beginners.

% almost certainly you want these
\usepackage{amssymb}
\usepackage{amsmath}
\usepackage{amsfonts}

% used for TeXing text within eps files
%\usepackage{psfrag}
% need this for including graphics (\includegraphics)
%\usepackage{graphicx}
% for neatly defining theorems and propositions
 \usepackage{amsthm}
% making logically defined graphics
%%%\usepackage{xypic}
\usepackage{pstricks}
\usepackage{pst-plot}

% there are many more packages, add them here as you need them

% define commands here

\theoremstyle{definition}
\newtheorem*{thmplain}{Theorem}

\begin{document}
\PMlinkescapeword{contains} \PMlinkescapeword{branches}

A {\em hyperbola} is the locus of points $P$ in the Euclidean plane such that the distances of $P$ from two \PMlinkescapetext{fixed points} (the foci $F_1$ and $F_2$) differ from each other by a constant amount ($\pm2a$).\, The line segments connecting a point of the hyperbola to the foci are called {\em focal radii}.

We obtain the simplest equation for the hyperbola by choosing the foci on the other coordinate axis and equidistant ($= c > a > 0$) from the origin.\; Let\; $F_1 = (-c,\,0)$\, and $F_2 = (c,\,0)$.\, Then the locus condition for the point\, $P = (x,\,y)$\, of the hyperbola is
$$\sqrt{(x+c)^2+y^2}-\sqrt{(x-c)^2+y^2} = \pm2a.$$
The simplifying of this, via two squarings, yields the equation of the hyperbola
\begin{align}
     \frac{x^2}{a^2}\!-\!\frac{y^2}{b^2} \;=\; 1.
\end{align}
Here we have denoted\, $c^2\!-\!a^2 = b^2$\, where\, $b > 0$.

Since the equation (1) contains only the squares of $x$ and $y$ we can infer that the hyperbola is symmetric with respect to the coordinate axes and the origin; this is naturally clear on grounds of the definition of hyperbola, too.

Solving (1) for $y$ we get
\begin{align}
   y \;=\; \pm\frac{b}{a}\sqrt{x^2\!-\!a^2}.
\end{align}
This shows that $y$ is real only for\, $x \geqq a$;\, for\, $x = \pm a$\, we have\, $y = 0.$\, When 
$|x|$ increases from $a$ to infinity, $|y|$ increases from $0$ to infinity.\, So we see that the hyperbola consists of two distinct branches from which the one is to the \PMlinkescapetext{right} from the line\, $x = a$\, and the other to the left from the line\, $x = -a$.\, These lines touch the branches at the points\, $(a,\,0)$\, and\, $(-a,\,0)$,\, which are called the {\em apices} of the hyperbola.

\begin{center}
\begin{pspicture}(-5.5,-4.5)(5.5,4)
\psaxes[Dx=1,Dy=1]{->}(0,0)(-4.5,-3.5)(4.5,3.5)
\rput(0.3,3.3){$y$}
\rput(4.4,0.4){$x$}
\psplot{1.5}{4.2}{x 2 exp -2.25 add 0.5 exp 1.5 div}
\psplot{-1.5}{-4.2}{x 2 exp -2.25 add 0.5 exp 1.5 div}
\psplot{1.5}{4.2}{x 2 exp -2.25 add 0.5 exp -1.5 div}
\psplot{-1.5}{-4.2}{x 2 exp -2.25 add 0.5 exp -1.5 div}
\rput(0,-4.5){$\mbox{Graph of \,}4x^2-9y^2=9$}
\rput(-5.5,-4.5){.}
\rput(5.5,4){.}
\end{pspicture}
\end{center}

The line segment connecting the apices is the {\em transversal axis} of the hyperbola.\, The line segment on the $y$-axis from $-b$ to $b$ is the {\em conjugate axis} of the hyperbola.\, By the Pythagorean theorem, the equation\, 
$b^2 = c^2-a^2$\, shows that the distance between an end of the transversal and an end of the conjucate axis is equal to $c$.

Let's consider the part 
$$y = \frac{b}{a}\sqrt{x^2-a^2}$$
of the hyperbola situated in the first quadrant ($x > a$) and the line
$$y \;=\; \frac{b}{a}x.$$
The difference of their ordinates corresponding a same abscissa $x$ may be written
$$\Delta \;:=\; \frac{b}{a}(x\!-\!\sqrt{x^2\!-\!a^2}) \;=\; \frac{ab}{x\!+\!\sqrt{x^2\!-\!a^2}}\;\;\;(> 0).$$
But\, $\Delta \to 0$\, as\, $x \to \infty$,\, whence this \PMlinkescapetext{branch} of the hyperbola approaches unlimitedly from below the line.\, Accordingly, the line\, $y = \frac{b}{a}x$\, is an asymptote of our curve.\, By the symmetry, the hyperbola (1) has two asymptotes
\begin{align}
y \;=\; \pm\frac{b}{a}x.
\end{align}
The asymptotes are easy to draw, since they are the lengthened diagonals of the rectangle whose sides are on the lines\, $x = \pm a$\, and\, $y = \pm b$.\, The hyperbola may be sketched by utilising that rectangle and the asymptotes.

\begin{center}
\begin{pspicture}(-5.5,-4.5)(5.5,4)
\psaxes[Dx=10,Dy=10]{->}(0,0)(-4.5,-3.5)(4.5,3.5)
\rput(-0.3,3.5){$y$}
\rput(4.55,-0.3){$x$}
\psline{-}(-4.5,-3)(4.5,3)
\psline{-}(-4.5,3)(4.5,-3)
\pspolygon(-1.5,-1)(1.5,-1)(1.5,1)(-1.5,1)
\rput(-1.85,-0.2){$-a$}
\rput(-0.3,-1.18){$-b$}
\rput(1.67,-0.2){$a$}
\rput(-0.17,1.18){$b$}
\rput(0.7,0.6){$c$}
\psplot{1.5}{4.5}{x 2 exp -2.25 add 0.5 exp 1.5 div}
\psplot{-1.5}{-4.5}{x 2 exp -2.25 add 0.5 exp 1.5 div}
\psplot{1.5}{4.5}{x 2 exp -2.25 add 0.5 exp -1.5 div}
\psplot{-1.5}{-4.5}{x 2 exp -2.25 add 0.5 exp -1.5 div}
\rput(0,-4.5){$\mbox{Graph of \,}\frac{x^2}{a^2}-\frac{y^2}{b^2}=1$\mbox{\, with asymptotes \,}$y = \pm\frac{b}{a}x$}
\rput(-5.5,-4.5){.}
\rput(5.5,4){.}
\end{pspicture}
\end{center}

Both asymptotes form with the transversal axis an angle whose tangent is equal to $\frac{b}{a}$.\, This equals 1, when the transversal axis and the conjugate axis are equal ($a = b$); then the rectangle is a square and one speaks of a {\em rectangular hyperbola}.  See also the entry transition to skew-angled coordinates.


\begin{thebibliography}{8}
\bibitem{LL}{\sc L. Lindel\"of}: {\em Analyyttisen geometrian oppikirja}.\, Kolmas painos.\, Suomalaisen Kirjallisuuden Seura, Helsinki (1924).
\end{thebibliography}

%%%%%
%%%%%
\end{document}
